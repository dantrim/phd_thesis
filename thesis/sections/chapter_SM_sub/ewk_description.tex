\section{The Electroweak Theory}
\label{sec:ewk_description}

It was the work of Glashow, Weinberg, and Salam (GWS) that ultimately put forth a
consistent picture of the chiral weak force and
its unification with electromagnetism~\cite{Glashow:1961tr,Weinberg:1967tq,Salam:1968rm}.
As a result, the theory of particles and fields that respect the \SUewk~gauge
invariance of the SM is sometimes referred to as `GWS theory', 
but is more typically known as the electroweak theory. Since all matter particles
are subject to the electroweak interaction, but only a subset of the particles  that
have color charge (the quarks) are subject to the strong interaction described by QCD, the study of the SM can essentially
be partitioned into two parts: the part that deals with the dynamics and interactions of
colored objects (the `QCD part', $\mathcal{L}_{\text{QCD}}$) and the part that deals with electroweak
interactions, including the Higgs (the `Electroweak part`, $\mathcal{L}_{\text{Electroweak}}$).
Given the broad reach of the electroweak interaction,
in the early days GWS theory was considered the heart of the SM and why
GWS were awarded the Nobel prize in 1979.\footnote{Actually, the acceptance of the GWS theory as the
de-facto SM of the time was not widely held until some years after its publication, when t'Hooft
proved that it was renormalizable~\cite{tHooft:1971akt,tHooft:1971qjg}.
Such a complete understanding in the QCD sector would not come until later, in the
1970's, with the work of Gross, Wilczek, and Politzer~\cite{GrossWilczek,Politzer}.}
In this section we will focus on the \SUewk~portion of \SML.

The first thing to remember is that the electroweak theory is \textit{chiral}, i.e., it distinguishes
between left- and right-chiral fermion fields.
For conceptual clarity, it can be useful to take the massless (relativistic) limit of fermions to
get an idea of what chirality represents.
For a massless fermion field, the chirality is equivalent to the perhaps more-familiar \textit{helicity},
defined as the projection of its spin onto its momentum (direction of motion).
The helicity of left-handed (right-handed) massless fermions is positive (negative), meaning
that their spin is parallel (anti-parallel) to its momentum.
Fermion fields, then, are commonly defined inclusive of their handedness, with the left-
and right-handed components projected out using the $P_L$ and $P_R$ projection operators,
\begin{equation}
	f_{\text{L}} = P_L f = \frac{1}{2}(1- \gamma_5)f \hspace{0.3cm}\text{and}\hspace{0.3cm} f_{\text{R}} = P_R f = \frac{1}{2}(1+\gamma_5)f.
	\label{eq:chiral_projection}
\end{equation}

Focusing only on the first generation of the leptons (the discussion holds equally well
for the second and third generations, as well as for the quarks), we can gather the
\Uone~terms of Equation~\ref{eq:sm_lagrangian},
\begin{align}
    -\mathcal{L}_{\text{ferm}}(\mathcal{U}(1), \text{leptons}) &= \bar{L} i \gamma^{\mu} (i g_1 \frac{Y_L}{2} B_{\mu})L + \bar{e}_R i \gamma^{\mu} (i g_1 \frac{Y_R}{2} B_{\mu}) e_R \nonumber \\
    &= \frac{g_1}{2} [ Y_L ( \bar{\nu}_L \gamma^{\mu} \nu_L + \bar{e}_L \gamma^{\mu} e_L) + Y_R \bar{e}_R \gamma^{\mu} e_R ] B_{\mu},
    \label{eq:ferm_L_u1}
\end{align}
where $L = (\nu_L, e_L)$ is used in going from the first to second line. 
Likewise, gathering the associated \SUtwo~terms and noting that $\tau^i W^i$ is a
$2\times2$ matrix since the $\tau^i$ are \SUtwo~generators (e.g. the Pauli matrices) gives,

\begin{equation}
	\begin{multlined}
		\hspace{-2cm}-\mathcal{L}_{\text{ferm}}(\mathcal{SU}(2), \text{leptons}) =  \bar{L}\, i \gamma^{\mu} [i g_2 \frac{\tau^i}{2} W_{\mu}^i ] \,L \\
		= -\frac{g_2}{2} \left[ \bar{\nu}_L \gamma^{\mu} \nu_L W_{\mu}^0 - \sqrt{2}  \bar{\nu}_L \gamma^{\mu} e_L W_{\mu}^+ - \sqrt{2} \bar{e}_L \gamma^{\mu} \nu_L W_{\mu}^- - \bar{e}_L \gamma^{\mu} e_L W_{\mu}^0 \right],
	\end{multlined}
	\label{eq:ferm_L_su2}
\end{equation}
where we have used the following re-definitions of the \SUtwo~gauge fields,
\begin{equation}
	W_{\mu}^+ = \frac{1}{\sqrt{2}} \left( -W_{\mu}^1 + i W_{\mu}^2 \right) \hspace{1cm} W_{\mu}^- = \frac{1}{\sqrt{2}} \left( -W_{\mu}^1 - i W_{\mu}^2 \right) \hspace{1cm} W_{\mu}^0 = W_{\mu}^3.
\end{equation}

In principle, Equations~\ref{eq:ferm_L_u1} and \ref{eq:ferm_L_su2} describe completely
all electroweak interactions
between matter and the gauge fields of \SUewk. We would like to make the correspondence
between these equations and what we know to empirically exist: the electromagnetic interaction
and the presence of a massive, charged mediator of the weak nuclear force responsible
for nuclear $\beta$-decay, for example. From the theory of QED, it is a-priori known what the
form of the interaction of the neutral photon and the electron should look like.
Inspecting all charge-preserving (i.e. neutral) terms of Equation~\ref{eq:ferm_L_u1} and \ref{eq:ferm_L_su2},
it can be seen that the \fieldB$_{\mu}$ and \fieldWzero$_{\mu}$ fields have this expected
fermion coupling, suggesting a re-definition as follows,
\begin{equation}
	\left( \begin{matrix} A_{\mu} \\ Z_{\mu} \end{matrix} \right) = \left( \begin{matrix} \cos \theta_W & \sin \theta_W \\ -\sin \theta_W & \cos \theta_W \end{matrix} \right) \left( \begin{matrix} B_{\mu} \\ W_{\mu}^0 \end{matrix} \right),
	\label{eq:su2rotation}
\end{equation}
where we have used $Y_L = -1$ and define the relations between the \SUtwo~and \Uone~couplings as,
\begin{equation}
\sin \theta_W = \frac{g_1}{\sqrt{g_1^2 + g_2^2}} \hspace{1cm} \cos \theta_W = \frac{g_2}{\sqrt{g_1^2 + g_2^2}} \hspace{1cm} e = \frac{g_1 g_2}{\sqrt{g_1^2 + g_2^2}}.
\label{eq:weinberg_angles}
\end{equation}
The angle $\theta_W$  is known as the \textit{Weinberg angle}, or the \textit{weak mixing angle}.
It quantifies the amount of \textit{gauge mixing} that occurs between the neutral
\SUewk~gauge fields, \fieldB$_{\mu}$ and \fieldWzero$_{\mu}$.

The above algebra allows to re-write the electroweak Lagrangian, now
describing interactions between the fermions and the newly defined $A_{\mu}$, \fieldZ,
and \fieldWpm, as,
\begin{equation}
\begin{multlined}
\hspace{-0.9cm}\mathcal{L}_{\text{ferm, first-gen.}} = \underbrace{\sum\limits_{f \in \nu_e, e, u, d} e Q_f
\left(\bar{f}\gamma^{\mu} f \right) A_{\mu}}_\text{Neutral, $\sim$ EM}  \\
+ \underbrace{\frac{g_2}{\cos \theta_W} \sum\limits_{f \in \nu_e, e, u, d} \left[ \,\right.\bar{f}_L \gamma^{\mu} f_L
\left( T_f^3 -  Q_f \sin^2 \theta_W \right)
+ \bar{f}_R \gamma^{\mu} f_R \left(-Q_f \sin^2 \theta_W \right) \left. \right]\, Z_{\mu}}_\text{Neutral weak interaction} \\
+ \underbrace{\frac{g_2}{\sqrt{2}} \left[ \right. \left( \bar{u}_L \gamma^{\mu} d_L + \bar{\nu}_{e,L} \gamma^{\mu} e_L \right) \, W_{\mu}^+ + h.c. \left. \right]}_\text{Charged weak interaction}
\end{multlined}
\label{eq:ewk_L_za}
\end{equation}
The first term of Equation~\ref{eq:ewk_L_za} has the expected form expected from QED, describing the
interaction between a neutral gauge boson and fermions and allows us to interpret
the parameter $e$, introduced in Equation~\ref{eq:weinberg_angles}, as the coupling
of electromagnetism (electric charge) with $Q_f$ as the fermion's electric charge quantum number
(in units of $e$). The $A_{\mu}$ arrived at via the gauge mixing of Equation~\ref{eq:su2rotation}
then must correspond to the photon of electromagnetism.
The second term of Equation~\ref{eq:ewk_L_za} predicts the existence of an additional
neutral gauge boson, the \fieldZ~boson, with its couplings to the left- and right-handed
fermions dictated by the \SUewk~gauge mixing. The quantity $T_f^3$ is the fermion field's quantum
number associated with the third component of weak-isospin (\SUtwo).
The third term of Equation~\ref{eq:ewk_L_za} involves charged weak-interactions
involving the \fieldWpm$_{\mu}$ gauge bosons that transform the up- and down-type fields of
the left-handed \SUtwo~doublet fields into each other. 

The terms involving \fieldWpm$_{\mu}$ in Equation~\ref{eq:ewk_L_za} are
of the form $\bar{\nu}_L \gamma^{\mu} e_L$
which, using the chiral projection operators (Equation~\ref{eq:chiral_projection}), can be
written as follows,
\begin{align}
	\bar{\nu}_L \gamma{\mu} e_L = \frac{1}{2} \bar{\nu} \gamma^{\mu}(1-\gamma_5) e,
	\label{eq:v_minus_a}
\end{align}
showing that the charged weak interactions involving \fieldWpm$_{\mu}$ are the coherent
sum of vector ($\gamma^{\mu}$) and axial-vector ($\gamma^{\mu}\gamma_5$) bilinear covariants: this is the famous
\textit{V-A} charged-current interaction of Fermi's nuclear $\beta$-decay.
It is this \textit{V-A} form that results in the charged interactions of the weak force
not being invariant under chiral transformations ($f_R \leftrightarrow f_L$): they involve only
the left-chiral fermion fields. For this reason, \textit{parity}
is said to be maximally violated by the weak interaction.\footnote{A parity transformation
	refers to inverting a field's space coordinates as $\vec{x} \rightarrow -\vec{x}$.}
This result, as presented in the above, is due to our having injected
it into our assumption on the field content in the first place out of hindsight. There
is no first-principles reason why the weak interactions should be this way, however,
and historically it was arrived at empirically.

What we have principally shown in this section is that, in order for the \SUewk~content of
the SM Lagrangian to correspond to what is found experimentally, it is expected that
the gauge fields of the underlying symmetries mix. Specifically, the neutral \SUtwo~gauge field
(\fieldWzero$_{\mu}$) mixes with that of the \Uone~gauge symmetry (\fieldB$_{\mu}$) by an amount
dictated by the Weinberg angle, $\theta_W$, resulting in descriptions of interactions consistent
with the experimentally observed photon and Fermi decay, as well as a prediction of a neutral $Z$-boson. 
The fermion electric charge is seen to be dependent on this mixing
and can be related to the underlying \SUtwo~and \Uone~gauge symmetries by the
Gell-Mann-Nishijima relation,
\begin{align}
	Q_f = T_f^3 + \frac{1}{2}Y,
	\label{eq:gell_mann_nishijima}
\end{align}
This relation summarises well the result of the \SUewk~gauge mixing and allows one to
infer that the electromagnetism of common experience
is related to the weak interaction and is in fact just one aspect of a unified electroweak interaction.
%Later on, we will see that (gauge) unification such as this plays a large role in our
%current understanding of the universe.

We have thus shown that the SM predicts the existence of the familiar electromagnetic force
and potentially provides an additional mediator (the \fieldWpm$_{\mu}$) for the charged weak interaction that, prior to the
formulation of GWS, was lacking a consistent physical description. However, it is still
not evident how the SM can support the experimental fact that fermions have mass and that
the predicted mediator of the charged weak-nuclear force (the \fieldWpm$_{\pm}$) \textit{must}
be massive given the very short range of the interaction. In order for such mass terms to
be allowed in \SML, we need the Higgs mechanism.
