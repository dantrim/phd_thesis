\section{The Complete Standard Model, Successes and Shortcomings}
\label{sec:final_sm_description}

The physical field content of the SM, after EWSB, is detailed in Table~\ref{tab:sm_content_EWSB}
Also shown in Table~\ref{tab:sm_content_EWSB} are the electric charge associated with each
particle field, $Q$, the relevant couplings, and particle masses.


\begin{table}[!htb]
    \caption{
        The particle content of the SM after the process of
        electroweak symmetry breaking.
    }
    \begin{center}
        \begin{tabularx}{1\textwidth}{m{1em} c c c c }
        \toprule
        \hline
        & Physical Field & Q & Coupling & Mass [GeV] \\
        \hline
        \rotatebox{90}{\hspace{-0.1cm}\textbf{Quarks} } 
            & \makecell{ \quarkU, \quarkC, \quarkT \\ \quarkD, \quarkS, \quarkB} % FIELD
            & \makecell{ $2/3$ \\ $-1/3$ }% Q
            %& \makecell{ $\mathbf{3}$ \\ $\mathbf{3}$ } % SU(3)
            & \makecell{ ($y_i=$) $1\times10^{-5}$, $7\times10^{-3}$, $1$ \\ ($y_i=$) $3\times10^{-5}$, $5\times10^{-4}$, $0.02$ } % Coupling
            & \makecell{ $2\times10^{-3}$, $1.27$, $173$ \\ $4\times10^{-4}$, $0.10$, $4.18$ }\\% Mass
        \rotatebox{90}{\hspace{-0.1cm}\textbf{Leptons} } 
            & \makecell{ \leptonE, \leptonMu, \leptonTau \\ \neutrinoE, \neutrinoMu, \neutrinoTau } % FIELD
            & \makecell{ $-1$ \\ $0$ }% Q
            %& \makecell{ $\mathbf{1}$ \\ $\mathbf{1}$ } % SU(3)
            & \makecell{ ($y_i=$) $3\times10^{-7}$, $6\times10^{-4}$, $0.01$ \\ -- } % Coupling
            & \makecell{ $5\times10^{-4}$, $0.106$, $1.777$ \\ --}\\% Mass
        \midrule
        \rotatebox{90}{\textbf{Bosons} } 
            & \makecell{ \fieldPhoton \\ \fieldZ \\ (\fieldWp, \fieldWm) \\ \fieldG } % FIELD
            & \makecell{ $0$ \\ $0$ \\ $(+1,-1)$ \\ $0$ }% Q
            %& \makecell{ $\mathbf{1}$ \\ $\mathbf{1}$ \\ $\mathbf{1}$ \\ $\mathbf{8}$ } % SU(3)
            & \makecell{ $\alpha_{\text{EM}} \simeq 1/137$ \\ $\sin \theta_{W} \simeq 0.5$ \\ -- \\ $\alpha_s \simeq 0.1$ } % Coupling
            & \makecell{ $0$ \\ $91.2$ \\ $80.4$ \\  $0$}\\% Mass
        \midrule
        \rotatebox{90}{\textbf{Higgs} } 
            & \makecell{ \fieldH } % FIELD
            & \makecell{ $0$ }% Q
            %& \makecell{ $\mathbf{1}$ } % SU(3)
            & \makecell{ $\lambda$, $\mu$ } % Coupling
            & \makecell{ $125.09$ }\\% Mass
        \hline
        \bottomrule
        \end{tabularx}
    \end{center}
    \label{tab:sm_content}
\end{table}


%%%%%%%%%%%%%%%%%%%%%%%%%%%%%%%%%%%%%%%%%%%%%%%%%%%%%%%%%%%%%%%%%%%%%%%%%%%%%%%%%
%%%%%%%%%%%%%%%%%%%%%%%%%%%%%%%%%%%%%%%%%%%%%%%%%%%%%%%%%%%%%%%%%%%%%%%%%%%%%%%%%
%%%%%%%%%%%%%%%%%%%%%%%%%%%%%%%%%%%%%%%%%%%%%%%%%%%%%%%%%%%%%%%%%%%%%%%%%%%%%%%%%
%
% SUCCESSES
%
%%%%%%%%%%%%%%%%%%%%%%%%%%%%%%%%%%%%%%%%%%%%%%%%%%%%%%%%%%%%%%%%%%%%%%%%%%%%%%%%%
%%%%%%%%%%%%%%%%%%%%%%%%%%%%%%%%%%%%%%%%%%%%%%%%%%%%%%%%%%%%%%%%%%%%%%%%%%%%%%%%%
%%%%%%%%%%%%%%%%%%%%%%%%%%%%%%%%%%%%%%%%%%%%%%%%%%%%%%%%%%%%%%%%%%%%%%%%%%%%%%%%%

\subsection{Successes of the Standard Model}
\label{sec:sm_successes}

The particle content presented in Table~\ref{tab:sm_content_EWSB} represents the current
picture of the visible matter content of the universe.
It is a concise picture.
With this particle content in hand, and the description of their fundamental particle interactions (Equation~\ref{eq:sm_lagrangian}),
the predictive power of the SM is immense.

Figure~\ref{fig:sm_xsec_summary} shows a summary of LHC measurements of cross-sections
for various production processes at 7, 8, and 13\,TeV.
The agreement of these measurements with the theoretical predictions provided by the SM, spanning over 10 orders of magnitude,
is an incredible testament to the power of the SM and the toolkit provided by QFT.
The power of the SM, and its internal consistency, is illustrated in Figure~\ref{fig:mw_mt_scan},
which shows the results of indirect measurements of the $W$-boson and top-quark masses based
on fits to electroweak precision data.
When all measurements other than those on $m_h$ are included, the constraints of the SM
only allow for a small region of the $(m_{\text{top}}, M_W)$ parameter space.
Adding the Higgs measurements results~\cite{HMassATLAS,HMassCMS} only shrinks this allowed area.
The fact that these indirect measurements agree so well with the direct measurements paints a picture
of the SM being a fundamentally complete picture of the known physical phenomena.
If the SM were not accounting for particular types of interactions or known fields,
the level of agreement between the direct measurements and those obtained indirectly via
fits to electroweak precision data would not be to the level seen in Figure~\ref{fig:mw_mt_scan}.

\begin{figure}[!htb]
    \begin{center}
        \includegraphics[width=0.75\textwidth]{figures/chapter1/sm_final/sm_xsec_summary}
        \caption{
            Summary of several Standard Model total production cross section measurements,
            corrected for branching fractions, compared to the corresponding theoretical expectations. 
        }
        \label{fig:sm_xsec_summary}
    \end{center}
\end{figure}
\begin{figure}[!htb]
    \begin{center}
        \includegraphics[width=0.75\textwidth]{figures/chapter1/sm_final/mw_vs_mt_indirect}
        \caption{
            Contours at 68\% and 95\% CL obtained from scans of $M_W$ versus $m_{\text{top}}$,
            for the global electroweak fit including the Higgs boson mass ($m_h$) measurements~\cite{HMassATLAS,HMassCMS} (blue)
            and excluding the $m_h$ measurements (grey), as compared t the direct
            measurements of these quantities (green bands and ellipses).
            From Ref.~\cite{GFitter}.
        }
        \label{fig:mw_mt_scan}
    \end{center}
\end{figure}

With the discovery a Higgs boson like particle with a mass at 125\,GeV in 2012~\cite{HDiscoveryATLAS,HDiscoveryCMS},
the final piece of the SM described by Equation~\ref{eq:sm_lagrangian} is potentially found.
The 2012 Higgs discovery meant the start of a very long experimental program began, focused
on studying this new particle and confirming its role as being the fundamental scalar boson, $\phi$,
appearing in Equation~\ref{eq:sm_lagrangian}.

It should be stressed that the terms associated with the Higgs potential appearing in the SM
Lagrangian, given by Equation~\ref{eq:higgs_potential}, are by no means fundamental.
They did (and do) not have to appear in this way.
The terms appearing in Equation~\ref{eq:higgs_potential} take the form they do since they
could lead to masses for the fermions and gauge bosons.
There is no fundamental symmetry motivating their form or appearance.
The BEH mechanism is inspired by the formation of composite (i.e. not elementary) scalar particles
appearing in super-conductivity (Cooper pairs).
The fact that the same type of phase transition should describe the generation of masses for
the elementary particles of the SM, and that it should presuppose the existence of an \textit{elementary}
scalar boson, was simply left as one of the last open questions of the SM.
In a sense, the truth of the form underlying Equation~\ref{eq:higgs_potential} was not important to BEH.
The more important takeaway was that there \textit{could} be a mechanism by which the SM particles acquired
mass without disrupting the gauge structure that had already held up to experimental scrutiny.
%In a certain sense, this is reflected by the fact that BEH were awarded the Nobel prize only \textit{after} the discoveries of the ATLAS and CMS experiments
%in 2012, whereas GWS were awarded theirs several years \textit{before} the experimental verification of
%the existence of the $W$ and $Z$ bosons.

It is then up to the experiments to verify that the 125\,GeV scalar boson discovered in 2012
is responsible for the BEH mechanism as described in Section~\ref{sec:higgs_description}.
The form of the Higgs potential as defined in Equation~\ref{eq:higgs_potential} makes
very clear predictions on the form and strengths of the couplings to the known fundamental particles:
the fermions and gauge bosons, with couplings to the Higgs predicted to take the forms
of Equations~\ref{eq:higgs_fermion_coupling} and \ref{eq:higgs_gauge_couplings}, respectively.
It is then up to the ATLAS and CMS experiments to verify that the new particle couples
to these `old' particles just as predicted.
The coupling strengths also dictate the Higgs decay rates into specific SM particles, as indicated
in Figure~\ref{fig:higgs_br_sm}.
Any deviation in the measured values of the SM-predicted Higgs decay branching ratios or
in the predicted values of the fermion or gauge coupling strengths, and their dependence on the
particle masses, would indicate that the particle discovered in 2012 is not the Higgs boson as predicted
in the SM.

All of the measurements of the properties of the 125 GeV particle
made by the ATLAS and CMS experiments are so far in fairly good agreement with the SM prediction
of a $m_h = 125$ GeV Higgs boson~\cite{HProp0,HProp1,HProp2,HProp3,HProp4,HProp5,HProp6,HProp7,HProp8}.
The agreement with the SM prediction, over a wide variety of measurements, is illustrated in
Figure~\ref{fig:higgs_measurements} which shows the measurements of the fermion and gauge
couplings and of the cross-sections of the leading Higgs production mechanism and decay
branching ratios.
Within the precision of these measurements, the SM is fully supported by the experiments
and it appears as though the particle discovered in 2012 is in fact the particle as predicted
by BEH.


\begin{figure}[!htb]
    \begin{center}
        \includegraphics[width=0.6\textwidth]{figures/chapter1/sm_final/higgs_br_sm}
        \caption{
            Predicted branching ratios for an SM-like Higgs boson with $m_{h} = 125\,\GeV$.
        }
        \label{fig:higgs_br_sm}
    \end{center}
\end{figure}

\begin{figure}[!htb]
    \begin{center}
        \raisebox{1cm}{\includegraphics[width=0.49\textwidth]{figures/chapter1/sm_final/higgs_kappa_v_f}}
        \includegraphics[width=0.49\textwidth]{figures/chapter1/sm_final/higgs_kappa_vs_mass}
        \includegraphics[width=0.49\textwidth]{figures/chapter1/sm_final/higgs_prod_and_br}
        \caption{
            Higgs precision measurements of couplings to SM particles.
            Figures from Ref.~\cite{HiggsProps}.
            \textit{\textbf{Left}}: Combined measurement of fermion and gauge-boson Higgs coupling modifiers, $\kappa_f$
                and $\kappa_V$ (assumed to be universal across fermion and gauge-boson species in the result pictured).
                Values of $\kappa_f$ or $\kappa_V$ equal to 1 correspond to the SM prediction for the Higgs' couplings to
                these particles.
            \textit{\textbf{Right}}: Measured values of the Higgs fermion and gauge-boson coupling parameters
                as a function of the fermion and gauge-boson masses.
                The blue dashed line shows the SM prediction (Equations~\ref{eq:higgs_gauge_couplings} and \ref{eq:higgs_fermion_coupling}).
            \textit{\textbf{Bottom}}: Measurements of Higgs production cross sections and (relative) decay branching ratios.
        }
        \label{fig:higgs_measurements}
    \end{center}
\end{figure}

%%%%%%%%%%%%%%%%%%%%%%%%%%%%%%%%%%%%%%%%%%%%%%%%%%%%%%%%%%%%%%%%%%%%%%%%%%%%%%%%%
%%%%%%%%%%%%%%%%%%%%%%%%%%%%%%%%%%%%%%%%%%%%%%%%%%%%%%%%%%%%%%%%%%%%%%%%%%%%%%%%%
%%%%%%%%%%%%%%%%%%%%%%%%%%%%%%%%%%%%%%%%%%%%%%%%%%%%%%%%%%%%%%%%%%%%%%%%%%%%%%%%%
%
% SHORTCOMINGS
%
%%%%%%%%%%%%%%%%%%%%%%%%%%%%%%%%%%%%%%%%%%%%%%%%%%%%%%%%%%%%%%%%%%%%%%%%%%%%%%%%%
%%%%%%%%%%%%%%%%%%%%%%%%%%%%%%%%%%%%%%%%%%%%%%%%%%%%%%%%%%%%%%%%%%%%%%%%%%%%%%%%%
%%%%%%%%%%%%%%%%%%%%%%%%%%%%%%%%%%%%%%%%%%%%%%%%%%%%%%%%%%%%%%%%%%%%%%%%%%%%%%%%%

%eq:higgs_potential
