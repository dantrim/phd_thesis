\chapter{Physics Beyond The Standard Model}
\label{chap:bsm}

%\epigraph{\textit{All models are wrong, but some are useful.}}{--George Box}
\epigraph{\textit{The only consolation he drew from the present chaos was that his theory managed to explain it.}}{--Thomas Pynchon, \textit{V.}}

%As seen in the previous section, the SM is not an adequate theory to describe the entirety of the
%observable phenomena in the universe. However, it is still exceedingly useful in describing
%the... and predicting...
%
%There are a multitude of theories describing phsyics beyond the standard model, each attempting to
%explain the shortcomings of the SM either in part or in whole, though none have so far appeared to be
%useful in the sense of being able to extend our current understanding of the the universe by providing
%objectively falsifiable prediction.


%%%%%%%%%%%%%%%%%%%%%%%%%%%%%%%%%%%%%%%%%%%%%%%%%%%%%%%%%%%%%%%%%%%%%%%%%%%%%
%%%%%%%%%%%%%%%%%%%%%%%%%%%%%%%%%%%%%%%%%%%%%%%%%%%%%%%%%%%%%%%%%%%%%%%%%%%%%
%%%%%%%%%%%%%%%%%%%%%%%%%%%%%%%%%%%%%%%%%%%%%%%%%%%%%%%%%%%%%%%%%%%%%%%%%%%%%
%
% GENERAL
%
%%%%%%%%%%%%%%%%%%%%%%%%%%%%%%%%%%%%%%%%%%%%%%%%%%%%%%%%%%%%%%%%%%%%%%%%%%%%%
%%%%%%%%%%%%%%%%%%%%%%%%%%%%%%%%%%%%%%%%%%%%%%%%%%%%%%%%%%%%%%%%%%%%%%%%%%%%%
%%%%%%%%%%%%%%%%%%%%%%%%%%%%%%%%%%%%%%%%%%%%%%%%%%%%%%%%%%%%%%%%%%%%%%%%%%%%%
%\cite{2HDMPheno,EWSBMSSM,SUSYPrimer,LightStopsHiggs}
%\cite{EWSBProbeRho,PDGRef}
%\cite{ColemanMandula,WessZuminoModel,HaagSUSY}

As seen in Section~\ref{sec:final_sm_description}, the SM provides an incredible toolkit
from which the majority of the observed physical phenomena can be described.
As we saw, however, there are several open questions whose answers must be pursued
if we are to truly understand the underlying nature of the Universe.
Given the scope of these open questions, and the apparent inconsistencies exposed by the Hierarchy Problem,
their solutions are all but guaranteed to further revolutionize our view of the Universe.
As a result, there have been a multitude of proposals of new theories of particles and fields over the preceding decades that attempt
to solve some of these remaining mysteries.


With the discovery of the 125\,GeV Higgs boson, the physics community is seemingly
on the verge of gaining further insight into the nature of these problems.
Any new source of physics beyond the SM (BSM) will necessarily have much to say
about the electroweak and Higgs sectors of the SM,
especially in view of the Hierarchy Problem's hinting at the electroweak scale
being so near the portal to new physics.
These sectors tightly constrain the forms that BSM physics can take, as well, if they
are to remain consistent with the precise confirmation of the SM predictions
over the past 50 to 60 years.
One such constraint, a beautiful result of the electroweak theory and one that
is fundamental to the nature of EWSB, is that of the $\rho$ parameter of the SM,
defined in Equation~\ref{eq:rho_sm}.
The precise equality of Equation~\ref{eq:rho_sm} can be seen to hold by using the relationships
between the electroweak gauge mixings and couplings in Equations~\ref{eq:weinberg_angles} and \ref{eq:gauge_boson_masses}.
The $\rho = 1$ equality holds to very high precision even when including higher order
corrections to the gauge boson masses and interactions, a fact that his been verified
by precision measurements~\cite{EWSBProbeRho,PDGRef}.

\begin{align}
    \rho = \frac{
                M_W
            }
            {
                M_Z \cos \theta_W
            }
        \underset{\text{\tiny{SM}}}{=} 1
    \label{eq:rho_sm}
\end{align}
The requirement that $\rho = 1$ restricts the form any new BSM physics attempting to alter
the electroweak sector can take.
It is interesting to note that the last equality in Equation~\ref{eq:rho_sm} can hold even for Higgs sectors that are far more complex than
the minimal one predicted by the SM, and that are perhaps composed of \textit{multiple} Higgs doublets.
%In fact, the relation holds for Higgs sectors that are more complex than the minimal one
%predicted by the SM.
This can be seen by casting the $\rho$ parameter into its more general form, shown in Equation~\ref{eq:rho_general}:
\begin{align}
    \rho = \frac{
                \sum\limits_{i = 1}^n \, \left[ T_{3,i} \left( T_{3,i} + 1 \right) - \frac{1}{4} Y_i^2 \right ] v_i
            }
            {
                \sum\limits_{i=1}^n \, \frac{1}{2} Y_i^2 v_i
            },
    \label{eq:rho_general}
\end{align}
where $n$ runs over the number of scalar Higgs doublets in the theory, $T_{3}$ and $Y$ are the \SUtwo~and \Uone~gauge
quantum numbers, respectively, associated with each doublet, and the $v$ quantities are the vacuum expectation values of
each Higgs boson.
%Equation~\ref{eq:rho_general} accentuates the $\rho$ parameter's deep relation to the electroweak gauge structure of the theory
%and it can be seen 
%so long as their \SUtwo~and \Uone~gauge structure satisfies the fundamental relation in Equation~\ref{eq:rho_general},
In the SM, with only the single Higgs doublet having $Y = 1$ and $T_3 = 1/2$ (Table~\ref{tab:sm_content}),
the equality $\rho = 1$ clearly holds.
Any extended Higgs sectors whose multiple Higgs doublets have \SUtwo~and \Uone~gauge quantum numbers
satisfying $\rho = 1$ are in principle allowed within the constraints of current precision electroweak measurements.

The possibility that the Higgs sector may be more complex without disrupting strict relationships
predicted very cleary by the SM has led to the development of many fields of research.
One such potential for BSM physics is the class of theories that minimally extend the
Higgs sector by only including a single additional scalar double ($n = 2$ in Equation~\ref{eq:rho_general}).
Such theories are referred to as Two Higgs-Doublet Models (`2HDMs'), and their
phenomenology is rich~\cite{2HDMPheno}.
The two Higgs doublets are typically referred to as `$H_u$' and `$H_d$', where the subscripts
indicate to which component of the \SUtwo~fermions they couple and give mass, and are complex scalar
fields with eight degrees of freedom between the two of them.
In these models, the three degrees of freedom needed for the $W^{\pm}$ and $Z$ bosons
to acquire mass after EWSB leave open five additional degrees of freedom.
As a result, 2HDMs are characterized by an extended Higgs sector in which there are \textit{five}
physical massive Higgs bosons: two neutral components ($h^0$ and $H^0$), one neutral pseudoscalar particle ($A^0$),
and two that are electrically charged ($H^{\pm}$).
The neutral component $h^0$ is typicaly considered to be lightest Higgs boson and is identified as the
125\,GeV boson discovered in 2012.
Other than this assumption on $h^0$, the mass spectrum of these five Higgs bosons and their 
phenomenology is dependent on the specific instantiation of the 2HDM in question.
As it turns out, one of the most promising all-encompassing theories of new physics,
Supersymmetry (SUSY), is a 2HDM.
Section~\ref{sec:susy} provides a brief introduction to SUSY that will be relevant
as background to the analyses presented in Chapters~\ref{chap:search_stop} and \ref{chap:search_hh}.

%%%%%%%%%%%%%%%%%%%%%%%%%%%%%%%%%%%%%%%%%%%%%%%%%%%%%%%%%%%%%%%%%%%%%%%%%%%%%
%%%%%%%%%%%%%%%%%%%%%%%%%%%%%%%%%%%%%%%%%%%%%%%%%%%%%%%%%%%%%%%%%%%%%%%%%%%%%
%%%%%%%%%%%%%%%%%%%%%%%%%%%%%%%%%%%%%%%%%%%%%%%%%%%%%%%%%%%%%%%%%%%%%%%%%%%%%
%
% SUSY
%
%%%%%%%%%%%%%%%%%%%%%%%%%%%%%%%%%%%%%%%%%%%%%%%%%%%%%%%%%%%%%%%%%%%%%%%%%%%%%
%%%%%%%%%%%%%%%%%%%%%%%%%%%%%%%%%%%%%%%%%%%%%%%%%%%%%%%%%%%%%%%%%%%%%%%%%%%%%
%%%%%%%%%%%%%%%%%%%%%%%%%%%%%%%%%%%%%%%%%%%%%%%%%%%%%%%%%%%%%%%%%%%%%%%%%%%%%
\section{Supersymmetry}
\label{sec:susy}

\begin{align}
    Q \ket{\text{fermion}} = \ket{\text{boson}}, \hspace{1cm} Q \ket{\text{boson}} = \ket{\text{fermion}}
    \label{eq:susy_operator}
\end{align}

\begin{align}
    \left( \Delta m_h^2 \right)_{S} = \frac{\lambda_S}{16 \pi^2} \left[ 2 \Lambda^2 - \mathcal{O} \left( m_S^2 \ln \left( \frac{\Lambda}{m_S} \right) \right) \right]
    \label{eq:higgs_corr_susy}
\end{align}

\begin{align}
    \Delta m_h^2 \underset{\text{w/ SUSY}}{\propto} \left( m_f^2 - m_S^2 \right) \ln \left( \frac{\Lambda}{m_S} \right) + 3 m_f^2 \ln \left( \frac{m_S}{m_f} \right) + \mathcal{O} \left( \frac{1}{\Lambda^2} \right)
    \label{eq:higgs_corr_fixed}
\end{align}

\begin{table}[!htb]
    \begin{center}
        \caption{
            Particle content of the MSSM.
        }
        \label{tab:mssm_particles}
        \begin{tabular}{c | c | c | Sc | Sc}
        \hline
        \hline
            \textbf{Names} & \textbf{Spin} & \textbf{R-Parity} & \textbf{Gauge Eigenstate} & \textbf{Mass Eigenstate} \\
            \hline
            \multirow{3}{*}{Squarks} & \multirow{3}{*}{0} & \multirow{3}{*}{$-1$} & $\tilde{u}_L, \,\tilde{u}_R, \, \tilde{d}_L, \, \tilde{d}_R$ & same  \\
                                                & & & $\tilde{c}_L, \,\tilde{c}_R, \, \tilde{s}_L, \, \tilde{s}_R$ & same  \\
                                                & & & $\tilde{t}_L, \,\tilde{t}_R, \, \tilde{b}_L, \, \tilde{b}_R$ & $\tilde{t}_1,\,\tilde{t}_2,\,\tilde{b}_1,\,\tilde{b}_2$  \\
            \hline
            \multirow{3}{*}{Sleptons} & \multirow{3}{*}{0} & \multirow{3}{*}{$-1$} & $\tilde{e}_L,\,\tilde{e}_R,\,\tilde{\nu}_e$ & same \\
                & & & $\tilde{\mu}_L,\,\tilde{\mu}_R,\,\tilde{\nu}_{\mu}$ & same \\
                & & & $\tilde{\tau}_L,\,\tilde{\tau}_R,\,\tilde{\nu}_{\tau}$ & $\tilde{\tau}_1,\,\tilde{\tau}_2,\,\tilde{\nu}_{\tau}$ \\
            \hline
            Neutralinos & $1/2$ & $-1$ & $\tilde{B}^0,\,\tilde{W}^0,\,\tilde{H}^0_u,\,\tilde{H}^0_d$ & $\tilde{\chi}^0_1,\,\tilde{\chi}^0_2,\,\tilde{\chi}^0_3,\,\tilde{\chi}^0_4$ \\
            \hline
            Charginos & $1/2$ & $-1$ & $\tilde{W}^{\pm},\,\tilde{H}^+_u,\,\tilde{H}^-_d$ & $\tilde{\chi}^{\pm}_1,\,\tilde{\chi}^{\pm}_2$ \\
            \hline
            Gluino & $1/2$ & $-1$ & $\tilde{g}$ & same \\
            \hline
            \hline
            Higgs Bosons & 0 & $+1$ & $H_u^0,\,H_d^0,\,H_u^+,\,H_d^-$ & $h^0,\,H^0,\,A^0,\,H^{\pm}$  \\
        \hline
        \hline
        \end{tabular}
    \end{center}
\end{table}

\begin{align}
    \underbrace{\begin{pmatrix}
        f_{\small{1/2}} \\
        \tilde{\phi}_0
    \end{pmatrix}}_{\substack{\text{Chiral} \\ \text{Supermultiplet}}}
    \hspace{1cm}
    \underbrace{\begin{pmatrix}
        V_1 \\
        \tilde{V}_{1/2}
    \end{pmatrix}}_{\substack{\text{Gauge} \\ \text{Supermultiplet}}}
    \hspace{1cm}
    \begin{matrix}
        \begin{cases}
            \begin{tabular}{l l}
                $f_{1/2}$~: & \text{Spin-1/2 Weyl fermion} \\
                $\tilde{\phi}_0$~: & \text{Complex scalar} 
            \end{tabular}
        \end{cases} \\
        %\hspace{1.3cm}
        \begin{cases}
            \begin{tabular}{l l}
                $V_1$~: & \text{Spin-1 boson} \\
                $\tilde{V}_{1/2}$~: & \text{Spin-1/2 Weyl fermion}
            \end{tabular}
        \end{cases}
    \end{matrix}
    \label{eq:susy_multiplets}
\end{align}

%\begin{align}
%    \rho = \frac{
%                M_W
%            }
%            {
%                M_Z \cos \theta_W
%            }
%        \underset{\text{\tiny{SM}}}{=} 1
%    \label{eq:rho_sm}
%\end{align}
%
%\begin{align}
%    \rho = \frac{
%                \sum\limits_{i = 1}^n \, \left[ T_{3,i} \left( T_{3,i} + 1 \right) - \frac{1}{2} Y_i^2 \right ] v_i
%            }
%            {
%                \sum\limits_{i=1}^n \, Y_i^2 v_i
%            }
%    \label{eq:rho_general}
%\end{align}

\begin{figure}[!htb]
    \begin{center}
        \includegraphics[width=0.9\textwidth]{figures/higgs_corr/higgs_mass_corrections_stopPDF}
        \caption{
            Top and stop-quark loop contributions to the higher-order computation of the
            Higgs mass.
        }
        \label{fig:higgs_mass_correction_stop}
    \end{center}
\end{figure}

\begin{figure}[!htb]
    \begin{center}
        \includegraphics[width=0.95\textwidth]{figures/higgs_corr/hh_stopsPDF}
        \caption{
            Single-loop stop-quark diagrams in the MSSM that contribute to the non-resonant production
            of Higgs boson pair, in addition to those already predicted in the SM (Figure~\ref{fig:hh_feynman}). From Ref.~\cite{LightStopsHiggs}.
        }
        \label{fig:hh_stops}
    \end{center}
\end{figure}

\begin{figure}[!htb]
    \begin{center}
        \includegraphics[width=0.65\textwidth]{figures/higgs_corr/sigma_hh_stops}
        \caption{
            Higgs pair production cross-section normalized to the SM prediction as a function
            of the mass of the lighter stop quark, $\tilde{t}_1$.
            Details in Ref.~\cite{LightStopsHiggs}.
        }
        \label{fig:hh_sigma_stops}
    \end{center}
\end{figure}

