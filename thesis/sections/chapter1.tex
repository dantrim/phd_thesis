\chapter{The Standard Model of Particle Physics}

%\epigraph{\textit{So it goes...}}{---Kurt Vonnegut, \textit{Slaughterhouse
%		Five}}
	
\epigraph{\textit{Science is a miracle.}}{--Ron Swanson}


As it stands, what has become known as the `Standard Model (SM) of Particle Physics'
is nothing less than one of the greatest achievments of mankind, due to both
the magnitude by which it has changed our perception of the underlying
nature of the universe and to the clever methods and tinkerings by which this
nature was unveiled by many clever physicists whose history has become veritable lore.
In terms of imagination and insight, it is second only to the Special and General Theories of Relativity --
though the two fields are nevertheless intricately intertwined.
%{\color{red}{The latter, though, being put forth by essentially a single person and the latter by a great many...}}.

Not considering the scientific progress made in the $18^{th}$ and $19^{th}$ centuries, and
ignoring the ancient Greeks despite their fabled invention of atomic theory,
the physical insights and major work that led to the current picture of elementary particle
physics described by the SM began with the \textit{annus mirabilis} papers of Albert
Einstein in the year 1905~\cite{einsteinPEE,einsteinSpecial,einsteinEnergyMass}.
In these papers, Einstein was able to shed light on the quantization of electromagnetic
radiation (building off of the seminal work of Max Planck~\cite{planckBlackBody})
and introduce the Special Theory of Relativity.
These works laid the conceptual
and philosophical groundwork for the major breakthroughs in fundamental physics
of $20^{th}$ century physics: leading from the `old quantum theory' of Bohr and Sommerfeld
in the early 1900's to the equivalent wavefunction and matrix-mechanics formulations
of Schr{\"o}dinger and Heisenberg that
coalesced in the early-to-mid 1920's into the `modern quantum mechanics`.
This modern approach, non-relativistic at its heart, provided a sufficient mathematical
and interpretable framework in which to work and match prediction to observed phenomena, old
and new. It has for the most part remained unchanged and is the quantum mechanics that is taught to
students at both the undergraduate and graduate level to this very day.
It is the theory that has since revolutionised all aspects of the physical sciences and
technologies that dictate our everyday-lives.
In the mid-1920's, however, despite
large efforts put forth by the forbears of modern quantum mechanics, the quantum-mechanical
world had yet to be made consistent with Einstein's theory of relativity -- a requirement
that must be met for all consistent physical theories.
It was the insight of Paul Dirac who was finally able to successfully
marry the theory of the quantum with that of relativity when he introduced
his relativistic quantum-mechanical treatment of the electron described by what is now
simply referred to as the \textit{Dirac equation}~\cite{diracEquation,diracPrinciples}
\footnote{
A complete history of the people and ideas involved in the development of the modern
theory of Quantum Mechanics can be found in references ~\cite{boffiRiseOfQM,historyQM},
and the references therein.
}.
This work provided necessary insights for Dyson, Feynman, Schwinger, and Tomanaga
in their work on quantum electrodynamics (QED): the consistent quantum-mechanical and
relativistic treatment of electrodynamics. QED, with its predictions verified by observation to
absurd accuracies, was the first example of a \textit{quantum field theory} (QFT) and
led to the adoption of the framework of QFT as being the go-to choice for constructing accurate
models to describe nature.
\footnote{
	For a complete historical treatment of the development of QED...
}
The SM is an ultimate conclusion of these works: a consistent set of relativistic
quantum field theories using the language developed by Feynman, et. al.
that describes essentially all aspects of the known particles and forces that make up the 
observed universe.


%leading to Bohr early insights into the origins of the
%energy spectra and atomic structure of the hydrogen atom~\cite{bohrSpectral}, de Broglie's 
%notion of wave-particle duality~\cite{deBroglie}, and, eventually, the breakthroughs of
%Werner Heisenberg, Max Born, and Erwin Schr{\"o}dinger in the early 1920's that provided a sufficient
%mathematical and interpretable framework in which to construct the modern theory
%of non-relativistic quantum mechanics that has since revolutionised all aspects of the
%physical sciences and technologies that we encounter in our everyday lives.
%The treatment of non-relativistic quantum mechanics put forth in the early 1920's
%is, for the most part, what is still taught to physics students today and is still an active
%area of scientific and philosophical research. However, it was Paul Dirac who was finally able
%to succesfully marry the theory of the quantum with that of relativity in his
%relativistic quantum-mechanical treatment of the electron described by
%the \textit{Dirac equation}~\cite{diracEquation,diracPrinciples}. The work and 
%strange imagination of Dirac is finally what led to the work of Dyson, Feynman, Schwinger,
%and Tomonaga in the area of quantum electrodynamics (QED). The supreme accuracies with
%which QED describes reality, with its deceptively simple `Feynman diagrams', has since led
%to the framework of \textit{quantum field theories} (QFT) being the go-to choice for
%describing nature. The SM is no less than \textit{an} ultimate conclusion of these works: a consistent set
%of quantum field theories arrived at in the intervening decades since Feynman, et. al.,
%that describes esentially all aspects of the observed particles and forces that shape the known universe.

%See~\cite{boffiRiseOfQM}.
%See~\cite{historyQM}.


\section{Particles and Forces}

%Here we will describe briefly what is known as the `Standard Model of Particle
%Physics', which will be referred to as the `Standard Model` or simply the `SM'
%throughout
%this work. The physical insights and major work that lead to the current picture
%of elementary
%particle physics as described by the SM is largely described by the theories of
%Glashow,
%Weinberg, and Salam in what is now referred to as the GWS Theory. In the
%literature,
%the theory of GWS is actually what is referred to as the `Standard Model'
%despite
%it being an incomplete picture of elementary particle physics (it does not
%mention the theory
%of coloured objects, Quantum Chromodynamics, for instance) since it laid the
%basic
%fundamentals of building models of particles and their field representations
%under the
%framework of \textit{gauge theories}.


\subsection{Gauge Theories}

\subsubsection{The Electroweak Theory}


