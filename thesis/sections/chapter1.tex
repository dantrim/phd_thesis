\chapter{The Standard Model of Particle Physics}

%\epigraph{\textit{So it goes...}}{---Kurt Vonnegut, \textit{Slaughterhouse
%		Five}}
	
%\epigraph{\textit{Science is a miracle.}}{--Ron Swanson}

\epigraph{\textit{If you wish to make an apple pie from scratch, you must first invent the universe.}}{--Carl Sagan, \textit{Cosmos: A Personal Voyage}}


As it stands, what has become known as the `Standard Model (SM) of Particle Physics'
is nothing less than one of the greatest achievments of mankind, due to both
the magnitude by which it has changed our perception of the underlying
nature of the universe and to the clever methods and tinkerings by which this
nature was unveiled by many clever physicists whose history has become veritable lore.
In terms of imagination and insight, it is second only to the special and general theories of relativity --
though the fields are nevertheless intricately intertwined.
%{\color{red}{The latter, though, being put forth by essentially a single person and the latter by a great many...}}.

Not considering the scientific progress made in the $18^{th}$ and $19^{th}$ centuries, and
ignoring the ancient Greeks despite their fabled invention of atomic theory,
the physical insights and major work that led to the current picture of elementary particle
physics described by the SM began with the \textit{annus mirabilis} papers of Albert
Einstein in the year 1905~\cite{einsteinPEE,einsteinSpecial,einsteinEnergyMass}.
In these papers, Einstein was able to shed light on the quantization of electromagnetic
radiation (building off of the seminal work of Max Planck~\cite{planckBlackBody})
and introduce the special theory of relativity.
These works laid the conceptual
and philosophical groundwork for the major breakthroughs in fundamental physics
of $20^{th}$ century physics: from the `old quantum theory' of Bohr and Sommerfeld
in the early 1900's to the equivalent wavefunction and matrix-mechanics formulations
of Schr{\"o}dinger and Heisenberg that
coalesced into `modern' quantum mechanics in the mid-1920's.
The modern approach, non-relativistic at its heart, provided a sufficient mathematical
and interpretable framework in which to work and match predictions to observed phenomena, old
and new. It has for the most part remained unchanged and is the quantum mechanics that is taught to
students at both the undergraduate and graduate level to this very day.
It is the theory that has since revolutionised all aspects of the physical sciences and
technologies that dictate our everyday-lives.
In the mid-1920's, however, despite
large efforts put forth by the forbears of modern quantum mechanics, the quantum-mechanical
world had yet to be made consistent with Einstein's theory of relativity --- a requirement
that must be met for all consistent physical theories of nature.
It was the insight of Paul Dirac who was finally able to successfully
marry the theory of the quantum with that of relativity when he introduced
his relativistic quantum-mechanical treatment of the electron in 1927 and 1928~\cite{diracEquation,Dirac:1927dy}.\footnote{
A complete history of the people and ideas involved in the development of the modern
theory of Quantum Mechanics can be found in references ~\cite{boffiRiseOfQM,historyQM},
and the references therein.
}
This work provided the starting point for a decades-long search of a consistent quantum-mechanical
and relativistic treatment of electrodynamics, known as \textit{quantum electrodynamics} (QED).
The search for QED ended at the end of the 1940's with the groundbreaking work of Dyson, Feynman, Schwinger, and Tomanaga~\cite{qedTomonaga,qedFeynman0,qedFeynman1,qedFeynman2,qedSchwinger0,qedSchwinger1,qedDyson0,qedDyson1} that introduced the covariant and gauge invariant
formulation of QED --- the first such relativistic quantum field theory (QFT).
QED allowed the physcists to make predictions that agreed with observation at unprecedented levels
of accuracy and has since led to the adoption of its language and mathematical toolkit as the
foundational framework in which to construct models that accurately describe nature.\footnote{
	For a complete discussion of the developments leading up to QED, see the fabulous
	book by S. Schweber~\cite{Schweber:1994qa}.	
}
The SM is no less than an ultimate conclusion of these works: a consistent set of relativistic
quantum field theories, using the language developed by Feynman et al.,
that describes essentially all aspects of the known particles and forces that make up the 
observed universe.


\section{Particles and Forces}

Here we introduce the SM particle content and provide a description of the interactions that
link the particles together.


\begin{table}[!htb]
    \caption{
        The particle content of the SM and their transformation
        properties under the SM gauge groups, prior to electroweak symmetry breaking.
        The representations of each of the gauge groups are shown in the three-right
        columns. The \Uone symmetry of weak-hypercharge transformations is one-dimensional
        and the column gives the weak-hypercharge $\mathcal{Y}$ associated with each
        field. For \SUthree and \SUtwo, $\mathbf{1}$ refers to the field belonging to
        the associated singlet representation, $\mathbf{2}$ to the doublet representation,
        $\mathbf{3}$ to the triplet representation, and $\mathbf{8}$ to the octet representation.
    }
    \begin{center}
        \begin{tabularx}{0.96\textwidth}{m{1em} c c c c c c }
        \toprule
        \hline
        & Field Label & Content & Spin & \Uone~($\mathcal{=Y}$) & \SUtwo & \SUthree \\
        \hline
        \rotatebox{90}{\hspace{-0.1cm}\textbf{Quarks} } 
         &   \makecell{\fieldQi \\ \fieldUri \\ \fieldDri} % FIELD
         &   \makecell{ (\fieldUl, \fieldDl), (\fieldCl, \fieldSl), (\fieldTl, \fieldBl) \\ \fieldUr \\ \fieldDr}% CONTENT
         &   \makecell{ $1/2$ \\ $1/2$ \\ $1/2$} % SPIN
         &   \makecell{ $1/6$ \\ $2/3$ \\ $-1/3$}% U(1)
         &   \makecell{ $\mathbf{2}$ \\ $\mathbf{1}$ \\ $\mathbf{1}$}% SU(2)
         &   \makecell{ $\mathbf{3}$ \\ $\mathbf{3}$ \\ $\mathbf{3}$}\\ % SU(3)
        %\cdashline{1-7}
        \rotatebox{90}{\hspace{-0.1cm}\textbf{Leptons} }
         &   \makecell{\fieldLi \\ \fieldEri} % FIELD
         &   \makecell{ (\fieldEl, \fieldNuEl), (\fieldMul, \fieldNuMul), (\fieldTaul, \fieldNuTaul) \\ \fieldEr, \fieldMur, \fieldTaur}% CONTENT
         &   \makecell{ $1/2$ \\ $1/2$ }% SPIN
         &   \makecell{ $1/2$ \\ $-1$ }% U(1)
         &   \makecell{ $\mathbf{2}$ \\ $\mathbf{1}$ }% SU(2)
         &   \makecell{ $\mathbf{1}$ \\ $\mathbf{1}$ } \\ % SU(3)
        \midrule
        \rotatebox{90}{\textbf{\stackanchor{Gauge}{Fields}} }
         &   \makecell{\fieldB \\ \fieldW \\ \fieldG } % FIELD
         &   \makecell{ \fieldB \\ (\fieldWone, \fieldWtwo, \fieldWthree) \\ \fieldG$_a$, $a\in[1,..,8]$ }% CONTENT
         &   \makecell{ $1$ \\ $1$ \\ $1$} % SPIN
         &   \makecell{ $0$ \\ $0$ \\ $0$}% U(1)
         &   \makecell{ $\mathbf{1}$ \\ $\mathbf{3}$ \\ $\mathbf{1}$}% SU(2)
         &   \makecell{ $\mathbf{1}$ \\ $\mathbf{1}$ \\ $\mathbf{8}$}\\ % SU(3)
        \midrule
        \rotatebox{90}{\textbf{\stackanchor{Higgs}{Field}}} 
         &   \makecell{\fieldPhi } % FIELD
         &   \makecell{ (\fieldPhip, \fieldPhizero) }% CONTENT
         &   \makecell{ $0$  } % SPIN
         &   \makecell{ $1/2$  }% U(1)
         &   \makecell{ $\mathbf{2}$ }% SU(2)
         &   \makecell{ $\mathbf{1}$ }\\ % SU(3)
        \hline
        \bottomrule
        \end{tabularx}
    \end{center}
    \label{tab:sm_content}
\end{table}


\begin{table}[!htb]
    \caption{
        The particle content of the SM after the process of
        electroweak symmetry breaking.
    }
    \begin{center}
        \begin{tabularx}{1\textwidth}{m{1em} c c c c }
        \toprule
        \hline
        & Physical Field & Q & Coupling & Mass [GeV] \\
        \hline
        \rotatebox{90}{\hspace{-0.1cm}\textbf{Quarks} } 
            & \makecell{ \quarkU, \quarkC, \quarkT \\ \quarkD, \quarkS, \quarkB} % FIELD
            & \makecell{ $2/3$ \\ $-1/3$ }% Q
            %& \makecell{ $\mathbf{3}$ \\ $\mathbf{3}$ } % SU(3)
            & \makecell{ ($y_i=$) $1\times10^{-5}$, $7\times10^{-3}$, $1$ \\ ($y_i=$) $3\times10^{-5}$, $5\times10^{-4}$, $0.02$ } % Coupling
            & \makecell{ $2\times10^{-3}$, $1.27$, $173$ \\ $4\times10^{-4}$, $0.10$, $4.18$ }\\% Mass
        \rotatebox{90}{\hspace{-0.1cm}\textbf{Leptons} } 
            & \makecell{ \leptonE, \leptonMu, \leptonTau \\ \neutrinoE, \neutrinoMu, \neutrinoTau } % FIELD
            & \makecell{ $-1$ \\ $0$ }% Q
            %& \makecell{ $\mathbf{1}$ \\ $\mathbf{1}$ } % SU(3)
            & \makecell{ ($y_i=$) $3\times10^{-7}$, $6\times10^{-4}$, $0.01$ \\ -- } % Coupling
            & \makecell{ $5\times10^{-4}$, $0.106$, $1.777$ \\ --}\\% Mass
        \midrule
        \rotatebox{90}{\textbf{Bosons} } 
            & \makecell{ \fieldPhoton \\ \fieldZ \\ (\fieldWp, \fieldWm) \\ \fieldG } % FIELD
            & \makecell{ $0$ \\ $0$ \\ $(+1,-1)$ \\ $0$ }% Q
            %& \makecell{ $\mathbf{1}$ \\ $\mathbf{1}$ \\ $\mathbf{1}$ \\ $\mathbf{8}$ } % SU(3)
            & \makecell{ $\alpha_{\text{EM}} \simeq 1/137$ \\ $\sin \theta_{W} \simeq 0.5$ \\ -- \\ $\alpha_s \simeq 0.1$ } % Coupling
            & \makecell{ $0$ \\ $91.2$ \\ $80.4$ \\  $0$}\\% Mass
        \midrule
        \rotatebox{90}{\textbf{Higgs} } 
            & \makecell{ \fieldH } % FIELD
            & \makecell{ $0$ }% Q
            %& \makecell{ $\mathbf{1}$ } % SU(3)
            & \makecell{ $\lambda$, $\mu$ } % Coupling
            & \makecell{ $125.09$ }\\% Mass

         %&   \makecell{ (\quarkUl, \quarkDl), (\quarkCl, \quarkSl), (\quarkTl, \quarkBl) \\ \quarkUr \\ \quarkDr}% CONTENT
         %&   \makecell{ $1/2$ \\ $1/2$ \\ $1/2$} % SPIN
         %&   \makecell{ $\mathbf{2}$ \\ $\mathbf{1}$ \\ $\mathbf{1}$}% SU(2)
         %&   \makecell{ $\mathbf{3}$ \\ $\mathbf{3}$ \\ $\mathbf{3}$}\\ % SU(3)
        %%\cdashline{1-7}
        %rotatebox{90}{\hspace{-0.1cm}\textbf{Leptons} }
         %&   \makecell{\quarkLi \\ \quarkEri} % FIELD
         %&   \makecell{ (\quarkEl, \quarkNuEl), (\quarkMul, \quarkNuMul), (\quarkTaul, \quarkNuTaul) \\ \quarkEr, \quarkMur, \quarkTaur}% CONTENT
         %&   \makecell{ $1/2$ \\ $1/2$ }% SPIN
         %&   \makecell{ $\mathbf{2}$ \\ $\mathbf{1}$ }% SU(2)
         %&   \makecell{ $\mathbf{1}$ \\ $\mathbf{1}$ } \\ % SU(3)
        %midrule
        %rotatebox{90}{\textbf{\stackanchor{Gauge}{Fields}} }
         %&   \makecell{\quarkB \\ \quarkW \\ \quarkG } % FIELD
         %&   \makecell{ \quarkB \\ (\quarkWone, \quarkWtwo, \quarkWthree) \\ \quarkG }% CONTENT
         %&   \makecell{ $1$ \\ $1$ \\ $1$} % SPIN
         %&   \makecell{ $\mathbf{1}$ \\ $\mathbf{3}$ \\ $\mathbf{1}$}% SU(2)
         %&   \makecell{ $\mathbf{1}$ \\ $\mathbf{1}$ \\ $\mathbf{8}$}\\ % SU(3)
        %midrule
        %rotatebox{90}{\textbf{\stackanchor{Higgs}{Field}}} 
         %&   \makecell{\quarkPhi } % FIELD
         %&   \makecell{ (\quarkPhip, \quarkPhizero) }% CONTENT
         %&   \makecell{ $0$  } % SPIN
         %&   \makecell{ $\mathbf{2}$ }% SU(2)
         %&   \makecell{ $\mathbf{1}$ }\\ % SU(3)
        \hline
        \bottomrule
        \end{tabularx}
    \end{center}
    \label{tab:sm_content}
\end{table}



\subsection{Gauge Theories}

\subsubsection{The Electroweak Theory}


