\section{General Analysis Strategy}
\label{sec:gen_strategy}

In this section the analysis strategy used in the searches for new physics
to be presented in Chapters~\ref{chap:search_stop2l} and \ref{chap:search_bbww}
will be given.
The general analysis workflow for designing an analysis is outlined in the following
sub-sections.

%%%%%%%%%%%%%%%%%%%%%%%%%%%%%%%%%%%%%%%%%%%%%%%%%%%%%%%%%%%%%%%%%%%%%%%%%%%%%
% SIGNAL PHENO
%%%%%%%%%%%%%%%%%%%%%%%%%%%%%%%%%%%%%%%%%%%%%%%%%%%%%%%%%%%%%%%%%%%%%%%%%%%%%
\subsection{Target the Signal}
\label{sec:sig_pheno}

The search for a particular source of new physics, such as a particular model of SUSY (Chapter~\ref{chap:bsm}),
begins first with the thorough understanding of the signatures that the new physics model
will leave in the ATLAS detector.
This generally requires a strict definition of the \textit{final state} of the
new physics model that one wishes to look for; for example, deciding to search for
evidence of SUSY via the production of the SUSY partners to the SM top-quark
in final states having exactly two leptons (electrons or muons) instead of
exactly zero or exactly one lepton, as in Chapter~\ref{chap:search_stop2l}.\footnote{Performing
searches for new physics by the partitioning of specific new physics models
by their resulting final states allows for separate, independent dedicated analyses to be carried out
for each possible final state with the idea that each one will be more sensitive
to the presence of the new physics in their respective final state than would be
a single analysis attempting to target all possible final states of the new physics production.
The results of the independent analyses' searches can be statistically combined once they are finished,
leading to enhanced sensitivities to the new physics scenario in question that is more or less independent
of the final state.}
Once a new physics model has been chosen, along with its final state, there is a well-defined
\textit{signal} to be looked for in the data recorded by the ATLAS detector.
The production and decay of the sought-for signal is then simulated via MC methods in the exact
same manner as for the SM processes, as described in Chapter~\ref{chap:simulation}.
In physics analyses, the physics processes not inclusive of the sought-for signal processes
are referred to as the \textit{background} processes.

The simulation of the signal process allows one to study the kinematics of the signal in detail, in order
to get an overall feel for what phase space the signal inhabits.
Knowledge of both the signal final state and its kinematics therein informs the analyst
about the specific SM background processes that are likely to be relevant to the analysis.
For example, if the sought-for signal decays to two leptons with opposite electric charge
that are of the same flavor (both leptons are electrons or both are muons, for example)
it is very likely that the SM processes inclusive of $Z$-boson production will be relevant,
since this is one of the main $Z$-boson decay final states, as opposed to the production of a single $W$-boson
whose decay does not lead directly to final states with two leptons.
Knowledge of the dominant SM background processes, then, allows
one to determine how the phenomenology and kinematics of the signal differ
with respect to those of the relevant backgrounds by comparing the simulated events
of each.
The aim of this is to be able to define a basis of kinematic observables that allows
for the discrimination between the signal and background.
From such a basis of observables, one can define regions of phase space in which
the signal-to-background ratio is large, such that the likelihood of observing
the presence of the signal is (ideally) maximal.
Such regions of increased signal purity\footnote{The `purity' of a process is defined
as the fraction of a given process in a region of phase space, relative to the sum
of all processes (inclusive of the process in question).} are referred
to as \textit{signal regions} (SR).
As an example, take the case where there is a single discriminating variable in our
basis of useful kinematic observables.
One would apply a selection on this observable in such a way that $pp$ collision events
satisfying this selection are likely to be enhanced in signal events.
This is one-diminsional SR case is illustrated in Figure~\ref{fig:sr_search_v}.

\begin{figure}[!htb]
    \begin{center}
        \includegraphics[width=0.65\textwidth]{figures/common_ana/sr_search_vPDF}
        \caption{
            Signal region concept illustrated in the case of a one-dimensional selection
            made on a discriminating kinematic observable.
            The dominant SM background (red) is characterised by typically small values
            of the discriminating variable whereas the signal (blue) has values that extend
            beyond that of the background.
            The signal region in this case is defined by requiring $pp$ collision events
            to have values of the discriminating variable that are larger than
            the value indicated by the dashed vertical line, where the signal purity is
            enhanced.
        }
        \label{fig:sr_search_v}
    \end{center}
\end{figure}



%%%%%%%%%%%%%%%%%%%%%%%%%%%%%%%%%%%%%%%%%%%%%%%%%%%%%%%%%%%%%%%%%%%%%%%%%%%%%
% GATHER THE DATA
%%%%%%%%%%%%%%%%%%%%%%%%%%%%%%%%%%%%%%%%%%%%%%%%%%%%%%%%%%%%%%%%%%%%%%%%%%%%%
\FloatBarrier

%%%%%%%%%%%%%%%%%%%%%%%%%%%%%%%%%%%%%%%%%%%%%%%%%%%%%%%%%%%%%%%%%%%%%%%%%%%%%
% THE CONTROL REGION METHOD
%%%%%%%%%%%%%%%%%%%%%%%%%%%%%%%%%%%%%%%%%%%%%%%%%%%%%%%%%%%%%%%%%%%%%%%%%%%%%
\subsection{Background Estimation and the Control Region Method}
\label{sec:control_region_method}

The general principle behind searches for new physics is to define a SR, or a set of SRs,
and then make predictions about how the signal and background behave therein.
Such predictions can then be compared to the data actually recorded by the ATLAS detector
and the statistical procedures described in Section~\ref{sec:stat_hypo} can be used to
make statements about whether or not --- or to what degree --- the data is likely to contain the specified signal.
The emphasis, then, in physics analyses is on the understanding and precise estimation of the backgrounds.
Without being able to properly estimate the contribution of the background processes to the
events in the SRs well-defined predictions cannot therein be made, resulting in ineffective
analyses.

The process of estimating the backgrounds in an analysis' SRs is aptly referred to as
\textit{background estimation}.
There are many background estimation methods that are used.
There exist general background estimation techniques, applicable to a wide range of SM processes,
as well as more dedicated estimation techniques that are specific to a smaller subset of
SM processes.
Most rely on the MC simulation of the SM processes, either as the primary source of providing
the prediction of a given SM process in an analysis' SR(s) or secondarily, as a means of providing a
cross-check on a prediction obtained using data.
The high levels of accuracy imposed upon the ATLAS MC simulation infrastructure is derived
from the large and dominant role that the MC simulation plays in the background estimation
procedures in almost all analyses performed by the ATLAS experiment.

\begin{figure}[!htb]
    \begin{center}
        \includegraphics[width=0.65\textwidth]{figures/common_ana/sr_search_v_CRPDF}
        \caption{
        }
        \label{fig:sr_search_v_CR}
    \end{center}
\end{figure}

%%%%%%%%%%%%%%%%%%%%%%%%%%%%%%%%%%%%%%%%%%%%%%%%%%%%%%%%%%%%%%%%%%%%%%%%%%%%%
% SYSTEMATIC UNCERTAINTIES
%%%%%%%%%%%%%%%%%%%%%%%%%%%%%%%%%%%%%%%%%%%%%%%%%%%%%%%%%%%%%%%%%%%%%%%%%%%%%
