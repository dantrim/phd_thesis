\section{General Analysis Strategy}
\label{sec:gen_strategy}

In this section the analysis strategy used in the searches for new physics
to be presented in Chapters~\ref{chap:search_stop2l} and \ref{chap:search_bbww}
will be given.
The general analysis workflow for designing an analysis is outlined in the following
sub-sections.

%%%%%%%%%%%%%%%%%%%%%%%%%%%%%%%%%%%%%%%%%%%%%%%%%%%%%%%%%%%%%%%%%%%%%%%%%%%%%
% SIGNAL PHENO
%%%%%%%%%%%%%%%%%%%%%%%%%%%%%%%%%%%%%%%%%%%%%%%%%%%%%%%%%%%%%%%%%%%%%%%%%%%%%
\subsection{Target Phenomenology}
\label{sec:sig_pheno}

The search for a particular source of new physics, such as a particular model of SUSY (Chapter~\ref{chap:bsm}),
begins first with the thorough understanding of the signatures that the new physics model
will leave in the ATLAS detector.
This generally requires a strict definition of the \textit{final state} of the
new physics model that one wishes to look for; for example, deciding to search for
evidence of SUSY via the production of the SUSY partners to the SM top-quark
in final states having exactly two leptons (electrons or muons) instead of
exactly zero or exactly one lepton, as in Chapter~\ref{chap:search_stop2l}.
Performing searches for new physics by the partitioning of specific new physics models
by their resulting final states allows for separate, independent analyses to be carried out
for each possible final state with the idea that each one will be more sensitive
to the presence of the new physics in their respective final state than would be
a single analysis attempting to target all possible final states of the new physics production.
The results of the independent analyses' searches can be combined once they are finished,
leading to enhanced sensitivities to the new physics scenario in question that is independent
of the final state.
Once a new physics model has been chosen, along with its final state, there is a well-defined
\textit{signal} to be looked for in the data recorded by the ATLAS detector.
The production and decay of the sought-for signal is then simulated via MC methods in the exact
same manner as for the SM processes, as described in Chapter~\ref{chap:simulation}.
In physics analyses, the physics processes not inclusive of the sought-for signal processes
are referred to as the \textit{background} processes.

The simulation of the signal process allows one to study the kinematics of the signal in detail, in order
to get an overall feel for what phase space the signal inhabits.
Knowledge of both the signal final state and its kinematics therein informs the analyst
about the specific SM background processes that are likely to be relevant to the analysis.
For example, if the sought-for signal decays to two leptons with opposite electric charge
that are of the same flavor (both leptons are electrons or both are muons, for example)
it is very likely that the SM processes inclusive of $Z$-boson production will be relevant,
since this is one of the main $Z$-boson decay final states, as opposed to the production of a single $W$-boson
whose decay does not lead to final states with two leptons.
Knowledge of the `dominant' SM background processes, then, allow
one to determine how the phenomenology and kinematics of the signal differ
to those of the relevant backgrounds.
The aim of this is to define regions of phase space in which the signal-to-background ratio
is (ideally) maximal, such that the likelihood of observing the presence of the signal process
is large.
These regions of increased signal purity are referred to as \textit{signal regions},
and are typically defined by choosing a basis of kinematic observables
that allow for discrimination between the signal and background processes.


%%%%%%%%%%%%%%%%%%%%%%%%%%%%%%%%%%%%%%%%%%%%%%%%%%%%%%%%%%%%%%%%%%%%%%%%%%%%%
% GATHER THE DATA
%%%%%%%%%%%%%%%%%%%%%%%%%%%%%%%%%%%%%%%%%%%%%%%%%%%%%%%%%%%%%%%%%%%%%%%%%%%%%


%%%%%%%%%%%%%%%%%%%%%%%%%%%%%%%%%%%%%%%%%%%%%%%%%%%%%%%%%%%%%%%%%%%%%%%%%%%%%
% BACKGROUND ESTIMATION
%%%%%%%%%%%%%%%%%%%%%%%%%%%%%%%%%%%%%%%%%%%%%%%%%%%%%%%%%%%%%%%%%%%%%%%%%%%%%

%%%%%%%%%%%%%%%%%%%%%%%%%%%%%%%%%%%%%%%%%%%%%%%%%%%%%%%%%%%%%%%%%%%%%%%%%%%%%
% THE CONTROL REGION METHOD
%%%%%%%%%%%%%%%%%%%%%%%%%%%%%%%%%%%%%%%%%%%%%%%%%%%%%%%%%%%%%%%%%%%%%%%%%%%%%

%%%%%%%%%%%%%%%%%%%%%%%%%%%%%%%%%%%%%%%%%%%%%%%%%%%%%%%%%%%%%%%%%%%%%%%%%%%%%
% SYSTEMATIC UNCERTAINTIES
%%%%%%%%%%%%%%%%%%%%%%%%%%%%%%%%%%%%%%%%%%%%%%%%%%%%%%%%%%%%%%%%%%%%%%%%%%%%%
