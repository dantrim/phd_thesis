\section{Hypothesis Testing and Statistics}
\label{sec:stat_hypo}

This section describes the statistical procedures used in the analyses to be
presented in Chapters~\ref{chap:seach_stop} and \ref{chap:search_hh} that allow
for conclusions to be drawn about the compatibility of the observed data with theories of
BSM physics.
The statistical inference tools described are inherently Frequentist and
are, for the most part, the \textit{de facto} standard for physics analyses searching for evidence of BSM physics
at the large experiments at the LHC.
Their widespread adoption by the experiments at the LHC does not indicate the
philosophical merit of Frequentist inference methodology, but rather highlights the technically simpler implementation
of Frequentist hypothesis testing that allows for physics analyses to not get
bogged down in some of the details associated with Bayesian analyses, computational
or otherwise.
Indeed, most people by default think and interact with the world around them
in a Bayesian manner.
Taking the path of least resistance, physicists have tended to opt for the simpler
implementation of reporting their results, which, at the end of the day,
tend to not lose out much in terms of the picture of the objective truth that they draw~\cite{CousinsBayes}.
Section~\ref{sec:hypo_test} will describe, in somewhat general terms, what
hypotheses tests are and the way in which they are performed in ATLAS.
Section~\ref{sec:likelihood} describes the details by which the measurements and systematic
uncertainties of
an analysis are transcribed into the language of the hypothesis test described
in Section~\ref{sec:hypotest} using a likelihood-based test statistic.


%%%%%%%%%%%%%%%%%%%%%%%%%%%%%%%%%%%%%%%%%%%%%%%%%%%%%%%%%%%%%%%%%%%
%%%%%%%%%%%%%%%%%%%%%%%%%%%%%%%%%%%%%%%%%%%%%%%%%%%%%%%%%%%%%%%%%%%
%
% HYPOTHESIS TESTING
%
%%%%%%%%%%%%%%%%%%%%%%%%%%%%%%%%%%%%%%%%%%%%%%%%%%%%%%%%%%%%%%%%%%%
%%%%%%%%%%%%%%%%%%%%%%%%%%%%%%%%%%%%%%%%%%%%%%%%%%%%%%%%%%%%%%%%%%%

\subsection{Hypothesis Testing and the \cls Construction}
\label{sec:hypo_test}

Hypothesis testing starts with the unambiguous formulation of the hypothesis being
tested.
In the search for evidence of BSM physics, there are two hypothesis pitted
against one another.
The first is the \textit{null hypothesis}, denoted $H_0$, which is the hypothesis
subject to the test and corresponds to the SM hypothesis. The null hypothesis is commonly
referred to simply as the background-only (B) hypothesis.
The second hypothesis is the \textit{alternative hypothesis}, denoted $H_1$, and corresponds
to the SM with the addition of the BSM physics process being sought out.
The hypothesis $H_1$ is commonly referred to as the signal-plus-background (S+B) hypothesis.
In both searches presented in Chapters~\ref{chap:search_stop} and \ref{chap:search_hh},
$H_0$ is taken to be the SM.
In the search presented in Chapter~\ref{chap:search_stop}, $H_1$ is taken to be a specific
instantiation of the MSSM ({\color{red}{Section XXX}}), with specific masses of the
stop quark and LSP.
In the search presented in Chapter~\ref{chap:search_hh}, $H_1$ is taken to be the
non-resonant production of Higgs boson pairs.
In this latter case, the $H_1$ hypothesis is indeed a process predicted by the SM ({\color{red}{SECTION XXX about HH pheno and EWSB}}) but
it is one that is not included in the $H_0$ hypothesis.

%%%%%%%%%%%%%%%%%%%%%%%%%%%%%%%%%%%%%%%%%%%%%%%%%%%%%%%%%%%%%%%%%%%
%%%%%%%%%%%%%%%%%%%%%%%%%%%%%%%%%%%%%%%%%%%%%%%%%%%%%%%%%%%%%%%%%%%
%
% TEST STATISTICS
%
%%%%%%%%%%%%%%%%%%%%%%%%%%%%%%%%%%%%%%%%%%%%%%%%%%%%%%%%%%%%%%%%%%%
%%%%%%%%%%%%%%%%%%%%%%%%%%%%%%%%%%%%%%%%%%%%%%%%%%%%%%%%%%%%%%%%%%%
\subsubsection{The Test Statistic and $p$-Values}

In order to perform a hypothesis test in the Frequentist arena, a \textit{test statistic}, $q(x)$,
is defined.
A test statistic is defined using the analysis' measurements alone and is
used in order to define metrics by which the observed data is said to agree with
one of the two hypothesis, either $H_0$ or $H_1$.
In Section~\ref{sec:likelihood}, the exact form of the likelihood used in modern LHC experiments,
and that used in the analyses discussed in Chapters~\ref{chap:search_stop} and \ref{chap:search_hh},
will be introduced.
Here we will discuss general features of Frequentist test statistics and introduce
some of the language that will be used later on when discussing the results of the
analyses.

The conclusions eventually drawn about a given hypothesis are baesd on the observed value
of $q(x)$ and where this value lies in relation to the pre-defined \textit{critical region}.
The critical region is defined by a cut value, $q_c$, on the distribution of $q(x)$ under a specified hypothesis.
In the one-sided tests to be considered in the present thesis, $H_1$
will tend to have larger values of $q(x)$ as compared to the $H_0$.
The critical region defines two important parameters associated with the hypothesis test.
The first is the quantity $\alpha$, which is referred to as the \textit{significance level},
and is defined as follows,
\begin{align}
    \int\limits_{q_c}^{+\infty} \, f(q | H_0) \, \mathrm{d}q = \alpha,
    \label{eq:sig_level}
\end{align}
where $f(q|H_0)$ is the probability distribution for the test statistic under
the background-only hypothesis.
The quantity $\alpha$ reports the probability for the background-only hypothesis (the SM) to be rejected when it is actually
true. This is commonly referred to as the Type I error rate.
The second quantity is $\beta$ and is defined as,
\begin{align}
    \int\limits_{-\infty}^{q_c} \, f(q|H_1) \, \mathrm{d}q = \beta,
    \label{eq:power_level}
\end{align}
where $f(q|H_1)$ is the probability distribution for the test statistic under
the signal-plus-background hypothesis.
The quantity $\beta$ gives the probability to reject the signal-plus-background hypothesis
when it is actually true. This is commonly referred to as the Type II error rate.
The quantity $(1-\beta)$ is referred to as the \textit{power of the test}.
The better a given physics analysis is at being able to discriminate between the signal
and background, i.e. to have clear separation between the $H_0$ and $H_1$ hypotheses,
the smaller (larger) is $\beta$ (the power of the test).

For simplicity, the two hypotheses $H_0$ and $H_1$ can be generalised by introducing a so-called
`signal strength' parameter, $\mu$, which acts as a multiplicative factor on the signal cross-section
appearing in $H_1$.
The hypothesis $H_0$, then, corresponds to the case $\mu = 0$ and that of $H_1$ corresponds to
$\mu = 1$.
With this general notation, then, the test statistic under either hypothesis is labelled as $q_{\mu}$.

Once a test statistic is specified, and its expected distribution under a given hypothesis is obtained,
$p$-values can be defined in order to compute the probability that the observed data originates from the
considered hypothesis (value of $\mu$).
They are computed as follows,
\begin{align}
    p_{\mu} = \int\limits_{q_{\mu, \text{obs}}}^{+\infty} \, f(q_{\mu} | \mu) \, \mathrm{d}q_{\mu},
    \label{eq:test_stat_pvalue}
\end{align}
where $q_{\mu, \text{obs}}$ is the observed value of the test statistic in data and $f(q_{\mu} | \mu)$ is the probability
density function of $q_{\mu}$ assuming hypothesis $\mu$.
A particular case of Equation~\ref{eq:test_stat_pvalue} is that of $p_0$, which quantifies the agreement of the data with the background-only
hypothesis ($\mu = 0$).
The $p_{\mu}$-value associated with a given hypothesis ($\mu$-value) is typically converted into the equivalent corresponding Gaussian significance, $Z$, defined
as the number of standard deviations that correspond to an upper-tail probability of $p_{\mu}$.
This is illustrated in Figure~\ref{fig:pval_sig}.

As the value of $p_{\mu}$ gets smaller, the confidence that the assumed hypothesis (value of $\mu$) is true
decreases.
At a certain point, it becomes acceptable to say that the assumed hypothesis is incompatible with
reality and the hypothesis described by the particular value of $\mu$ is said to be \textit{excluded}.
In the particle physics community, the conventional threshold to take for the value of $p_{\mu}$
at which point a hypothesis is said to be excluded is $p_{\mu} = 0.05$, corresponding to $Z=1.64$ as
illustrated in Figure~\ref{fig:pval_sig}.
This choice of the $p_{\mu}$-value at which point exclusion is said to occur defines
the critical region, described above, of the test.
The value of $0.05$ corresponds to the significance level of the test (c.f. Equation~\ref{eq:sig_level}), and is referred
to a hypothesis test being performed at the $95\%$ confidence level (CL) (i.e. CL $\equiv (1-\alpha)$)

In order to claim that new physics has been seen, the null hypothesis ($\mu = 0$) must be rejected.
The thresholds at which new physics can be said to have been observed and discovered are
much more stringent than that used for the exclusion of a specified hypothesis.
Incompatibilities with the null-hypothesis at the level of $p_0 = 1.3 \times 10^{-3}$ and
$p_0 = 2.9\times 10^{-7}$ are required in order to state that observation and discovery, respectively,
of new phenomena has occurred.
These thresholds, illustrated in Figure~\ref{fig:pval_sig}, for claiming observation and discovery are the fabled `$3\sigma$' and `$5\sigma$'
$p_0$-value criterion adopted by the particle physics community.

\begin{figure}[!htb]
    \begin{center}
        \includegraphics[width=0.48\textwidth]{figures/common_ana/stat_hypo/pval_sig_lin}
        \includegraphics[width=0.48\textwidth]{figures/common_ana/stat_hypo/pval_sig_log}
        \caption{
            Gaussian tail significance levels corresponding to $Z = 1.64\sigma$, $3\sigma$, and $5\sigma$
            significances, corresponding to $p_0$-values of $0.05$, $1.3 \times 10^{-3}$, and $2.9\times 10^{-7}$, respectively.
            \textbf{\textit{Left}}: Linear $y$-scale. \textbf{\textit{Right}}: Logarithmic $y$-scale.
        }
        \label{fig:pval_sig}
    \end{center}
\end{figure}

%The process of performing a hypothesis test, then, is specified by the following procedure:
%\begin{enumerate}
%    \item Unambiguously define the background-only (the SM prediction) and signal-plus-background (the prediction of the SM with BSM physics added)
%            hypotheses, $H_0$ and $H_1$.
%    \item Define an appropriate test statistic, $t(x)$ (see Section~\ref{sec:likelihood}).
%    \item Construct the distribution of the test statistic under the background-only hypothesis, $g(t|H_0)$.
%    \item Define the desired significance level of the hypothesis, $\alpha$
%    \item Obtain the value of $t(x)$ as observed in data
%    \item If the observed value of $t(x)$ is within the critical region defined by $\alpha$ as in Equation~\ref{eq:sig_level} ($t_{\text{obs}} > t_c$), reject $H_0$.
%            Otherwise, $H_0$ cannot be rejected.
%\end{enumerate}


%%%%%%%%%%%%%%%%%%%%%%%%%%%%%%%%%%%%%%%%%%%%%%%%%%%%%%%%%%%%%%%%%%%
%%%%%%%%%%%%%%%%%%%%%%%%%%%%%%%%%%%%%%%%%%%%%%%%%%%%%%%%%%%%%%%%%%%
%
% THE CLS METHOD
%
%%%%%%%%%%%%%%%%%%%%%%%%%%%%%%%%%%%%%%%%%%%%%%%%%%%%%%%%%%%%%%%%%%%
%%%%%%%%%%%%%%%%%%%%%%%%%%%%%%%%%%%%%%%%%%%%%%%%%%%%%%%%%%%%%%%%%%%

\subsubsection{The \cls Construction}
\label{sec:cls_method}

In searches for new physics, the statement that a given signal hypothesis has been excluded
is an important one.
Once made by the LHC experiments, the specific signal model is essentially considered
no longer important to be searched for.
Therefore, the metrics by which the experiments make claims of exclusion have surrounding them
a wide-ranging literature discussing the merits and drawbacks of the many such metrics
that have been proposed over the years.
The bare $p_{\mu}$-value, for example, extracted from the observed data is subject to statistical fluctuations
and it can lead to unphysical exclusions when a downward fluctuation in the observed
number of events occurs.
This would lead to a premature exclusion of perhaps a broad region of new physics
that would perhaps no longer be looked into by future analyses or experiments.

The standard metric used by the LHC experiments today is known as `\cls'~\cite{CLSReadI,CLSReadII},
and is constructed in such a way as to minimize the likelihood of excluding signal
hypotheses that a search is not a-priori sensitive to.
The \cls metric is given by,
\begin{align}
    \text{CL}_s = \frac{p_{\mu}}{1-p_0},
    \label{eq:cls_def}
\end{align}
where the quantities $p_{\mu}$ and $p_0$ quantify the compatibilities between the data and the signal-plus-background
and background-only hypotheses, respectively.
Downward fluctuations in data, as those described above, will lead to larger values of $p_0$; thus
leading to larger values of \cls that avoid premature exclusion.

At the LHC, the \cls metric is used primarily for performing hypothesis tests aimed at claiming exclusion.
The standard null-hypothesis $p_0$-value is still used for claiming observation and discovery, as described above.
A given signal hypothesis with $\mu = 1$ is considered excluded when $\cls \le 0.05$.
Note that this prescription for exclusion, $\cls \le \alpha$, is generally a stronger requirement than the
standard prescription, $p_{\mu} \le \alpha$.
The \cls metric is also used to compute \textit{upper limits}.
An upper limit on a given signal hypothesis specified by $\mu$
is the largest value of $\mu$ satisfying $\cls \ge 0.05$.
The interpretation being that this corresponds to the largest possible signal cross-section
that is unable to be excluded and therefore smaller values of $\mu$, corresponding
to smaller signal cross-sections, are still consistent with the observed data and cannot therefore
be excluded.
The process of scanning $\mu$ hypotheses and computing the \cls in order to find
an upper limit on $\mu$ is illustrated in Figure~\ref{fig:upper_limit_scan_cartoon}.
%An upper limit on a given signal hypothesis specified by $\mu$ is
%the value of $\mu$ at which $\cls = 0.05$.
%An illustration of how an upper limit is obtained is provided by Figure~\ref{fig:upper_limit_scan_cartoon}.
%%the largest value of $\mu$ describing the signal process that satisfies $\cls \le 0.05$.

\begin{figure}[!htb]
    \begin{center}
        \includegraphics[width=0.5\textwidth]{figures/common_ana/stat_hypo/upper_limit_scan_examplePDF}
        \caption{
            An upper limit scan on the signal strength parameter $\mu$ associated with a signal hypothesis.
            The \cls, given by Equation~\ref{eq:cls_def}, is recomputed for a range of $\mu$ values
            describing a given signal hypothesis.
            This is shown by the blue line.
            The $\mu$ value at which the \cls curve crosses the line $\cls = 0.05$, $\mu^{\text{UL}}$, is the
            upper limit on $\mu$ for the signal hypothesis.
            Values of $\mu$ smaller than $\mu^{\text{UL}}$ remain compatible with the observed data,
            while those values greater than $\mu^{\text{UL}}$ are excluded at $95\%$ CL.
        }
        \label{fig:upper_limit_scan_cartoon}
    \end{center}
\end{figure}

%%%%%%%%%%%%%%%%%%%%%%%%%%%%%%%%%%%%%%%%%%%%%%%%%%%%%%%%%%%%%%%%%%%
%%%%%%%%%%%%%%%%%%%%%%%%%%%%%%%%%%%%%%%%%%%%%%%%%%%%%%%%%%%%%%%%%%%
%
% PROFILE LIKELIHOOD
%
%%%%%%%%%%%%%%%%%%%%%%%%%%%%%%%%%%%%%%%%%%%%%%%%%%%%%%%%%%%%%%%%%%%
%%%%%%%%%%%%%%%%%%%%%%%%%%%%%%%%%%%%%%%%%%%%%%%%%%%%%%%%%%%%%%%%%%%

\subsection{Likelihood Analysis}
\label{sec:likelihood}

