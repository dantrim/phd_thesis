\section{Hypothesis Testing and Statistics}
\label{sec:stat_hypo}

This section describes the statistical procedures used in the analyses to be
presented in Chapters~\ref{chap:seach_stop} and \ref{chap:search_hh} that allow
for conclusions to be drawn about the compatibility of the observed data with theories of
BSM physics.
The statistical inference tools described are inherently Frequentist and
are, for the most part, the \textit{de facto} standard for physics analyses searching for evidence of BSM physics
at the large experiments at the LHC.
Their widespread adoption by the experiments at the LHC does not indicate the
philosophical merit of Frequentist inference methodology, but rather highlights the technically simpler implementation
of Frequentist hypothesis testing that allows for physics analyses to not get
bogged down in some of the details associated with Bayesian analyses, computational
or otherwise.
Indeed, most people by default think and interact with the world around them
in a Bayesian manner.
Taking the path of least resistance, physicists have tended to opt for the simpler
implementation of reporting their results, which, at the end of the day,
tend to not lose out much in terms of the picture of the objective truth that they draw~\cite{CousinsBayes}.
Section~\ref{sec:hypo_test} will describe, in somewhat general terms, what
hypotheses tests are and the way in which they are performed in ATLAS.
Section~\ref{sec:likelihood} describes the details by which the measurements and systematic
uncertainties of
an analysis are transcribed into the language of the hypothesis test described
in Section~\ref{sec:hypotest} using a likelihood-based test statistic.


%%%%%%%%%%%%%%%%%%%%%%%%%%%%%%%%%%%%%%%%%%%%%%%%%%%%%%%%%%%%%%%%%%%
%%%%%%%%%%%%%%%%%%%%%%%%%%%%%%%%%%%%%%%%%%%%%%%%%%%%%%%%%%%%%%%%%%%
%
% HYPOTHESIS TESTING
%
%%%%%%%%%%%%%%%%%%%%%%%%%%%%%%%%%%%%%%%%%%%%%%%%%%%%%%%%%%%%%%%%%%%
%%%%%%%%%%%%%%%%%%%%%%%%%%%%%%%%%%%%%%%%%%%%%%%%%%%%%%%%%%%%%%%%%%%

\subsection{Hypothesis Testing and the \cls Construction}
\label{sec:hypo_test}

Hypothesis testing starts with the unambiguous formulation of the hypothesis being
tested.
In the search for evidence of BSM physics, there are two hypothesis pitted
against one another.
The first is the \textit{null hypothesis}, denoted $H_0$, which is the hypothesis
subject to the test and corresponds to the SM hypothesis. The null hypothesis is commonly
referred to simply as the background-only (B) hypothesis.
The second hypothesis is the \textit{alternative hypothesis}, denoted $H_1$, and corresponds
to the SM with the addition of the BSM physics process being sought out.
The hypothesis $H_1$ is commonly referred to as the signal-plus-background (S+B) hypothesis.
In both searches presented in Chapters~\ref{chap:search_stop} and \ref{chap:search_hh},
$H_0$ is taken to be the SM.
In the search presented in Chapter~\ref{chap:search_stop}, $H_1$ is taken to be a specific
instantiation of the MSSM ({\color{red}{Section XXX}}), with specific masses of the
stop quark and LSP.
In the search presented in Chapter~\ref{chap:search_hh}, $H_1$ is taken to be the
non-resonant production of Higgs boson pairs.
In this latter case, the $H_1$ hypothesis is indeed a process predicted by the SM ({\color{red}{SECTION XXX about HH pheno and EWSB}}) but
it is one that is not included in the $H_0$ hypothesis.

%%%%%%%%%%%%%%%%%%%%%%%%%%%%%%%%%%%%%%%%%%%%%%%%%%%%%%%%%%%%%%%%%%%
%%%%%%%%%%%%%%%%%%%%%%%%%%%%%%%%%%%%%%%%%%%%%%%%%%%%%%%%%%%%%%%%%%%
%
% TEST STATISTICS
%
%%%%%%%%%%%%%%%%%%%%%%%%%%%%%%%%%%%%%%%%%%%%%%%%%%%%%%%%%%%%%%%%%%%
%%%%%%%%%%%%%%%%%%%%%%%%%%%%%%%%%%%%%%%%%%%%%%%%%%%%%%%%%%%%%%%%%%%
\subsubsection{The Test Statistic and $p$-Values}

In order to perform a hypothesis test in the Frequentist arena, a \textit{test statistic}, $t(x)$,
is defined.
A test statistic is defined using the analysis' measurements alone and is
used in order to define metrics by which the observed data is said to agree with
one of the two hypothesis, $H_0$ or $H_1$.
In Section~\ref{sec:likelihood}, the exact form of the likelihood used in modern LHC experiments,
and that used in the analyses discussed in Chapters~\ref{chap:search_stop} and \ref{chap:search_hh},
will be introduced.
Here we will discuss general features of Frequentist test statistics and introduce
some of the language that will be used later on when discussing the results of the
analyses.

The conclusions eventually drawn about a given hypothesis are baesd on the observed value
of $t(x)$ and where this value lies in relation to the pre-defined \textit{critical region}.
The critical region is defined by a cut value, $t_c$, on the distribution of $t(x)$ under a specified hypothesis.
In the one-sided tests to be considered in the present thesis, $H_1$
will tend to have larger values of $t(x)$ as compared to the $H_0$.
The critical region defines two important parameters associated with the hypothesis test.
The first is the quantity $\alpha$, which is referred to as the \textit{significance level},
and is defined as follows,
\begin{align}
    \int\limits_{t_c}^{+\infty} \, g(t | H_0) \, \mathrm{d}t = \alpha,
    \label{eq:sig_level}
\end{align}
where $g(t|H_0)$ is the probability distribution for the test statistic under
the background-only hypothesis.
The quantity $\alpha$ reports the probability for the background-only hypothesis (the SM) to be rejected when it is actually
true. This is commonly referred to as the Type I error rate.
The second quantity is $\beta$ and is defined as,
\begin{align}
    \int\limits_{-\infty}^{t_c} \, g(t|H_1) \, \mathrm{d}t = \beta,
    \label{eq:power_level}
\end{align}
where $g(t|H_1)$ is the probability distribution for the test statistic under
the signal-plus-background hypothesis.
The quantity $\beta$ gives the probability to reject the signal-plus-background hypothesis
when it is actually true. This is commonly referred to as the Type II error rate.
The quantity $(1-\beta)$ is referred to as the \textit{power of the test}.
The better a given physics analysis is at being able to discriminate between the signal
and background, i.e. to have clear separation between the $H_0$ and $H_1$ hypotheses,
the smaller (larger) is $\beta$ (the power of the test).

For simplicity, the two hypotheses $H_0$ and $H_1$ can be generalised by introducing a so-called
`signal strength' parameter, $\mu$, which acts as a multiplicative factor on the signal cross-section
appearing in $H_1$.
The hypothesis $H_0$, then, corresponds to the case $\mu = 0$ and that of $H_1$ corresponds to
$\mu = 1$.

%The process of performing a hypothesis test, then, is specified by the following procedure:
%\begin{enumerate}
%    \item Unambiguously define the background-only (the SM prediction) and signal-plus-background (the prediction of the SM with BSM physics added)
%            hypotheses, $H_0$ and $H_1$.
%    \item Define an appropriate test statistic, $t(x)$ (see Section~\ref{sec:likelihood}).
%    \item Construct the distribution of the test statistic under the background-only hypothesis, $g(t|H_0)$.
%    \item Define the desired significance level of the hypothesis, $\alpha$
%    \item Obtain the value of $t(x)$ as observed in data
%    \item If the observed value of $t(x)$ is within the critical region defined by $\alpha$ as in Equation~\ref{eq:sig_level} ($t_{\text{obs}} > t_c$), reject $H_0$.
%            Otherwise, $H_0$ cannot be rejected.
%\end{enumerate}


%%%%%%%%%%%%%%%%%%%%%%%%%%%%%%%%%%%%%%%%%%%%%%%%%%%%%%%%%%%%%%%%%%%
%%%%%%%%%%%%%%%%%%%%%%%%%%%%%%%%%%%%%%%%%%%%%%%%%%%%%%%%%%%%%%%%%%%
%
% THE CLS METHOD
%
%%%%%%%%%%%%%%%%%%%%%%%%%%%%%%%%%%%%%%%%%%%%%%%%%%%%%%%%%%%%%%%%%%%
%%%%%%%%%%%%%%%%%%%%%%%%%%%%%%%%%%%%%%%%%%%%%%%%%%%%%%%%%%%%%%%%%%%

\subsubsection{The \cls Construction}
\label{sec:cls_method}

%%%%%%%%%%%%%%%%%%%%%%%%%%%%%%%%%%%%%%%%%%%%%%%%%%%%%%%%%%%%%%%%%%%
%%%%%%%%%%%%%%%%%%%%%%%%%%%%%%%%%%%%%%%%%%%%%%%%%%%%%%%%%%%%%%%%%%%
%
% PROFILE LIKELIHOOD
%
%%%%%%%%%%%%%%%%%%%%%%%%%%%%%%%%%%%%%%%%%%%%%%%%%%%%%%%%%%%%%%%%%%%
%%%%%%%%%%%%%%%%%%%%%%%%%%%%%%%%%%%%%%%%%%%%%%%%%%%%%%%%%%%%%%%%%%%

\subsection{Likelihood Analysis}
\label{sec:likelihood}

