\section{Hypothesis Testing and Statistics}
\label{sec:stat_hypo}

This section describes the statistical procedures used in the analyses to be
presented in Chapters~\ref{chap:seach_stop} and \ref{chap:search_hh} that allow
for conclusions to be drawn about the compatibility of the observed data with theories of
BSM physics.
The statistical inference tools described are inherently Frequentist and
are, for the most part, the \textit{de facto} standard for physics analyses searching for evidence of BSM physics
at the large experiments at the LHC.
Their widespread adoption by the experiments at the LHC does not indicate the
philosophical merit of Frequentist inference methodology, but rather highlights the technically simpler implementation
of Frequentist hypothesis testing that allows for physics analyses to not get
bogged down in some of the details associated with Bayesian analyses, computational
or otherwise.
Indeed, most people by default think and interact with the world around them
in a Bayesian manner.
Taking the path of least resistance, physicists have tended to opt for the simpler
implementation of reporting their results, which, at the end of the day,
tend to not lose out much in terms of the picture of the objective truth that they draw~\cite{CousinsBayes}.
Section~\ref{sec:hypo_test} will describe, in somewhat general terms, what
hypotheses tests are and the way in which they are performed in ATLAS.
Section~\ref{sec:likelihood} describes the details by which the measurements and systematic
uncertainties of
an analysis are transcribed into the language of the hypothesis test described
in Section~\ref{sec:hypotest} using a likelihood-based test statistic.


%%%%%%%%%%%%%%%%%%%%%%%%%%%%%%%%%%%%%%%%%%%%%%%%%%%%%%%%%%%%%%%%%%%
%%%%%%%%%%%%%%%%%%%%%%%%%%%%%%%%%%%%%%%%%%%%%%%%%%%%%%%%%%%%%%%%%%%
%
% HYPOTHESIS TESTING
%
%%%%%%%%%%%%%%%%%%%%%%%%%%%%%%%%%%%%%%%%%%%%%%%%%%%%%%%%%%%%%%%%%%%
%%%%%%%%%%%%%%%%%%%%%%%%%%%%%%%%%%%%%%%%%%%%%%%%%%%%%%%%%%%%%%%%%%%

\subsection{Hypothesis Testing and the \cls Construction}
\label{sec:hypo_test}

Hypothesis testing starts with the unambiguous formulation of the hypothesis being
tested.
In the search for evidence of BSM physics, there are two hypothesis pitted
against one another.
The first is the \textit{null hypothesis}, denoted $H_0$, which is the hypothesis
subject to the test and corresponds to the SM hypothesis. The null hypothesis is commonly
referred to simply as the background-only (B) hypothesis.
The second hypothesis is the \textit{alternative hypothesis}, denoted $H_1$, and corresponds
to the SM with the addition of the BSM physics process being sought out.
The hypothesis $H_1$ is commonly referred to as the signal-plus-background (S+B) hypothesis.
In both searches presented in Chapters~\ref{chap:search_stop} and \ref{chap:search_hh},
$H_0$ is taken to be the SM.
In the search presented in Chapter~\ref{chap:search_stop}, $H_1$ is taken to be a specific
instantiation of the MSSM ({\color{red}{Section XXX}}), with specific masses of the
stop quark and LSP.
In the search presented in Chapter~\ref{chap:search_hh}, $H_1$ is taken to be the
non-resonant production of Higgs boson pairs.
In this latter case, the $H_1$ hypothesis is indeed a process predicted by the SM ({\color{red}{SECTION XXX about HH pheno and EWSB}}) but
it is one that is not included in the $H_0$ hypothesis.

%%%%%%%%%%%%%%%%%%%%%%%%%%%%%%%%%%%%%%%%%%%%%%%%%%%%%%%%%%%%%%%%%%%
%%%%%%%%%%%%%%%%%%%%%%%%%%%%%%%%%%%%%%%%%%%%%%%%%%%%%%%%%%%%%%%%%%%
%
% TEST STATISTICS
%
%%%%%%%%%%%%%%%%%%%%%%%%%%%%%%%%%%%%%%%%%%%%%%%%%%%%%%%%%%%%%%%%%%%
%%%%%%%%%%%%%%%%%%%%%%%%%%%%%%%%%%%%%%%%%%%%%%%%%%%%%%%%%%%%%%%%%%%
\subsubsection{The Test Statistic and $p$-Values}

In order to perform a hypothesis test in the Frequentist arena, a \textit{test statistic}, $t(x)$,
is defined.
A test statistic is defined using the analysis' measurements alone and is
used in order to define metrics by which the observed data is said to agree with
one of the two hypothesis, $H_0$ or $H_1$.


%%%%%%%%%%%%%%%%%%%%%%%%%%%%%%%%%%%%%%%%%%%%%%%%%%%%%%%%%%%%%%%%%%%
%%%%%%%%%%%%%%%%%%%%%%%%%%%%%%%%%%%%%%%%%%%%%%%%%%%%%%%%%%%%%%%%%%%
%
% THE CLS METHOD
%
%%%%%%%%%%%%%%%%%%%%%%%%%%%%%%%%%%%%%%%%%%%%%%%%%%%%%%%%%%%%%%%%%%%
%%%%%%%%%%%%%%%%%%%%%%%%%%%%%%%%%%%%%%%%%%%%%%%%%%%%%%%%%%%%%%%%%%%

\subsubsection{The \cls Construction}
\label{sec:cls_method}

%%%%%%%%%%%%%%%%%%%%%%%%%%%%%%%%%%%%%%%%%%%%%%%%%%%%%%%%%%%%%%%%%%%
%%%%%%%%%%%%%%%%%%%%%%%%%%%%%%%%%%%%%%%%%%%%%%%%%%%%%%%%%%%%%%%%%%%
%
% PROFILE LIKELIHOOD
%
%%%%%%%%%%%%%%%%%%%%%%%%%%%%%%%%%%%%%%%%%%%%%%%%%%%%%%%%%%%%%%%%%%%
%%%%%%%%%%%%%%%%%%%%%%%%%%%%%%%%%%%%%%%%%%%%%%%%%%%%%%%%%%%%%%%%%%%

\subsection{Likelihood Analysis}
\label{sec:likelihood}

