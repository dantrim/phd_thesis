\section{Systematic Uncertainties}
\label{sec:common_systematics}

There are many sources of systematic uncertainty that affect the results of the analyses
to be presented in Chapters~\ref{chap:search_stop} and \ref{chap:search_hh}.
The considered sources of systematic uncertainty are listed in Table~\ref{tab:syst_summary}.
There are uncertainties related to the overall event (e.g. the uncertainty
on the luminosity measurement), on the reconstruction and identification of
the physics objects described in Section~\ref{chap:objects}, and in the MC
simulation of both the SM background and signal processes.
Brief descriptions of the sources of systematic uncertainties (`systematics')
appearing in Table~\ref{tab:syst_summary} are given in Sections~\ref{sec:syst_experimental}-\ref{sec:syst_sig_modelling}.

\begin{table}[!htb]
    \caption{
        Summary of the sources of systematic uncertainties affecting the measurements
        in the analyses discussed in Chapters~\ref{chap:search_stop} and \ref{chap:search_hh}.
        Sources of uncertainty in {\color{red}{red}} ({\color{blue}{blue}}) pertain only to the
        search presented in Chapter~\ref{chap:search_stop} (\ref{chap:search_hh}).
        Those in black are considered in both analyses.
        For the uncertainties related to the SM background modelling, it is indicated
        whether or not they are computed using the Transfer Factor Method (Section~\ref{sec:transfer_factor}).
    }
    \label{tab:syst_summary}
    \begin{footnotesize}
    \begin{center}
        %\begin{tabularx}{\textwidth}{@{\extracolsep{\fill}}c c c c}
        %\begin{tabularx}{\textwidth}{c c c c}
        \begin{tabular}{c c c c}
        \toprule
        \hline
        \multicolumn{4}{c}{\textbf{Event-level}} \\
        \hline
        \multicolumn{4}{c}{Luminosity Measurement} \\
        \multicolumn{4}{c}{Pile-up Modelling} \\
%        \multicolumn{3}{c}{Luminosity} & \multicolumn{2}{c}{ Pile-up }  \\
        \hline
        \midrule
        \multicolumn{4}{c}{\textbf{Object Reconstruction}} \\ \hline
        \hspace{-2cm} \underline{\textbf{Jets}} &     \hspace{-1.8cm} \underline{\textbf{Flavor Tagging}} & \hspace{0.5cm}\underline{\textbf{Leptons}} & \hspace{0.3cm} \underline{\textbf{\met}} \\
        \hspace{-2cm} Jet Energy Scale (JES) &        \hspace{-1.8cm} $b$-tag Eff. & \hspace{0.5cm}Reconstruction Eff. & \hspace{0.3cm} Soft-term Resolution \\
        \hspace{-2cm} Jet Energy Resolution (JER) &   \hspace{-1.8cm} Mis-tag Eff. & \hspace{0.5cm}Identification Eff. & \hspace{0.3cm}  Soft-term Scale \\
        \hspace{-2cm} Pile-up Suppression (JVT) &     \hspace{-1.8cm}   & \hspace{0.5cm}Isolation Eff. &\hspace{0.3cm}  \\
        \hspace{-2cm} &     \hspace{-1.8cm}   & \hspace{0.5cm}Trigger Eff. &\hspace{0.3cm}  \\
        \midrule
        \midrule
        \multicolumn{4}{c}{\textbf{Background Modelling}} \\
        \hline
        \multicolumn{1}{l}{\textbf{Souce of Uncertainty}} & \multicolumn{2}{c}{\textbf{Affected Background Processes}} & \textbf{Transfer Factor Approach?} \\
        \hline
        \multicolumn{1}{l}{Hard-Scatter Generation}  & \multicolumn{2}{c}{\ttbar, \wt, \color{blue}{\zhf}} & Yes \\
        \multicolumn{1}{l}{Fragmentation} & \multicolumn{2}{c}{\ttbar, \wt} & Yes \\
        \multicolumn{1}{l}{Additional Radiation (ISR and FSR variation)} & \multicolumn{2}{c}{\ttbar, \wt} & Yes \\
        \multicolumn{1}{l}{PDF Choice and Uncertainty} & \multicolumn{2}{c}{\ttbar, \wt, \color{blue}{\zhf}, \color{red}{\vv}} & Yes \\
        \multicolumn{1}{l}{Scale ($\mu_R$, $\mu_F$) Variations} & \multicolumn{2}{c}{\ttbar, \wt, \color{red}{\vv}, \color{blue}{\zhf}} & Yes \\
        \multicolumn{1}{l}{Cross-section Uncertainty} & \multicolumn{2}{c}{\color{blue}{\ttbar}, \color{blue}{\wt}} & No \\
        \multicolumn{1}{l}{\ttbar~Interference Uncertainty} & \multicolumn{2}{c}{\wt} & Yes \\
        \multicolumn{1}{l}{Prompt Subtraction \& Fake Composition} & \multicolumn{2}{c}{Fake Lepton Background} & No \\ 
        \multicolumn{1}{l}{SS-OS Extrapolation} & \multicolumn{2}{c}{\color{blue}{Fake Lepton Background}} & No \\ 
        \midrule
        \midrule
        \multicolumn{4}{c}{\textbf{Signal Modelling}} \\
        \hline
        \multicolumn{4}{c}{{\color{blue}{Fragmentation}}} \\
        \multicolumn{4}{c}{Scale ($\mu_R$, $\mu_F$) Variations} \\
        \multicolumn{4}{c}{PDF Choice and Uncertainty} \\
        \multicolumn{4}{c}{Cross-section Uncertainty} \\
        
%        %\multicolumn{1}{l|}{ }  & \underline{\ttbar} & \underline{$Wt$} & \underline{Diboson} & \underline{$Z$+jets} & \underline{Fake Lepton Estimate} \\
%        \multicolumn{1}{l|}{Uncertainty Source }  & \ttbar & $Wt$ & Diboson & $Z$+jets & Fake Lepton \\
%        \hline
%        \multicolumn{1}{l|}{Hard-Scatter Generation}  & \checkmark & \checkmark & & & n/a\\
%        \multicolumn{1}{l|}{Fragmentation}  & \checkmark & \checkmark  & & & \\
%        \multicolumn{1}{l|}{Additional Radiation} & \checkmark & \checkmark & & & \\
%        \multicolumn{1}{l|}{PDF} & \checkmark & \checkmark &  & \checkmark & \\
%        \multicolumn{1}{l|}{Scale ($\mu_R$, $\mu_F$) Variations} & \checkmark & \checkmark & & & \\
%        \multicolumn{1}{l|}{Cross-section} & \checkmark$_{HH}$& & & & \\
%        HS + Matching & HS + Matching & & &  \\
%        Frag. + Had. & Frag. + Had. & & &  \\
        \hline
        \bottomrule
        \end{tabular}
    \end{center}
    \end{footnotesize}
\end{table}


%%%%%%%%%%%%%%%%%%%%%%%%%%%%%%%%%%%%%%%%%%%%%%%%%%%%%%%%%%%%%%%%%%%
%%%%%%%%%%%%%%%%%%%%%%%%%%%%%%%%%%%%%%%%%%%%%%%%%%%%%%%%%%%%%%%%%%%
%
% EXPERIMENTAL UNCERTAINTIES
%
%%%%%%%%%%%%%%%%%%%%%%%%%%%%%%%%%%%%%%%%%%%%%%%%%%%%%%%%%%%%%%%%%%%
%%%%%%%%%%%%%%%%%%%%%%%%%%%%%%%%%%%%%%%%%%%%%%%%%%%%%%%%%%%%%%%%%%%
\subsection{Experimental Uncertainties}
\label{sec:syst_experimental}

\subsubsection{Event-wide Uncertainties}
Event-wide (process-independent) uncertainties affecting the overall normalisation of the processes
relate to both the luminosity and pileup measurements.
The uncertainty on the integrated luminosity used to normalise all MC simulated
processes is derived following the methodology described in Ref.~\cite{LumiUncert}.
For the 2015+2016 (full Run 2) dataset, relevant to the analysis described in Chapter~\ref{chap:search_stop} (\ref{chap:search_hh}),
this uncertainty was found to be 2.1\% (1.7\%).
The uncertainty on the luminosity measurement does not affect processes whose
SR normalisation is constrained by data in dedicated CRs.

An uncertainty is considered on re-weighting the pileup distributions in the
MC simulation.
The re-weighting is applied in order to correct for the differences in the actual pile-up
distributions observed in data and those assumed at the time of producing the MC simulation (Section~\ref{sec:pileup_sim}).

\subsubsection{Jet Uncertainties}
The systematic uncertainties on reconstructed jet objects are related to the jet energy
resolution (JER), jet energy scale (JES), and jet vertex tagger (JVT) {\color{red}{don't forget to describe JVT}}.
There are many sources of uncertainties related to the JES and JER, each related to a specific
part of the JES and JER calibration measurements, as described in Section~\ref{sec:jet_calib}.
They arise from the techniques and corrections derived in MC, including statistical, detector,
modelling effects, jet flavor compositions, pileup corrections, and $\eta$-dependence effects.
The effects of the JES and JER uncertainties are among the more dominant sources of uncertainty
on the final analysis results in both analyses to be presented.
Given the complexity of the JES and JER calibrations, there are nearly 100 components associated with their uncertainties
that must be incorporated into an analysis' measurement uncertainty.

\subsubsection{Flavor Tagging Uncertainties}
There are uncertainties in the jet flavor tagging efficiencies, as well as in
the measured mis-tagging efficiencies associated with $c$- and light jets.
They are a mixture of statistical, experimental, and modelling uncertainties
incurred during the flavor tagging calibration procedures.
The uncertainties enter into the analyses through their impact on the scale-factors,
described in Section~\ref{sec:ftag_calib},
that are applied in the analysis.
Given the importance of $b$-tagged jets in the final states of the signal processes
in the analyses described in Chapters~\ref{chap:search_stop} and \ref{chap:search_hh},
these uncertainties have non-negligible impact on the analyses' results.

\subsubsection{Lepton Uncertainties}
Uncertainties on the measurement of leptons correspond to the electron and muon reconstruction,
identiication, trigger, and isolation {\color{red}{don't forget to describe lepton isolation}} efficiencies in a manner similar to the flavor tagging
in that systematic variations incurred in the associated scale-factor measurements are applied
in the analysis.
Additional uncertainties related to the lepton kinematics due to the resolution and scale of the
electron (muon) energy (momentum) measurement are considered.
The muon momentum measurement uncertainties are derived for both the ID and MS measurement
of the combined muons used in the analyses.

\subsubsection{Missing Transverse Momentum, \met}
Systematic variations of the \met are coherently incurred as a result of the
systematic variations, described above, being applied to the objects provided as input to the \met calculation:
the leptons and jets.
Additional uncertainties related to the scale and resolution of the soft-term of the \met calculation
are also considered.
Given that the analyses considered in Chapters~\ref{chap:search_stop} and \ref{chap:search_hh}
are characterised by real sources of \met, the soft-term component plays a small role in the magnitude
of the \met and therefore its uncertainties have negligible impact on the analyses.
Generally, the \met uncertainties are sub-dominant.

\subsubsection{Fake Lepton Estimate}

\begin{description}
    \item{Matrix Method (Section~\ref{sec:matrix_method}):} Hello World
    \item{Same-sign Extrapolation Method (Section~\ref{sec:same_sign_extrap}):}  YERP
\end{description}



%%%%%%%%%%%%%%%%%%%%%%%%%%%%%%%%%%%%%%%%%%%%%%%%%%%%%%%%%%%%%%%%%%%
%%%%%%%%%%%%%%%%%%%%%%%%%%%%%%%%%%%%%%%%%%%%%%%%%%%%%%%%%%%%%%%%%%%
%
% TRANSFER FACTOR METHOD
%
%%%%%%%%%%%%%%%%%%%%%%%%%%%%%%%%%%%%%%%%%%%%%%%%%%%%%%%%%%%%%%%%%%%
%%%%%%%%%%%%%%%%%%%%%%%%%%%%%%%%%%%%%%%%%%%%%%%%%%%%%%%%%%%%%%%%%%%
\subsection{The Transfer Factor Method}
\label{sec:transfer_factor}

For estimating the impact of modelling uncertainties on the SM backgrounds that
have dedicated CRs to constrain their overall normalisation in the SRs,
the so-called Transfer Factor (TF) Method is typically used.
Since the purpose of the CRs is to constrain the process' normalisation using
the observed data, we do not want the systematic variations to directly
impact the processes' normalisations within the SRs.
Instead, the impact of the systematic variations is assessed via their effect
on the SM processes' acceptance in the CRs and SRs.
If a given systematic variation for a given SM process affects the process
in a coherent manner across both the CR and SR, then the resulting affect of the systematic variation
on the analysis should be reduced.
This will nearly be guaranteed if the phase spaces being probed by the CR and SR
are similar.
The more dissimilar the phase space being probed by the CR and SR, the larger
the expected impact of a given systematic variation due to the larger kinematic
extrapolation required.
% for example, a systematic shift
%of some underlying modelling parameter may result in an upward fluctuation of the process'
%prediction in the CR but a downward shift in the SR.
This can be seen by considering Equation~\ref{eq:cr_tf}:
\begin{align}
    N_{p}^{\text{SR}} &= \mu_p \times N_{p,\,\text{MC}}^{\text{SR}} \nonumber \\
        &= \left( \frac{N_{p, \text{data}}^{\text{CR}}}{N_{p,\,\text{MC}}^{\text{CR}}} \right) \times N_{p,\,\text{MC}}^{\text{SR}} \nonumber \\
        &= N_{p, \text{data}}^{\text{CR}} \times \left( \frac{ N_{p,\,\text{MC}}^{\text{SR}}  }{ N_{p,\,\text{MC}}^{\text{CR}} } \right) \label{eq:cr_tf} \\
        &= N_{p, \text{data}}^{\text{CR}} \times \underbrace{\tau_p}_{\substack{\text{Transfer} \\ \text{ Factor}}} \nonumber,
\end{align}
where `$p$' is the process for which the CR and normalisation factor are defined, $\mu_p$
is the CR-derived normalisation factor (c.f. Equation~\ref{eq:mu_fac}),
$N_{p, \text{data}}^{\text{CR}}$ is the observed data in the CR with the MC simulation
for all processes that are not process $p$ subtracted, and $N_{p,\,\text{MC}}^{\text{SR}}$
is the MC-based SR prediction of process $p$.
The quantity $\tau_p$ is the process' TF that extrapolates the observed data in the CR
to the SR.
It can be seen that if the MC simulation response for a given process for a given systematic variation is the same
across both the CR and SR, that the TF will be unchanged as a result of the systematic
variation and therefore the predicted contribution of this process in the SR will
be unaffected by the systematic uncertainty.
If kinematics differ across the CR and SR, the acceptance for a given process may vary
in going from the CR and SR (or vice versa) and therefore such a cancellation is not likely to
occur due to the larger extrapolation required.

It can be seen, then, that for processes whose SR normalisation is derived in dedicated CRs,
that the TF appearing in Equation~\ref{eq:cr_tf} quantifies the acceptance
variation between the CR and SR.
Systematic variations of the SM backgrounds with dedicated CRs, then, are quantified
by their impact on the resulting TF values.
This is contrary to assessing their impact by measuring the change in a process' SR prediction
by simply comparing the SR predictions before and after a given systematic variation is applied.
The uncertainties ascribed to SM processes via the TF Method are computed as follows,
\begin{align}
    \Delta \tau = \frac{ \lvert \tau_{\,\text{nominal}} - \tau_{\,\text{variation}} \rvert} { \tau_{\,\text{nominal}} },
    \label{eq:tf_uncert}
\end{align}
where $\tau_{\,\text{nominal}}$ ($\tau_{\,\text{variation}}$) is the TF computed
using the nominal (systematically varied) prediction of the process.
The quantity $\Delta \tau$ is then taken as a fractional uncertainty on the corresponding
process' SR predicted yield.

%%%%%%%%%%%%%%%%%%%%%%%%%%%%%%%%%%%%%%%%%%%%%%%%%%%%%%%%%%%%%%%%%%%
%%%%%%%%%%%%%%%%%%%%%%%%%%%%%%%%%%%%%%%%%%%%%%%%%%%%%%%%%%%%%%%%%%%
%
% BACKGROUND MODELLING
%
%%%%%%%%%%%%%%%%%%%%%%%%%%%%%%%%%%%%%%%%%%%%%%%%%%%%%%%%%%%%%%%%%%%
%%%%%%%%%%%%%%%%%%%%%%%%%%%%%%%%%%%%%%%%%%%%%%%%%%%%%%%%%%%%%%%%%%%

\subsection{Background Modelling Uncertainties}
\label{sec:syst_bkg_modelling}

Uncertainties in the modelling of specific processes, SM or otherwise, are typically
assessed by comparing the nominal MC simulation for the processes in question to
that of an MC simulation with certain theoretical or phenomenological parameters varied.
In this way, one can assess the impact of the underlying assumptions made
in the MC simulation on the analyses' final results.

%%%%%%%%%%%%%%%%%%%%%%%%%%%%%%%%%%%%%%%%%%%%%%%%%%%%%%%%%%%%%%%%%%%
%%%%%%%%%%%%%%%%%%%%%%%%%%%%%%%%%%%%%%%%%%%%%%%%%%%%%%%%%%%%%%%%%%%
%
% SIGNAL MODELLING
%
%%%%%%%%%%%%%%%%%%%%%%%%%%%%%%%%%%%%%%%%%%%%%%%%%%%%%%%%%%%%%%%%%%%
%%%%%%%%%%%%%%%%%%%%%%%%%%%%%%%%%%%%%%%%%%%%%%%%%%%%%%%%%%%%%%%%%%%

\subsection{Signal Modelling Uncertainties}
\label{sec:syst_sig_modelling}
