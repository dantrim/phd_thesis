\section{Estimation of Sources of Fake and Non-prompt Leptons}
\label{sec:fakes}

Despite both the high levels of accuracy achieved by the ATLAS simulation
infrastructure and the lepton reconstruction and identification algorithms
described in Chapter~\ref{chap:objects}, sources of misidentified reconstructed
leptons still exist and lead to an additional source of backgrounds to
the analyses discussed in Chapters~\ref{chap:search_stop} and \ref{chap:search_hh}.
These background sources of leptons are broken down into two categories:
\begin{itemize}
    \item \textbf{Fake leptons}: Cases in which signals in the ATLAS detector
        are selected as being leptons when in fact there is no real lepton present
    \item \textbf{Non-prompt leptons}: When real, genuine leptons are identified
        but they are not leptons originating from the primary $pp$ hard-scatter interaction process
        of interest
\end{itemize}
In the subsequent discussion, the term `fake' will be used in reference to the two
categories listed above, unless specified otherwise.

The contribution of backgrounds leading to sources of fake leptons are generally predicted
using methods based on the observed data --- referred to as `data-driven' methods ---
and arise from various sources and mechanisms.
The sources of fake leptons will be described in Section~\ref{sec:fake_lepton_sources}, separately for
electrons and muons.
Section~\ref{sec:fake_dd_motivation} provides some reasoning for why a data-driven
approach is generally taken for estimating these backgrounds.
Sections~\ref{sec:matrix_method} and \ref{sec:same_sign_extrap} go on to describe
the two data-driven approaches taken in the analyses to be presented in this thesis
for estimating fake lepton contributions: the so-called `Matrix Method' and the `Same-sign Extrapolation Method', respectively.

%%%%%%%%%%%%%%%%%%%%%%%%%%%%%%%%%%%%%%%%%%%%%%%%%%%%%%%%%%%%%%%%%%%
%%%%%%%%%%%%%%%%%%%%%%%%%%%%%%%%%%%%%%%%%%%%%%%%%%%%%%%%%%%%%%%%%%%
%
% SOURCES OF FAKE LEPTONS
%
%%%%%%%%%%%%%%%%%%%%%%%%%%%%%%%%%%%%%%%%%%%%%%%%%%%%%%%%%%%%%%%%%%%
%%%%%%%%%%%%%%%%%%%%%%%%%%%%%%%%%%%%%%%%%%%%%%%%%%%%%%%%%%%%%%%%%%%
\subsection{Sources of Fake Leptons}
\label{sec:fake_lepton_sources}

The types and sources of fake leptons generally have different experimental signatures
than those leptons that genuinely originate from the $pp$ hard-scatter.
However, due to the non-perfect lepton identification and isolation algorithms,
such sources are able to contaminate the various regions of an analysis.
The rates of contamination are generally quite low for the analyses to be presented, but their inclusion in the background
estimates of the analyses has measurable consequences nevertheless.

The analyses to be presented in the current thesis make use of $b$-tagging algorithms
to identify jets originating from $b$-hadrons.
Fake leptons, both electrons and muons, can originate from the semi-leptonic decays of
$b$- and $c$-quarks within these $b$-tagged jets, following $b\rightarrow \ell$ or cascade-type
$b \rightarrow c \rightarrow \ell$ decays of the $B$ hadrons within the jets.
The leptons resulting from such decays are typically embedded within or very close to
the originating reconstructed jet object and the lepton isolation requirements
are intended to reduce this type of background.
The subsequent paragraphs will describe additional sources of fake electrons
and muons, which generally differ between the two lepton species.

%%%%%%%%%%%%%%%%%%%%%%%%%%%%%%%%%%%%%%%%%%%%%%%%%%%%%%%%%%%%%%%%%%%
%%%%%%%%%%%%%%%%%%%%%%%%%%%%%%%%%%%%%%%%%%%%%%%%%%%%%%%%%%%%%%%%%%%
%
% SOURCES OF FAKE ELECTRONS
%
%%%%%%%%%%%%%%%%%%%%%%%%%%%%%%%%%%%%%%%%%%%%%%%%%%%%%%%%%%%%%%%%%%%
%%%%%%%%%%%%%%%%%%%%%%%%%%%%%%%%%%%%%%%%%%%%%%%%%%%%%%%%%%%%%%%%%%%
\subsubsection{Sources of Fake Electrons}
\label{sec:fake_electron_sources}

As described in Section~\ref{sec:electrons}, electrons are reconstructed based
on the presence of well-reconstructed tracks in the ID matched to deposited
energy clusters in the EM calorimeter.
Light-flavor jets, originating from the production of light quarks ($u$, $d$, $s$),
or gluon jets, which are associated with a large number of tracks due to their
increased radiation pattern, are able to fake electrons as they leave
tracks in the ID as well as subsequent energy depositions in both the EM and hadronic calorimeters.
This background, due to mis-identified jets, is typically suppressed by the use
of lepton isolation and by jet shower-shape information used in the electron identification: the
hadronic shower shapes and radial extent differ with respect to the electromagnetic shower
produced by a genuine electron.

An additional large source of fake electrons is due to photon conversion processes,
$\gamma \rightarrow e^+ e^-$, and other electromagnetic scattering processes
that happen as a result of detector material interactions.
These processes leave both tracks in the ID and electromagnetic energy depositions
in the EM calorimeter which are difficult to distinguish from genuine electrons.
Neutral hadron decays, such as the $\pi^0 \rightarrow e^+ e^- \gamma$ Dalitz decay,
also lead to electron-like signatures.
This decay of the $\pi^0$ only has a branching fraction of just over $1\%$~\cite{PDGRef}, but given
the large production of $\pi^0$ states in the $pp$ collision this has the potential to be
a relevant source of fake electrons.
These electromagnetic sources of fake electrons are distinguished by their generally
larger impact parameters relative to genuine prompt electrons.


%%%%%%%%%%%%%%%%%%%%%%%%%%%%%%%%%%%%%%%%%%%%%%%%%%%%%%%%%%%%%%%%%%%
%%%%%%%%%%%%%%%%%%%%%%%%%%%%%%%%%%%%%%%%%%%%%%%%%%%%%%%%%%%%%%%%%%%
%
% SOURCES OF FAKE MUONS
%
%%%%%%%%%%%%%%%%%%%%%%%%%%%%%%%%%%%%%%%%%%%%%%%%%%%%%%%%%%%%%%%%%%%
%%%%%%%%%%%%%%%%%%%%%%%%%%%%%%%%%%%%%%%%%%%%%%%%%%%%%%%%%%%%%%%%%%%
\subsubsection{Sources of Fake Muons}
\label{sec:fake_muon_sources}

As described in Section~\ref{sec:muons}, muons are primarily reconstructed via the combination
of tracking information provided by the ID and MS, and, generally speaking, they should be the only particle species to reach
the MS.
In addition to the semi-leptonic decays of heavy-flavored jets described above, however,
there are several sources of fake muons.
Highly energetic jets can have elongated shower profiles that reach the outer
radii of the hadronic calorimeter, with a non-zero chance of exiting the calorimeter
and resulting in particle leakage into the MS.
Such cases are referred to as calorimeter punch-through, and have been illustrated
in Figure~\ref{fig:jet_punch_through}.
Punch-through particles can leave signatures similar to charged muons whose subsequent
MS tracks are associated with a track in the ID, leading to a reconstructed combined muon
faking a genuine muon.
An additional source of fake muons come from the in-flight decays of charged hadrons,
such as the $K^\pm$ that can decay to $\mu^{\pm} \nu$.
Non-prompt muons arising as a result of such in-flight decays are typically
characterised by combined tracks that exhibit a kink topology, mentioned briefly
in the discussion of muon combined reconstruction in Section~\ref{sec:muon_id} and
illustrated in Figure~\ref{fig:fake_muon_kink}.

\begin{figure}[!htb]
    \begin{center}
        \includegraphics[width=0.65\textwidth]{figures/common_ana/fake_muon_kinkPDF}
        \caption{
            Illustration of a reconstructed non-prompt muon resulting from a kinked-track topology.
            A promptly produced hadron is produced and decays to a final state including a
            high-\pT muon.
            The point at which the hadron decays is indicated by the yellow dot.
            An example of such a decay is the decay $K^+ \rightarrow \mu^+ \nu$.
            The red circles indicate detector hits in the ID and MS layers indicated
            by the horizontal black lines.
            Implied particle lifetimes and detector sizes are not to scale.
        }
        \label{fig:fake_muon_kink}
    \end{center}
\end{figure}


%%%%%%%%%%%%%%%%%%%%%%%%%%%%%%%%%%%%%%%%%%%%%%%%%%%%%%%%%%%%%%%%%%%
%%%%%%%%%%%%%%%%%%%%%%%%%%%%%%%%%%%%%%%%%%%%%%%%%%%%%%%%%%%%%%%%%%%
%
% DATA DRIVEN MOTIVATION
%
%%%%%%%%%%%%%%%%%%%%%%%%%%%%%%%%%%%%%%%%%%%%%%%%%%%%%%%%%%%%%%%%%%%
%%%%%%%%%%%%%%%%%%%%%%%%%%%%%%%%%%%%%%%%%%%%%%%%%%%%%%%%%%%%%%%%%%%
\subsection{The Need for a Data-driven Approach}
\label{sec:fake_dd_motivation}

In the analyses to be presented in Chapters~\ref{chap:search_stop} and \ref{chap:search_hh},
relatively tight identification working points are used for electrons and muons.
As a result, the contamination of fake leptons in these analyses is relatively
minor.
Although small, their contamination does have measurable effects and so their contribution
must be accounted for in order to achieve accurate estimates of the backgrounds
in each analysis.

Several methods exist to estimate the background rates arising from the sources of fake
leptons, those relying on data-driven methods or those based entirely on the MC
simulation.
Relying on the MC simulation of these sources of fake leptons, described in previous
sections, means to rely entirely on the \textsc{GEANT4} simulation of the ATLAS detector
and on the MC generation and showering processes to accurately predict the
rates of these processes.
There are several problems with this approach and they are (nonexhaustively) as follows.
Given the very small region of phase space being probed by the analysis,
the number of MC events needed to appropriately sample the sources of production of fake
leptons as described above would be prohibitively large if a statistically relevant sample
is desired.
An accurate prediction of the production rates of several of these fake lepton sources
would require an accurate underlying theoretical model of many processes, such as
heavy-flavor jet fragmentation, which is challenging.
Additionally, many sources of fake leptons arise as a result of detector material interactions
or as a result of subtle and difficult-to-model failure modes of the detector response.
%or inaccurate simulation of the detector response.
The accurate prediction of the rate of photon conversions, for example, requires
high levels of precision in the simulation and measurement of the active and passive material in the ATLAS detector and cavern,
which is not necessarily possible.
The rates of jets being mis-identified as electrons and jet punch-through, for example,
require that the MC simulation of the calorimeter response and shower evolution are
accurately modelled.
The MC simulation is not expected to perform to the degree at which these subtle, and comparatively rare, effects
are accurately predicted.
For this reason, data-driven approaches are typically taken for estimating the background rates
of these fake lepton sources.
In the analyses to be presented, two data-driven approaches are taken.
In the search described in Chapter~\ref{chap:search_stop}, the Matrix Method
is used.
In the search described in Chapter~\ref{chap:search_hh}, the Same-sign Extrapolation
Method is used.
These methods are introduced in Sections~\ref{sec:matrix_method} and \ref{sec:same_sign_extrap}, respectively.


%%%%%%%%%%%%%%%%%%%%%%%%%%%%%%%%%%%%%%%%%%%%%%%%%%%%%%%%%%%%%%%%%%%
%%%%%%%%%%%%%%%%%%%%%%%%%%%%%%%%%%%%%%%%%%%%%%%%%%%%%%%%%%%%%%%%%%%
%
% THE MATRIX METHOD
%
%%%%%%%%%%%%%%%%%%%%%%%%%%%%%%%%%%%%%%%%%%%%%%%%%%%%%%%%%%%%%%%%%%%
%%%%%%%%%%%%%%%%%%%%%%%%%%%%%%%%%%%%%%%%%%%%%%%%%%%%%%%%%%%%%%%%%%%
\subsection{The Matrix Method}
\label{sec:matrix_method}

%%%%%%%%%%%%%%%%%%%%%%%%%%%%%%%%%%%%%%%%%%%%%%%%%%%%%%%%%%%%%%%%%%%
%%%%%%%%%%%%%%%%%%%%%%%%%%%%%%%%%%%%%%%%%%%%%%%%%%%%%%%%%%%%%%%%%%%
%
% SAME SIGN EXTRAPOLATION
%
%%%%%%%%%%%%%%%%%%%%%%%%%%%%%%%%%%%%%%%%%%%%%%%%%%%%%%%%%%%%%%%%%%%
%%%%%%%%%%%%%%%%%%%%%%%%%%%%%%%%%%%%%%%%%%%%%%%%%%%%%%%%%%%%%%%%%%%
\subsection{Same-sign Extrapolation Method}
\label{sec:same_sign_extrap}


