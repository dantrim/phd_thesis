\section{Jets}
\label{sec:jets}

Due to the confining nature of QCD, color-charged quarks and gluons produced as a result of
initial $pp$ interactions do not exist as free states for observably meaningful
timescales and therefore do not leave unambiguous signatures in the detector.
Instead, their production is characterised by the bremsstrahlung-like radiation of additional
quarks and gluons roughly collinear with the initiating colored particles.
The radiation pattern of these colored objects is dictated by the color field
that binds them and eventually results in the production of color-neutral hadrons.
The collimated spray of hadrons as a result of this \textit{hadronisation} process
leads to the phenomenology of \textit{jets}, which are the macroscopically observable signature
of produced quarks and gluons.
The reconstruction of jets refers to any suitable, i.e. physically meaningful and stable,
method for grouping together, or \textit{clustering}, the end-products of the hadronisation
process in such a way that the properites of the initiating quarks or gluons, such
as their quantum numbers and/or kinematics, can be inferred from the resulting clustered object.
The standard method for reconstructing jets in ATLAS will be introduced in Section~\ref{sec:jet_reco}.
In Section~\ref{sec:jet_calib}, the steps taken to turn these reconstructed jets into
accurate representations of the initiating quarks and/or gluons, so that they can
be used meaningfully in physics analyses, will be discussed.

\subsection{Jet Reconstruction}
\label{sec:jet_reco}

\subsection{Jet Calibration}
\label{sec:jet_calib}
