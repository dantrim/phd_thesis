\subsection{Flavor Tagging of Jets}
\label{sec:flavor_tagging}

The ability to identify jets containing heavy-flavored hadrons, i.e. jets containing
$b$- and $c$-flavored hadrons, is an important aspect to many of the most critical measurements
and analyses being done at the large LHC experiments.
The SM top-quark is the heaviest known elementary particle, the second-most recent elementary particle to be discovered,
and, given its importance to electroweak and Higgs physics, is an object subjected to high levels
of precise study at the LHC.
The top-quark decays before hadronisation timescales and therefore its decay products, which
are a $W$-boson and a $b$-quark nearly 100\% of the time, carry away information characterising its properties.
Being able to characterise the jets initiated by the hadronisation of these $b$-quarks, then,
is of critical import if the physics of the top-quark wish to be understood.
The recent discovery of an SM-like Higgs boson, with a mass of $m_h = 125\,\GeV$, decays
to a pair of $b$-quarks nearly 60\% of the time.
Without question, then, the thorough study of the Higgs boson necessitates the ability to precisely identify
the presence of the pair of $b$-initiated jets from its decays.
Additionally, we will see in subsequent chapters that the presence of $b$-quark initiated jets is
a characteristic signature of many BSM physics scenarios.
The identification of these types of jets, a process referred to as `flavor tagging', is of the utmost importance
to analyses performed with the ATLAS detector as well as in the work to be presented in this thesis.
Jets tagged as having likely arisen as a result of the hadronisation of an initiating $b$-($c$-)quark
are referred to as `$b$-tagged jets` (`$c$-tagged jets'), or simply as `$b$-jets` (`$c$-jets').
All other jets then are assumed to have arisen from the decay of light-flavor quarks and are referred to as `light-flavor jets' (`light-jets').

Heavy-flavor tagging of jets relies on the relatively long lifetimes of the $b$- and $c$-hadrons which
initiate them. The typical $b$-hadron lifetime is $\tau \approx 1.6$\,ps ($c\tau \approx 450\,\micron$),
which leads to $b$-hadrons traversing typically macroscopic distances away from the primary hard-scatter vertex
before they decay.
As seen in Figure~\ref{fig:bhadron_decay_length}, $b$-hadrons with transverse momenta on the order of
$50\,\GeV$ will travel nearly half a centimeter before decaying.
Also seen in Figure~\ref{fig:bhadron_decay_length}, those $b$-hadrons with \pT~values nearing $250\,\GeV$ will actually decay
outside of the beam-pipe within the region of the IBL.
Given the high spatial resolution of the ID pixel and SCT detectors discussed in Section~\ref{sec:inner_detector}, the presence of these long-lived particles
should be detectable via the presence of at least one secondary decay vertex corresponding to the point at which
the heavy-flavored hadron decays.
Figure~\ref{fig:bjet_decay} illustrates the standard topology of a heavy-flavor initiated jet with a secondary decay vertex that is displaced
with respect to the primary hard-scatter vertex and is the source of displaced tracks.

The algorithms used to identify heavy-flavor initiated jets, then, rely on information characterising
the long lifetimes of the initiating particles and on the presence of secondary decay vertices within the jet.
These high-level algorithms used to identify these jets take as input information provided by taggers that rely on low-level information based on
the presence of displaced tracks and secondary vertices.
These low-level taggers will be introduced in Section~\ref{sec:ftag_low_level} and the construction of the
high-level tagger, used in the analyses to be presented in this thesis, will be presented in Section~\ref{sec:ftag_high_level}.


%%%%%

%As in Ref.~\cite{ATL-PHYS-PUB-2017-013}.


\begin{figure}[!htb]
    \begin{center}
        \includegraphics[width=0.7\textwidth]{figures/bhadron_decay_length_ibl}
        \caption{
            Particle decay length as a function of its lifetime and transverse momentum normalised
            to its rest mass.
            The white-dashed line indicates the average lifetime of $B$-hadron species, taken
            as 1.6\,ps, with a mass taken to be 5.5\,GeV.
            The red-dots along the $B$-hadron line indicate locations for specific transverse momenta
            for the decaying $B$-hadron.
            The yellow contours indicate the locations of the IBL (Figure~\ref{fig:pixel_detector_trans}),
            with 2.43\,cm corresponding to the beam-pipe radius.
            {\color{red}{Perhaps move this plot elsewhere?}}
            {\color{red}{Add references to PDG}}
        }
        \label{fig:bhadron_decay_length}
    \end{center}
\end{figure}

\begin{figure}[!htb]
    \begin{center}
        \includegraphics[width=0.7\textwidth]{figures/chapter3/ftag/bhadron_decayPDF}
        \caption{
            Topology of a $b$-jet.
            The $b$-hadron produced near the primary hard-scatter vertex (green dot), initiating the
            $b$-jet, has a long lifetime and decays a macroscopic distance away from the primary
            hard-scatter vertex to produce a secondary vertex (red dot) from which additional tracks
            are produced and subsequently reconstructed.
            The tracks originating from the secondary vertex will have larger impact parameters relative
            to the primary hard-scatter vertex as compared to tracks originating from the primary
            hard-scatter vertex.
        }
        \label{fig:bjet_decay}
    \end{center}
\end{figure}

%\FloatBarrier
\subsubsection{Low Level Taggers and Inputs}
\label{sec:ftag_low_level}

\begin{figure}[!htb]
    \begin{center}
        \includegraphics[width=0.32\textwidth]{figures/chapter3/ftag/ftag_track_d0_sig_ip2d}
        \includegraphics[width=0.32\textwidth]{figures/chapter3/ftag/ftag_ip2d_pb}
        \includegraphics[width=0.32\textwidth]{figures/chapter3/ftag/ftag_sv1_fE}
        \caption{
            Examples of a few low-level quantities used in the ATLAS flavor tagging algorithms.
            The blue histograms are distributions associated with $b$-jets, green are those of $c$-jets, and red
            are those of light-flavor jets.
            \textit{Left}: Two-dimensional (signed) $d_0$ significance for tracks matched to jets.
            \textit{Middle}: IP2D $b$-jet log-likelihood ratio.
            \textit{Right}: Energy fraction, defined as the energy of the tracks in the displaced
                vertex reconstructed by the SV1 algorithm relative to the energy of all tracks in the jet.
%            Figures taken from Ref.~\ref{ATL-PHYS-PUB-2016-022}.
        }
        \label{fig:ftag_low_level_var}
    \end{center}
\end{figure}

%\FloatBarrier
\subsubsection{High Level Tagger: MV2c10}
\label{sec:ftag_high_level}

\begin{figure}[!htb]
    \begin{center}
        \includegraphics[width=0.5\textwidth]{figures/chapter3/ftag/ftag_mv2c10_disc}
        \caption{
            Distribution of the BDT-based MV2c10 $b$-tagging algorithm output score, shown for
            $b$-jets (blue), $c$-jets (green), and light-flavor jets (red).
%            Figures taken from Ref.~\ref{ATL-PHYS-PUB-2016-012}.
        }
        \label{fig:ftag_mv2c10_disc}
    \end{center}
\end{figure}

\begin{table}[!htb]
    \caption{
        Variables used as input to the high-level tagger MV2c10.
        From Ref.~\cite{ATL-PHYS-PUB-2015-022}.
    }
    \label{tab:ftag_mv2_inputs}
    \begin{scriptsize}
    \begin{center}
    \begin{tabularx}{\textwidth}{|X|l|X|}
    \hline
    \hline
    \textbf{Input Source} & \textbf{Input Name} & \textbf{Description} \\
    \hline
    \multirow{2}{*}{Kinematics} & $\pT(\text{jet})$ & Jet transverse momentum \\
    \cline{2-3}
                & $\eta(\text{jet})$ & Jet pseudorapidity \\
    \hline
    \multirow{3}{*}{IP2D, IP3D} & $\log(p_b/p_{\text{light}})$ & Likelihood ratio between the $b$- and light-jet hypotheses \\
    \cline{2-3}
                & $\log(p_b / p_c)$ & Likelihood ratio between the $b$- and $c$-jet hypotheses \\
    \cline{2-3}
                & $\log(p_c / p_{\text{light}})$ & Likelihood ratio between the $c$- and light-jet hypotheses \\
    \hline
    \multirow{8}{*}{Secondary Vertex} & $m_{\text{SV}}$ & Invariant mass of tracks at the secondary vertex assuming pion masses \\
    \cline{2-3}
            & $f_E(\text{SV})$ & Fraction of the charged jet energy in the secondary vertex \\
    \cline{2-3}
            & $N_{\text{TrkAtVtx}}(\text{SV})$ & Number of tracks used in the secondary vertex \\
    \cline{2-3}
            & $N_{\text{2TrkVtx}}(\text{SV})$ & Number of two-track vertex candidates \\
    \cline{2-3}
            & $L_{xy}(\text{SV})$ & Transverse distance between the primary and secondary vertices \\
    \cline{2-3}
            & $L_{xyz}(\text{SV})$ & Distance between the primary and secondary vertices \\
    \cline{2-3}
            & $S_{xyz}(\text{SV})$ & Distance between the primary and secondary vertices divided by its uncertainty \\
    \cline{2-3}
            & $\Delta R(\text{jet, SV})$ & $\Delta R$ between the jet axis and the direction of the secondary vertex relative to the primary vertex \\
    \hline
    \multirow{8}{*}{JetFitter} & $N_{\text{2TrkVtx}}(\text{JF})$ & Number of two-track vertex candidates (prior to JetFitter decay-chain fit) \\
    \cline{2-3}
            & $m(\text{JF})$ & Invariant mass of tracks from displaced vertices assuming pion masses \\
    \cline{2-3}
            & $S_{xyz}(\text{JF})$ & Significance of the average distance between the primary and displaced vertices \\
    \cline{2-3}
            & $f_E(\text{JF})$ & Fraction of the charged jet energy in the secondary vertices \\
    \cline{2-3}
            & $N_{\text{1-trk vertices}}(\text{JF})$ & Number of displaced vertices with one track \\
    \cline{2-3}
            & $N_{\ge\text{2-trk vertices}}(\text{JF})$ & Number of displaced vertices with more than one track \\
    \cline{2-3}
            & $N_{\text{TrkAtVtx}}(\text{JF})$ & Number of tracks from displaced vertices with at least two tracks \\
    \cline{2-3}
            & $\Delta R(\vec{p}_{\text{jet}}, \vec{p}_{\text{vtx}})$ & $\Delta R$ between the jet axis and the vectorial sum of the momentum of all tracks attached to displaced vertices \\
    \hline
    \hline
    \end{tabularx}
    \end{center}
    \end{scriptsize}
\end{table}



\FloatBarrier
