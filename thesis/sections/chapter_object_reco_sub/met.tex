\section{The Missing Transverse Momentum}
\label{sec:met}

The colliding protons within each successful $pp$ interaction are assumed to be colliding
\textit{head on}, with momentum only parallel to the beam-axis.
Momentum conservation then implies that the vectorial sum of the transverse momenta of all objects
originating from the primary hard-scatter vertex should be exactly zero.
The missing transverse momentum, \ptmiss, whose magnitude is denoted by \met, is calculated
each event and its components are defined as the negative vectorial sum of the
reconstructed objects associated with the event~\cite{METPaper},
\begin{align}
    E_{x\,(y)}^{\text{miss}} = E_{x\,(y)}^{\text{miss},\,\text{electron}} + E_{x\,(y)}^{\text{miss},\,\text{photon}} + E_{x\,(y)}^{\text{miss},\,\tau} + E_{x\,(y)}^{\text{miss},\,\text{jets}} + E_{x\,(y)}^{\text{miss},\,\text{muon}} + E_{x\,(y)}^{\text{miss},\,\text{soft}},
    \label{eq:met_def}
\end{align}
where each of the $E_{x\,(y)}^{\text{miss},\,i}$ terms ($i \in \{ \text{electron}, \text{photon}, \tau, \text{jets}, \text{muon} \}$)
are the negative sum of the momenta in the $x$- ($y$-) direction for the respective (calibrateD) objects discussed in previous sections.
The $E_{x\,(y)}^{\text{miss},\,\text{soft}}$ term is track-based and is reconstructed from the transverse momentum of 
reconstructed tracks originating from the primary hard-scatter vertex but not associated with any of the
hard objects indicated in the other terms appearing in Equation~\ref{eq:met_def}.
The total \met and its direction in aziumuth is then given by,
\begin{align}
    \met = \sqrt{ \left(E_x^{\text{miss}} \right)^2 + \left( E_y^{\text{miss}} \right)^2 }, \\
    \phi^{\text{miss}} = \arctan \left( E_y^{\text{miss}} / E_x^{\text{miss}} \right).
\end{align}

Any imbalance in the visible momenta in a given event will result in nonzero \met.
Nonzero values of \met may indicate the presence of weakly interacting particles that do not leave
detectable signatures in ATLAS, such as neutrinos from leptonically decaying $W$ bosons
in the case of the SM, or may indicate the presence of additionaly weakly interacting particles
such as the weakly interacting dark matter candidates that many BSM physics scenarios predict.
Precise measurement of the \met, then, is of the utmost importance to many searches for BSM physics.
Searches for $R$-parity conserving Supersymmetry (Section~\ref{sec:susy}) are commonly characterised
by large values of \met due to a multiplicity of stable weakly interacting SUSY particle in the final state.
Additionally, detector mismodelling, detecor noise, limited detector coverage, or miscalibration of the reconstructed objects used in the reconstruction
of \ptmiss can also contribute nonzero values to \met.
The measurement resolution of the \met is also susceptible to pileup effects, and generally
degrades as the levels of pileup increase.
