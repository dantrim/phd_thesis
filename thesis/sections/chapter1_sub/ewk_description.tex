\section{The Electroweak Theory}
\label{sec:ewk_description}

It was the work of Glashow, Weinberg, and Salam (GWS) that ultimately put forth a
consistent picture of the chiral weak force and, ultimately,
its unification with electromagnetism~\cite{Glashow:1961tr,Weinberg:1967tq,Salam:1968rm}.
As a result, the theory of particles and fields that respect the \SUtwoXone~gauge
invariance of the SM is sometimes referred to as `GWS theory', 
but is more typically known as the electroweak theory, or interaction. Since all matter particles
are subject to the electroweak interaction, but only a subset of the particles that
have color charge are subject to the strong interaction described by QCD, the study of the SM can essentially
be partitioned into two parts: the part that deals with the dynamics and interactions of
colored objects (the `QCD part', $\mathcal{L}_{\text{QCD}}$) and the part that deals solely with electroweak interactions, including the Higgs (the `Electroweak part`, $\mathcal{L}_{\text{Electroweak}}$).
Given the ubiquity and broad reach into all observable phenomena of the electroweak interaction,
in the early days the GWS theory was thought of as the heart of the SM and why, ultimately,
GWS were awarded the Nobel prize in 1979.\footnote{Actually, the acceptance of the GWS theory as the
de-facto SM of the time was not widely held until some years after its publication, when t'Hooft
proved that it was renormalizable~\cite{tHooft:1971akt,tHooft:1971qjg}.
Such a complete understanding in the QCD sector would not come until almost a decade later, in the late 1970's{\color{red}{Woijkec etc CITE}}.} In this section, then, we will focus closely on a part of the  \SUtwoXone~portion of \SML.

The first thing to re-iterate is that the electroweak theory is \textit{chiral}, i.e., it distinguishes
between left- and right-chiral fermion fields. For some conceptual clarity, in the limit of
a massless fermion, the chirality is the same as the more-familiar \textit{helicity}, which
is defined by the projection of the particles spin angular momentum, $\vec{\sigma}$, onto their momentum (direction of motion): $\vec{\sigma}\cdot \vec{p}$. If the massless fermion's spin
is aligned parallel to its momentum ($\vec{\sigma}\cdot \vec{p} >0$), it is said to be right-handed. If otherwise ($\vec{\sigma}\cdot \vec{p} <0$), it is said to be left-handed. One typically defines
fermion fields without the chiral specification and then defines projections onto the left-
and right-handed components. For example:
\begin{equation*}
	f_{\text{L}} = P_L f = \frac{1}{2}(1- \gamma_5)f, \hspace{0.6cm} f_{\text{R}} = P_R f = \frac{1}{2}(1+\gamma_5)f
	\label{eq:chiral_projection}
\end{equation*}
First we introduce re-defined \fieldW~ bosons as follows:
\begin{align}
	\mathcal{W}^+ &= \frac{1}{\sqrt{2}} ( -\mathcal{W}_1 + i\mathcal{W}_2)\notag \\
	\mathcal{W}^- &= \frac{1}{\sqrt{2}} ( -\mathcal{W}_2 - i\mathcal{W}_2) \label{eq:w_redefine} \\ 
	\mathcal{W}^0 &= \mathcal{W}_3 \notag
    %\fieldW^+ = \frac{1}{\sqrt{2}}(-\fieldWone + i\fieldWtwo)
\end{align}
Since the electroweak interaction affects the leptons and quarks similarly, and the same goes for
fermions across generations (e.g. \fieldEl $\leftrightarrow$ \fieldMul $\leftrightarrow$ \fieldTaul),
we focus only on the leptons and consider for simplicity only a single generation.
Taking the associated \Uone~terms of Eqn.~\ref{eq:sm_lagrangian}, we get:
\begin{align}
    -\mathcal{L}_{\text{ferm}}(\mathcal{U}(1), \text{leptons}) &= \bar{L} i \gamma^{\mu} (i g_1 \frac{Y_L}{2} \mathcal{B}_{\mu})L + \bar{e}_R i \gamma^{\mu} (i g_1 \frac{Y_R}{2} B_{\mu}) e_R \nonumber \\
    &= \frac{g_1}{2} [ Y_L ( \bar{\nu}_L \gamma^{\mu} \nu_L + \bar{e}_L \gamma^{\mu} e_L) + Y_R \bar{e}_R \gamma^{\mu} e_R ] B_{\mu}
    \label{eq:ferm_L_u1}
\end{align}
Where we use the fact that $L = (\nu_L, e_L)$ in going from the first to second line of Eqn.~\ref{eq:ferm_L_u1}.
Taking the associated \SUtwo~terms of Eqn.~\ref{eq:sm_lagrangian}, and noting
that, since the $\tau^i$ are the Pauli matrices, $\tau^i W^i$ is a $2\times2$ matrix, we get:
\begin{align}
	-\mathcal{L}_{\text{ferm}}(\mathcal{SU}(2), \text{leptons}) &= \bar{L} i \gamma^{\mu} [i g_2 \frac{\tau^i}{2} \mathcal{W}_{\mu}^i ] L \notag \\
	&= -\frac{g_2}{2} (\bar{\nu}_L, \bar{e}_L) \gamma^{\mu} \left( \begin{matrix} \mathcal{W}_{\mu}^3 & \mathcal{W}_{\mu}^1 - i\mathcal{W}_{\mu}^2 \\ \mathcal{W}_{\mu}^1 + i \mathcal{W}_{\mu}^2  & - \mathcal{W}_{\mu}^3\end{matrix} \right) \left( \begin{matrix} \nu_L \\ e_L \end{matrix} \right) \notag \\
	&= -\frac{g_2}{2} \left[ \bar{\nu}_L \gamma^{\mu} \nu_L \mathcal{W}_{\mu}^0 - \sqrt{2}  \bar{\nu}_L \gamma^{\mu} e_L \mathcal{W}_{\mu}^+ - \sqrt{2} \bar{e}_L \gamma^{\mu} \nu_L \mathcal{W}_{\mu}^- - \bar{e}_L \gamma^{\mu} e_L \mathcal{W}_{\mu}^0 \right]
    \label{eq:ferm_L_su2}
\end{align}
Where Eqn.~\ref{eq:w_redefine} has been used in getting to the last line of Eqn.~\ref{eq:ferm_L_su2}.
Considering only those terms which are charge-neutral in Eqn.~\ref{eq:ferm_L_u1} and Eqn.~\ref{eq:ferm_L_su2}, one can consider performing a second field-redefinition using
\fieldB~ and \fieldWzero:
\begin{align}
	A_{\mu} &= \frac{g_2 \mathcal{B}_{\mu} - g_1 Y_L \mathcal{W}_{\mu}^0}{\sqrt{g_2^2 + g_1^2 Y_L^2}} \hspace{1cm} Z_{\mu} = \frac{g_1 Y_L \mathcal{B}_{\mu} + g_2 \mathcal{W}_{\mu}^0}{\sqrt{g_2^2 + g_1^2 Y_L^2}} \notag \\
	\hookrightarrow B_{\mu} &= \frac{g_2 A_{\mu} + g_1 Y_L Z_{\mu}}{\sqrt{g_2^2 + g_1^2 Y_L^2}} \hspace{1cm} \mathcal{W}_{\mu}^0 = \frac{-g_1 Y_L A_{\mu} + g_2 Z_{\mu}}{\sqrt{g_2^2 + g_1^2 Y_L^2}}
\end{align}
Using this last result for the re-defined \fieldB~and \fieldWzero~in terms the
fields $A_{\mu}$ and $Z_{\mu}$ to re-write the terms in Eqn.~\ref{eq:ferm_L_u1} and~\ref{eq:ferm_L_su2} involving $\bar{e}_{L,R} \gamma^{\mu} e_{L,R}$, allows
one to see that $A_{\mu}$ should correspond to the photon of electromagnetism if we fix,
\begin{align}
	-e = \frac{g_1 g_2 Y_L} {\sqrt{g_2^2 + g_1^2 Y_L^2}} \hspace{1cm} -e = \frac{g_1 g_2 Y_R}{2 \sqrt{g_2^2 + g_1^2 Y_L^2}} \nonumber
\end{align}
where $e$ is the charge of the positron. Setting $Y_L = -1$ then allows to formulate
the positron charge in terms of the weak couplings $g_1$ and $g_2$ simply as,
\begin{align}
	\label{eq:e_charge_coupling}
	e = \frac{g_1 g_2}{\sqrt{g_1^2 + g_2^2}}
\end{align}
Eqn.~\ref{eq:e_charge_coupling} leads one to introduce the following relations,
\begin{align}
	\sin \theta_W = \frac{g_1}{\sqrt{g_1^2 + g_2^2}} \hspace{1cm} \cos \theta_W = \frac{g_2}{\sqrt{g_1^2 + g_2^2}}
\end{align}
The angle $\theta_W$ is the \textit{Weinberg angle}, and it represents the amount of mixing
occuring between the \SUtwo~and \Uone~gauge fields $\mathcal{W}_{\mu}^0$ and $\mathcal{B}_{\mu}$, respectively. Using Eqn.~\ref{eq:e_charge_coupling} gives,
\begin{align}
	g_1 = \frac{e}{\cos \theta_W} \hspace{1cm} g_2 = \frac{e}{\sin \theta_W}
\end{align}
This defines the gauge couplings, $g_1$ and $g_2$, of the electroweak theory purely
in terms of known or measurable quantities.\footnote{We do not describe this in detail, but $\theta_W$
can in principle be determined once the masses of the $\mathcal{W^{\pm}}$ and $\mathcal{Z}$ bosons
are measured.}

From the above algebra, we can re-write the portion of the electroweak Lagrangian describing the
interactions of the first-generation fermions with the electroweak gauge bosons as,
\begin{align}
	\mathcal{L}_{\text{ferm, first-gen.}} &= \sum_{\limits{f \in \nu_e, e, u, d}} e Q_f \left( \bar{f}\gamma^{\mu} f \right) A_{\mu} \notag \\
	&+ \sum_{\limits{f \in \nu_e, e, u, d}} \frac{g_2}{\cos \theta_W} \left[ \bar{f}_L \gamma^{\mu} f_L \left( T_f^3 - Q_f \sin^2 \theta_W \right) + \bar{f}_R \gamma^{\mu} f_R \left( - Q_f \sin^2 \theta_W \right) \right] Z_{\mu} \label{eq:ewk_L_za} \\
	&+ \frac{g_2}{\sqrt{2}} \left[ \left( \bar{u}_L \gamma^{\mu} d_L + \bar{\nu}_L \gamma^{\mu} e_L \right) \mathcal{W}_{\mu}^+ + \text{hermitian conjugate} \right] \notag
\end{align}
The related terms for the second and third generations of the fermions,
($\nu_{\mu}, \mu, c, s$) and ($\nu_{\tau}, \tau, t, b$), respectively, are identical.
In Eqn.~\ref{eq:ewk_L_za}, $T_f^3$ ($Q_f$) is the third component of the weak-isospin (electric charge)
of the fermion species $f$. These last terms are related via the Gell-Mann-Nishijima relation,
\begin{align}
	Q_f = T_f^3 + \frac{1}{2}Y
	\label{eq:gell_mann_nishijima}
\end{align}
which can be deduced by following the algebra above, and our having fixed $Y_L = -1$ for
the $(\nu_L, e_L)$ \SUtwo~doublet.

Note that the terms involving \fieldWpm~in Eqn.~\ref{eq:ewk_L_za} are of the form $\bar{\nu}_L \gamma^{\mu} e_L$ which, given the chiral projects of Eqn.~\ref{eq:chiral_projection}, can be
re-written as follows,
\begin{align}
	\bar{\nu}_L \gamma{\mu} e_L = \frac{1}{2} \bar{\nu} \gamma^{\mu}(1-\gamma_5) e
	\label{eq:v_minus_a}
\end{align}
which shows that the charged weak interactions involving \fieldWpm~are the coherent
sum of vector ($\gamma^{\mu}$) and axial-vector bilinear covariants ($\gamma^{\mu}\gamma_5$); this is the famous \textit{V-A} charged-current interaction of Fermi's nuclear $\beta$-decay.

What we have shown in this section is that, due to mixing of the SM \SUtwo$_L$
and \Uone$_Y$ gauge fields, we can arrive at an electroweak theory that supports
the known fact that their exists an electromagnetic force that is mediated
by a neutral boson $A_{\mu}$ (the photon) that couples to electrically charged fields: this is what
is shown in the first line of Eqn.~\ref{eq:ewk_L_za}. The field re-definitions described above
also introduce the \fieldWpm~boson as the mediators of the charged electroweak interaction,
responsible for radioactivity, and an additional neutral electroweak interaction mediated
by the \fieldZ~boson.
The fact that the \SUtwo$_L$ and \Uone$_Y$ gauge fields mix, by an amount dictated by $\theta_W$,
suggest that the weak and electromagnetic interactions can be unified into the single
electroweak interaction, as mentioned at the beginning of this section. Later on, we will see
that (gauge) unification such as this plays a large role in our current understanding
of how the universe works.

We have thus shown that the SM predicts the existence of the familiar electromagnetic force,
which is not immediately apparent based on \SML~of Eqn.~\ref{eq:sm_lagrangian}. 
%However, 
%neither the fermions nor the \fieldWpm~or \fieldZ~ appear to have mass -- they still
%have no mass terms appearing in their respective sectors of \SML.
However, it is still not evident how it can
support the experimental fact that fermions have mass and that the mediators of the
weak-nuclear force (the \fieldWpm) \textit{must} be massive, a fact known
even before the formulation of QED. No such mass terms
have been provided for in the SM Lagrangian yet. To resolve this, we
need the Higgs mechanism.