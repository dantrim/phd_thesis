\subsection{The Inner Detector}
\label{sec:inner_detector}

The innermost subdetector of ATLAS is the Inner Detector (ID)~\cite{Haywood:331064}.
The ID covers the region $\lvert \eta \rvert < 2.5$ and is composed, in order
of increasing radial distance from the beam-pipe, of the pixel detector,
semiconductor tracker (SCT), and the transition radiation tracker (TRT).
These detectors enable the reconstruction of the tracks associated with
the $\mathcal{O}(1000)$ charged particles emerging from each $pp$ bunch collision occuring
every 25\,ns.
An illustration of the ID and its subdetectors is shown in Figure~\ref{fig:atlas_inner_detector}.
Additional, more detailed views of the barrel and endcap sections of the ID are shown in Figure~\ref{fig:atlas_ID_exploded}.
The ID is situated inside of the central solenoid, indicated in Figure~\ref{fig:atlas_magnet_system},
which provides an axial 2\,T magnetic field and extends over a length of 5.3\,m with a diameter of 2.5\,m.
The bending of charged particles in the $xy$-plane due to the presence of the solenoidal
field allows for their momenta to be measured using the curvature of their reconstructed tracks.

\begin{figure}[!htb]
    \begin{center}
        \includegraphics[width=0.75\textwidth]{figures/chapter2/atlas_inner_detector}
        \caption{
            Cross-sectional view of the ATLAS inner detector. Shown are the barrel
            and end-cap portions of the pixel, SCT, and TRT detectors.
        }
        \label{fig:atlas_inner_detector}
    \end{center}
\end{figure}

\begin{figure}[!htb]
    \begin{center}
        \includegraphics[width=0.6\textwidth]{figures/chapter2/atlas_ID_barrel_exploded}
        \raisebox{1.4cm}{\includegraphics[width=0.85\textwidth]{figures/chapter2/endcap_ID_exploded}}
        \caption{
            Exploded views of the barrel (\textit{left}) and endcap (\textit{right}) portions
            of the inner-detector.
        }
        \label{fig:atlas_ID_exploded}
    \end{center}
\end{figure}

\subsubsection{The Pixel Detector and IBL}
\label{sec:id_pixel}

The pixel detector is the innermost subdetector of the ID, situated very near to and surrounding
the beam-pipe.
It is composed of three separate sections: a barrel section and two end-cap sections.
The barrel section  of the pixel detector has a cylindrical geometry and the end-cap sections
are disks centered on the beam-pipe.
The barrel section has four layers, each with increasing radius, and there are three disks in each
of the end-caps. This ID geometry, shown in Figure~\ref{fig:atlas_ID_exploded}, covers
the region $\lvert \eta \rvert < 2.5$.

The pixel detector, being so near the $pp$ collisions, is subject to the highest particle
fluxes of any other subsystem.
As a result, it is built to have very fine granularity: its sensing elements consist of
$250$\,\micron~thick detectors housing pixels of reverse-biased n-type semiconductor material,
each having a nominal size of $50\times400\,\micron^2$.
In total, there are roughly 80 million channels read out from the pixel detector alone.
This allows for the pixel detector's fine spatial hit resolution of $10\,\micron$ in
$(r-\phi)$ and $115\,\micron$ along $z$.

The innermost layer of the pixel detector's barrel section is referred to as the
\textit{Insertable B-Layer} (IBL), and was installed at the beginning of the Run-II
data-taking period~\cite{Capeans:1291633}.
It corresponds, essentially, to the instrumentation of the ATLAS beam-pipe, as seen in Figure~\ref{fig:pixel_detector_trans},
and is located at a radial distance of 3.3\,cm.
It alone accounts for 8 million readout channels of
the pixel detector --- resulting in an ultra precise spatial hit resolution of $8\,\micron$ in $(r-\phi)$ and
$40\,\micron$ along $z$.
Beyond improving the overall measurements and reconstruction of charged particle tracks,
the IBL was installed in order to improve the performance of secondary vertex
reconstruction --- an essential ingredient to the algorithms associated with
the reconstruction and identification of jets originating from the decays
of $b$-hadrons whose decays occur at radial distances frequently beyond that
of the IBL.

\begin{figure}[!htb]
    \begin{center}
        \includegraphics[width=0.8\textwidth]{figures/chapter2/pixel_detector_trans}
        \caption{
            Transverse view of the barrel section of the pixel detector, showing
            the innermost layer, the Insertable B-Layer (IBL) (red), and the
            three surrounding layers (blue). From Ref.~\cite{Backhaus:2016ctq}.
        }
        \label{fig:pixel_detector_trans}
    \end{center}
\end{figure}

\subsubsection{The Semiconductor Tracker}
\label{sec:id_sct}

\subsubsection{The Transition Radiation Tracker}
\label{sec:trt}
