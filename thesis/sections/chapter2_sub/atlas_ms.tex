\subsection{The Muon Spectrometer}
\label{sec:ms}

Surrounding the calorimeters is the muon spectrometer (MS)~\cite{CERN-LHCC-97-022}, responsible
for the detection of high-momentum, minimum-ionizing muons originating from the $pp$ interaction.
The MS is based on the magnetic deflection of muon tracks, allowing for their
momentum determination.
The bending of the muon trajectories is provided by the large
superconducting air-core toroid magnet system, illustrated in Figure~\ref{fig:atlas_magnet_system},
consisting of a large barrel toroid over the range $\lvert \eta \rvert < 1.4$
and end-cap toroid systems in the range $1.6 < \lvert \eta \rvert < 2.7$.
The superconducting toroid magnet provides an average field of $4\,$T.
The magnetic field bending strength is roughly constant in $\eta$, except in the
region in which the transition between the barrel and end-cap toroids takes place
($1.4 < \lvert \eta \rvert < 1.6$).
A view of the ATLAS detector is shown in Figure~\ref{fig:atlas_in_cavern},
where it can be seen that the volume enclosed by the MS takes up most of the available volume
outside of the calorimeter systems in the underground experimental cavern at Point 1.
It should be noted that the overall design of the superconducting toroid structure,
dictated by the performance requirements of the MS, is what gives ATLAS its large size and essentially
drove the original design of all subdetectors discussed in the previous sections.

There are four types of gaseous radiation detector used in the MS, and their chamber
layout is based on the concept of projective towers.
The chambers follow the structure of the toroid magnet structure and have a 16-fold segmentation
in azimuth, shown in Figure~\ref{fig:muon_segmentation}.
They are arranged in large and small sectors, with the large sectors covering
the regions between the coils of the toroid and the small sectors the azimuthal range
in which the coils sit.
The detector types can be broken into two classes and are either
\textit{precision} or \textit{trigger} chambers.
The precision chambers allow for
the precise measurement the muon tracks as they traverse the MS, specifically the
precise measurement in the bending plane of these tracks so as to allow for accurate
determination of the muon momenta through their curvature.
The trigger chambers have fast signal formation and readout times, allowing for
accurate assignment of a passing muon to a specific $pp$ bunch crossing.
Both types of detectors exist in the barrel and end-cap \textit{stations} of the
MS and there are typically at least three layers of precision-type chambers over the
entire $\lvert \eta \rvert$ range of the MS in order to allow for the sagitta measurement
of the muon tracks necessary for momentum determination.
The layout of these detectors, in both the barrel and end-cap, is shown in
Figure~\ref{fig:muon_plan_view_eta}.

\begin{figure}[!htb]
    \begin{center}
        \includegraphics[width=0.8\textwidth]{figures/chapter2/atlas_in_cavern}
        \caption{
            A view of the ATLAS detector inside the underground experimental area
            UX15.
            The cut-away view exposes the toroid structure as well as the
            calorimeter system.
            Notice that the outermost muon stations in the forward regions are located
            at the extreme ends of the cavern.
            {\color{red}{Should move this figure either above or entirely}}
        }
        \label{fig:atlas_in_cavern}
    \end{center}
\end{figure}

\begin{figure}[!htb]
    \begin{center}
        \includegraphics[width=0.4\textwidth]{figures/chapter2/muon_spec/atlas_muon_barrel}
        \includegraphics[width=0.35\textwidth]{figures/chapter2/muon_spec/atlas_muon_endcap}
        \caption{
            View of the 16-fold segmentation of the muon spectrometer in the barrel (\textit{left})
            and end-cap (\textit{right}).
            Clearly seen in both is the arrangment of the detector chambers into large and
            small sectors, allowing for complete coverage in azimuth.
            The view of the end-cap is that only of the MDT chambers located at $z\approx13$\,m.
        }
        \label{fig:muon_segmentation}
    \end{center}
\end{figure}

\begin{figure}[!htb]
    \begin{center}
        \includegraphics[width=0.8\textwidth]{figures/chapter2/muon_spec/atlas_muon_plan_view_eta}
        \caption{
            A view in the $r-z$ plane of a quadrant of the muon spectrometer (MS).
            Indicated by color are the four detector technologies used in the MS:
            MDT (blue), RPC (grey), TGC (red), and CSC (yellow).
            The light grey boxes at $6 < r < 9$\,m indicate the location of the
            barrel toroid structures.
            Also shown are the envelopes in $\lvert \eta \rvert$ of the barrel,
            small wheel, and big wheel sections of the MS.
        }
        \label{fig:muon_plan_view_eta}
    \end{center}
\end{figure}
\FloatBarrier


\begin{figure}[!htb]
    \begin{center}
        \includegraphics[width=0.7\textwidth]{figures/chapter2/muon_spec/atlas_muon_overlap}
        \caption{
        }
        \label{fig:muon_overlap}
    \end{center}
\end{figure}

\begin{figure}[!htb]
    \begin{center}
        \includegraphics[width=0.7\textwidth]{figures/chapter2/muon_spec/atlas_ms_nchamber_crossed}
        \caption{
        }
        \label{fig:muon_nchambers_crossed}
    \end{center}
\end{figure}


\subsubsection{Precision Muon Chambers}
\label{sec:muon_precision}

\begin{figure}[!htb]
    \begin{center}
        \includegraphics[width=0.5\textwidth]{figures/chapter2/muon_spec/mdt_chamber}
        \raisebox{1.22cm}{\includegraphics[width=0.32\textwidth]{figures/chapter2/muon_spec/mdt_trackfit}}
        \caption{
            \textit{Left}: Illustration of a double-multilayer MDT chamber with its internal alignment
                and support structure exposed. A zoom-in on the multilayer of MDT tubes is shown.
            \textit{Right}: Illustration of the multilayer MDT tracklet-fitting algorithm~\cite{MDTtrackfit}.
        }
        \label{fig:mdt_chamber}
    \end{center}
\end{figure}

\begin{figure}[!htb]
    \begin{center}
        \includegraphics[width=0.55\textwidth]{figures/chapter2/muon_spec/csc_chamber}
        \caption{
            Diagram showing the main components of a cathode-strip chamber (CSC).
            On the \textit{left} (\textit{right}) is a view parallel (perpendicular) to the anode
            wires and perpendicular (parallel) to the cathode strips.
        }
        \label{fig:csc_chamber}
    \end{center}
\end{figure}

\subsubsection{Muon Trigger Chambers}
\label{sec:muon_trigger}

\begin{figure}[!htb]
    \begin{center}
        \includegraphics[width=0.5\textwidth]{figures/chapter2/muon_spec/rpc_chamber}
        \includegraphics[width=0.38\textwidth]{figures/chapter2/muon_spec/tgc_chamber}
        \caption{
            \textit{Left}: Illustration of a resistive plate chamber (RPC) and its principle of operation.
            \textit{Right}: Diagram showing the main components of a thin-gap chamber (TGC).
        }
        \label{fig:muon_trigger_chamber}
    \end{center}
\end{figure}
