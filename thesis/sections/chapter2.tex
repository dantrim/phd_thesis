\chapter{Experimental Setup}

%\epigraph{\textit{So it goes...}}{---Kurt Vonnegut, \textit{Slaughterhouse
%		Five}}
	
%\epigraph{\textit{Science is a miracle.}}{--Ron Swanson}

%\epigraph{\textit{If you wish to make an apple pie from scratch, you must first invent the universe.}}{--Carl Sagan, \textit{Cosmos: A Personal Voyage}}
\epigraph{\textit{Nice piece of wood in that counter. Nicely planed. Like the way it curves there.}}{--Leopold Bloom, in James Joyce's \textit{Ulysses}}
%\epigraph{\textit{The movements which work revolutions in the world are born
%out of the dreams and visions in a peasant's heart on the hillside.}}{--``Leopold Bloom'', in \textit{Ulysses} by James Joyce}

The work to be described in the present thesis was done at CERN\footnote{
The acronym CERN was historically derived from `\textit{Conseil europ{\'e}en pour la recherche
nucl{\'e}aire'}. Nowadays, `CERN' has become a standalone name for the lab itself and
is currently referred to as the `\textit{Organisation europ{\'e}enne pour la recherche nucl{\'e}aire}'; or, in English: the
`\textit{European Organisation for Nuclear Research.}'}, the particle
physics laboratory located along the French-Swiss border just outside of Geneva, Switzerland.
CERN is comprised of almost 18,000 personnel, of which over 13,000 are researchers in the
field of experimental particle physics.
It is a truly international workplace, with the personnel comprised of representatives of over 110 nationalities
and who are either working directly
for CERN\footnote{Of the roughly 18,000 researchers in experimental particle physics, only about
5\% are employed directly by CERN itself.} or for their respective home institutions
--- universities or national labs ---
located in more than 70 countries~\cite{CERN-HR-STAFF-STAT-2018}.
These researchers will generally work at any of the independent experiments located along the various
beamlines that network throughout the CERN campus (see Fig.~\ref{fig:cern_complex}).

As the present author is a member of one of the two general-purpose experiments at CERN located
along the Large Hadron Collider (LHC) -- the ATLAS experiment -- this chapter will present a
brief introduction to the workings of the LHC (Section~\ref{sec:lhc}) and then describe in some
detail the various components that make up the ATLAS detector (Section~\ref{sec:atlas}), the largest
and most complex scientific piece of equipment ever 
constructed by humans.\footnote{The ATLAS detector, along with its operation, is by far more complex
than any previous human endeavour --- generally more complex than anything operated and enacted by NASA, for
example. The only difference being the tolerance for failure: in the case of NASA space-based experiments and missions
this tolerance approaches zero, whereas the terrestrial particle physics experiments happening at the
LHC are generally accessible and amenable to errors.}


\begin{figure}[!htb]
    \begin{center}
        \includegraphics[width=0.8\textwidth]{figures/chapter2/cern_accelerator_complex2}
        \caption{
            Illustration of the various beamlines, accelerator and storage rings, and experimental
            points that the CERN accelerator complex is home to.
            The protons that circulate through the LHC, and that are eventually made to collide inside
            the ATLAS detector, follow the path: Linac 2 $\rightarrow$ Booster $\rightarrow$ Proton Synchotron (PS)
            $\rightarrow$ Super Proton Synchotron (SPS) $\rightarrow$ LHC.
        }
        \label{fig:cern_complex}
    \end{center}
\end{figure}


%%%%%%%%%%%%%%%%%%%%%%%%%%%%%%%%%%%%%%%%%%%%%%%%%%%%%%%%%%%%%%%%%%%
%%%%%%%%%%%%%%%%%%%%%%%%%%%%%%%%%%%%%%%%%%%%%%%%%%%%%%%%%%%%%%%%%%%
% sub-section describing the LHC
%%%%%%%%%%%%%%%%%%%%%%%%%%%%%%%%%%%%%%%%%%%%%%%%%%%%%%%%%%%%%%%%%%%
%%%%%%%%%%%%%%%%%%%%%%%%%%%%%%%%%%%%%%%%%%%%%%%%%%%%%%%%%%%%%%%%%%%
\section{The Large Hadron Collider}
\label{sec:lhc}

The LHC~\cite{Evans_2008} is a circular particle accelerator with a 27~kilometer ($\approx17$ miles)
circumference located, on average, approximately 100 meters beneath the Earth's surface. It is nominally
a proton-proton ($pp$) collider
but can also be run in heavy-ion configurations: proton-lead ($p$-Pb), lead-lead (Pb-Pb), or even
proton-gold ($p$-Au). It is designed to accelerate protons to a center-of-mass
energy of $\sqrt{s} = 14\,\TeV$.

To avoid the exorbitant costs in civil engineering and real-estate works associated with
constructing an even larger tunnel, it was decided that the LHC should be housed in the already-existing
tunnel that housed the Large Electron Positron (LEP) collider, in operation from 1989 to 2000.
LEP, a \textit{particle-antiparticle} collider, was able to take advantage of the fact that
 particle and anti-particle beams can be made to occupy the same phase space within a single ring: the same magnetic
fields could produce counter-rotating electron (negatively charged) and positron (positvely charged) beams.



\begin{figure}[!htb]
    \begin{center}
        \includegraphics[width=0.8\textwidth]{figures/chapter2/lhc_layout}
        \caption{
            Layout of the LHC and its two counter-rotating beams. Beam 1 is in blue and rotates
            counter-clockwise. Beam 2 is in red and rotates clock-wise.
            At the center of each octant is a straight section which houses
            the experimental caverns or LHC beam facilities.
            At the boundaries of each octant are located the curved sections.
            Figure taken from Figure 2.1 of Ref.~\cite{Evans_2008}.
        }
        \label{fig:lhc_layout}
    \end{center}
\end{figure}

\begin{figure}[!htb]
    \begin{center}
        \includegraphics[width=0.5\textwidth]{figures/chapter2/lhc_dipole_fig3p3}
        \caption{
        }
        \label{fig:lhc_dipole_xsec}
    \end{center}
\end{figure}

\subsection{Injection Chain}
\label{sec:lhc_injection}

\subsection{The Concept of Luminosity}
\label{sec:lhc_luminosity}

The Large Hadron Collider (LHC) can be thought of as the final part of the particle-beam injection line
that is comprised of many parts whose goal is to accelerate protons, or other particles, to
the energies requisite for CERN's large experiments to do perform fundamental physics research
at the high-energy frontier.


%%%%%%%%%%%%%%%%%%%%%%%%%%%%%%%%%%%%%%%%%%%%%%%%%%%%%%%%%%%%%%%%%%%
%%%%%%%%%%%%%%%%%%%%%%%%%%%%%%%%%%%%%%%%%%%%%%%%%%%%%%%%%%%%%%%%%%%
% sub-section describing ATLAS
%%%%%%%%%%%%%%%%%%%%%%%%%%%%%%%%%%%%%%%%%%%%%%%%%%%%%%%%%%%%%%%%%%%
%%%%%%%%%%%%%%%%%%%%%%%%%%%%%%%%%%%%%%%%%%%%%%%%%%%%%%%%%%%%%%%%%%%
\section{The ATLAS Detector}
\label{sec:atlas}

