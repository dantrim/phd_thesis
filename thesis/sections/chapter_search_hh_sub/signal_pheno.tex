\section{The Dilepton $hh \rightarrow \bbww$ Signal Process}
\label{sec:hh_pheno}

%{\color{red}{MENTION SCALAR WW DECAY AND WW SPIN CORRELATION THAT ONLY THE DILEPTON CHANNEL CAN TAKE ADVANTAGE OF}}

In the dilepton decay of the $hh \rightarrow \bbww$ channel, the $WW^*$ system
cannot be fully reconstructed in order to produce, for example, a $WW^*$ invariant
mass observable that has a resonance structure at $m_h = 125$\,GeV.
This is due to the
kinematically under-constrained final state in which the \met is a result of the double neutrino system
arising from the leptonically decaying $W$ bosons.
Additionally, the final state is the same as that of SM \ttbar~production.
As a result, the dominant SM background to the analysis is expected to be SM \ttbar~and top-quark processes.

There are crucial \textit{topological} differences between the observable final states of dilepton SM \ttbar~production and
that of \bbww, however.
The dileptonic \bbww final state is characterised by `Higgs hemispheres'.
One hemisphere contains the two $b$-tagged jets from the $h \rightarrow bb$ decay and is generally
opposite in the transverse ($xy$) plane to the second hemisphere that contains the two leptons
and \met from the $h \rightarrow WW^*$ decay.
In the case of \ttbar, the $b$-tagged jets, leptons, and neutrinos tend to be distributed more uniformly
within the event, leading to events that do not typically exhibit the same back-to-back hemispheres
topology as the \bbww signal process.
This is illustrated in Figure~\ref{fig:hh_topo}.

An additional feature of the $hh \rightarrow \bbww$ decay, available only in the dilepton final state, is related to the spin-structure
of the $h \rightarrow WW^*$ decay.
The Higgs boson is a fundamental scalar particle with spin-0 while the $W$ bosons are spin-1 particles.
In the $h \rightarrow WW^*$ decay, then, the spins of the $W$ bosons must be oppositely aligned in order
to conserve angular momentum.
The spin-information encoded in the anti-aligned $W$-boson spins is preserved in their leptonic decays, governed by
the parity-violating electroweak interaction.
This is illustrated in Figure~\ref{fig:hdecay_ww}, showing the $h \rightarrow WW^* \rightarrow \ell \nu \ell \nu$
decay chain for two possible orientations of the $W$ boson spins.
The parity-violating weak decays require that the $\nu$ ($\bar{\nu}$) be left-(right-)handed which ultimately
forces the charged leptons to emerge from the $W$-boson decays in the same direction.
As a result, the charged leptons from the $WW^*$ decays will be relatively collinear,
as will the neutrinos, which further enhances the Higgs hemisphere signature of the dilepton $hh \rightarrow \bbww$ final state.
These effects are most clearly observed in the dilepton invariant mass, $m_{\ell \ell}$, and
in the opening angle between the two leptons, $\dphill$, shown in Figure~\ref{fig:hh_kin_1}.
The analyses studying single Higgs production in the $h \rightarrow WW^*$ channel rely primarily on these effects,
particularly on their impact on the $m_{\ell \ell}$ distribution, in order to separate out the
Higgs events from the SM backgrounds~\cite{Aaboud:2018jqu,Aad:2016lvc,ATLAS:2014aga}.


The presence of these Higgs hemispheres in the signal, in which the momentum flow of the $WW^*$ system
is correlated with that of the $bb$ system's, will be relied upon to inform the analysis' choice
of kinematic observables that are sensitive to the dilepton $hh \rightarrow \bbww$ signal process.

\begin{figure}[!htb]
    \begin{center}
        \includegraphics[width=0.52\textwidth]{figures/search_hh/signal_pheno/ttbar_topo}
        \includegraphics[width=0.46\textwidth]{figures/search_hh/signal_pheno/hh_topo}
        \caption{
            Dilepton $bbWW$ shapes.
            \textit{\textbf{Left}}: SM top-quark pair production.
            \textit{\textbf{Right}}: $hh \rightarrow \bbww$.
        }
        \label{fig:hh_topo}
    \end{center}
\end{figure}

\begin{figure}[!htb]
    \begin{center}
        \includegraphics[width=0.75\textwidth]{figures/search_hh/signal_pheno/hdecay_wwPDF}
        \caption{
            Illustration of the correlation in lepton momentum direction as a result of the decay of the
            spin-zero Higgs boson to $W$-boson pairs, which decay via the parity-volating electroweak interaction
            to the final state leptons.
            Double-lined arrows indicate the direction in which the particle travels and the
            dashed red arrows indicate the orientation of the particle spin-angular momentum relative to their direction of travel.
            \textit{\textbf{Top}}: Scenario with $W$-boson spins outward.
            \textit{\textbf{Bottom}}: Scenario with $W$-boson spins inward.
        }
        \label{fig:hdecay_ww}
    \end{center}
\end{figure}

\begin{figure}[!htb]
    \begin{center}
        \includegraphics[width=0.45\textwidth]{figures/search_hh/signal_pheno/shape_plots/hh_shape_plot_mll}
        \includegraphics[width=0.45\textwidth]{figures/search_hh/signal_pheno/shape_plots/hh_shape_plot_dphi_ll}
        \caption{
            Normalized distributions showing the shapes of kinematic distributions for the SM
            top-quark processes (\ttbar~and single-top $Wt$) and the dilepton $hh \rightarrow \bbww$ signal process.
            \textit{\textbf{Left}}: Dilepton system invariant mass, $m_{\ell \ell}$.
            \textit{\textbf{Right}}: $|\Delta \phi|$ between the two leptons, $|\dphill|$.
        }
        \label{fig:hh_kin_1}
    \end{center}
\end{figure}

%%%%%%%%%%%%%%%%%%%%%%%%%%%%%%%%%%%%%%%%%%%%%%%%%%%%%%%%%%%%%%%%%%%%%%%%%%%%%%%%%%%
%%%%%%%%%%%%%%%%%%%%%%%%%%%%%%%%%%%%%%%%%%%%%%%%%%%%%%%%%%%%%%%%%%%%%%%%%%%%%%%%%%%
%%%%%%%%%%%%%%%%%%%%%%%%%%%%%%%%%%%%%%%%%%%%%%%%%%%%%%%%%%%%%%%%%%%%%%%%%%%%%%%%%%%
%
% KINEMATIC DISTRIBUTIONS/SHAPES
%
%%%%%%%%%%%%%%%%%%%%%%%%%%%%%%%%%%%%%%%%%%%%%%%%%%%%%%%%%%%%%%%%%%%%%%%%%%%%%%%%%%%
%%%%%%%%%%%%%%%%%%%%%%%%%%%%%%%%%%%%%%%%%%%%%%%%%%%%%%%%%%%%%%%%%%%%%%%%%%%%%%%%%%%
%%%%%%%%%%%%%%%%%%%%%%%%%%%%%%%%%%%%%%%%%%%%%%%%%%%%%%%%%%%%%%%%%%%%%%%%%%%%%%%%%%%

In the following we present additional kinematic distributions illustrating the unique topologies
of the dilepton $hh \rightarrow \bbww$ signal process as compared to the dominant top-quark background
processes.

Figure~\ref{fig:hh_kin_0} shows distributions related to the $bb$ system.
It can be seen from the $|\dphibb|$ and \drbb distributions that the two $b$-tagged jets
are collinear in the signal process as compared to the background processes.
For the \ttbar~background, these tendencies in $|\dphibb|$ and \drbb are reversed,
as expected from their different topology with the $b$-tagged jets on opposing sides of the event.
For single-top $Wt$, similar trends as observed in \ttbar are observe but, since the second $b$-tagged
jet in the $Wt$ process is due to higher-order contributions (or a light-flavor jet mistakenly identified as a $b$-jet),
the $b$-tagged jets are often more collinear.
The invariant mass of the leading two $b$-tagged jets, $m_{bb}$, is nicely peaking at the mass of the Higgs
boson for the signal, as expected.

%Figure~\ref{fig:hh_kin_1} shows distributions illustrating the collinear nature of the leptons arising from
%the scalar decay of the $WW^*$ system.
%As a result of the spin-correlation, described above, the leptons are collinear and result in small values
%of $m_{\ell \ell}$ ($\dphill$) in the $hh$ signal process.

Figure~\ref{fig:hh_kin_2} shows a few distributions related to the dilepton, \met, and $\met$ + dilepton systems.
We can see from the $|\dphimetll|$ and \drll distributions that for the signal $hh$ process, the leptons
are collinear as expected, with the leptons contained almost entirely within a cone of radius $\drll = 1$.
For the top-quark backgrounds, there is a tendency for the leptons to be back-to-back, with $|\dphill| (\drll) \rightarrow \pi$.

The $|\Delta \phi|$ between the \met and dilepton system, $|\dphimetll|$, shows that the $WW^*$ system is collinear, as well.
For the SM top-quark backgrounds, $|\dphimetll|$ tends slowly towards $\pi$.
In the case of single-top $Wt$, however, there is a tendency for $|\dphimetll|$ to gather towards lower values due
to similar effects as described previously: the more collinear $b$-tagged jets force the
$WW^*$ system to also become collinear so that it balances the $bb$ system.

\begin{figure}[!htb]
    \begin{center}
        \includegraphics[width=0.45\textwidth]{figures/search_hh/signal_pheno/shape_plots/hh_shape_plot_dphi_bb}
        \includegraphics[width=0.45\textwidth]{figures/search_hh/signal_pheno/shape_plots/hh_shape_plot_dRbb}
        \includegraphics[width=0.45\textwidth]{figures/search_hh/signal_pheno/shape_plots/hh_shape_plot_mbb}
        \caption{
            Normalized distributions showing the shapes of kinematic distributions for the SM
            top-quark processes (\ttbar~and single-top $Wt$) and the dilepton $hh \rightarrow \bbww$ signal process.
            \textit{\textbf{Left}}: $|\Delta \phi|$ between the two $b$-tagged jets, $|\dphibb|$.
            \textit{\textbf{Right}}: $\Delta R$ between the two $b$-tagged jets, $\drbb$.
            \textit{\textbf{Bottom}}: Invariant mass of the leading two $b$-tagged jets, $m_{bb}$.
        }
        \label{fig:hh_kin_0}
    \end{center}
\end{figure}


\begin{figure}[!htb]
    \begin{center}
        \includegraphics[width=0.45\textwidth]{figures/search_hh/signal_pheno/shape_plots/hh_shape_plot_dRll}
        \includegraphics[width=0.45\textwidth]{figures/search_hh/signal_pheno/shape_plots/hh_shape_plot_met}
        \includegraphics[width=0.45\textwidth]{figures/search_hh/signal_pheno/shape_plots/hh_shape_plot_dphi_met_ll}
        \caption{
            Normalized distributions showing the shapes of kinematic distributions for the SM
            top-quark processes (\ttbar~and single-top $Wt$) and the dilepton $hh \rightarrow \bbww$ signal process.
            \textit{\textbf{Left, upper}}: $|\Delta \phi|$ between the two leptons, $|\dphill|$.
            \textit{\textbf{Right, upper}}: $\Delta R$ between the two leptons, $\drll$.
            \textit{\textbf{Left, lower}}: Magnitude of the missing transverse momentum, \met.
            \textit{\textbf{Right, lower}}: $\Delta \phi$ between \met and dilepton system, $|\dphimetll|$.
        }
        \label{fig:hh_kin_2}
    \end{center}
\end{figure}

%%%%%%%%%%%%%%%%%%%%%%%%%%%%%%%%%%%%%%%%%%%%%%%%%%%%%%%%%%%%%%%%%%%%%%%%%%%%%%%%%%%
%%%%%%%%%%%%%%%%%%%%%%%%%%%%%%%%%%%%%%%%%%%%%%%%%%%%%%%%%%%%%%%%%%%%%%%%%%%%%%%%%%%
%%%%%%%%%%%%%%%%%%%%%%%%%%%%%%%%%%%%%%%%%%%%%%%%%%%%%%%%%%%%%%%%%%%%%%%%%%%%%%%%%%%
%
% TARGETTED OBSERVABLES
%
%%%%%%%%%%%%%%%%%%%%%%%%%%%%%%%%%%%%%%%%%%%%%%%%%%%%%%%%%%%%%%%%%%%%%%%%%%%%%%%%%%%
%%%%%%%%%%%%%%%%%%%%%%%%%%%%%%%%%%%%%%%%%%%%%%%%%%%%%%%%%%%%%%%%%%%%%%%%%%%%%%%%%%%
%%%%%%%%%%%%%%%%%%%%%%%%%%%%%%%%%%%%%%%%%%%%%%%%%%%%%%%%%%%%%%%%%%%%%%%%%%%%%%%%%%%

To take further advantage of the topological differences mentioned in the preceding, we can construct
an observable that is sensitive to the overall distribution of the momentum flow in the event.
This is to capture the more `planar' nature of the dilepton $hh \rightarrow \bbww$ decay as compared
to that of the SM top-quark backgrounds.
This observable, \htratio, is defined as follows,
\begin{align}
    \htratio &= \frac{\htnum}{\htden},
    \label{eq:ht2ratio_def}
\end{align}
where the $\bm{p}_{\text{T}}^{\ell(b),0 \{1\}}$ are the transverse momenta of the leading \{subleading\} lepton ($b$-tagged jet).

The numerator of Equation~\ref{eq:ht2ratio_def} contains two parts, each the magnitude of a vectorial sum of transverse
momenta: the first part containing the objects associated with the $WW^*$ decay and the second with those associated
with the $bb$ decay.
For the dilepton $hh \rightarrow \bbww$ signal, then, the numerator is the sum of the magnitudes of each of the Higgs boson's
momenta.
The denominator appearing in Equation~\ref{eq:ht2ratio_def} is the scalar sum of the transverse momenta of each of the
final state objects considered in teh analysis.
By taking this ratio, a measure of the overall spread of the momentum in the event is obtained.

For the relevant SM top-quark processes, with the visible final state objects distributed more evenly
about the event, the quantities appearing in the numerator of Equation~\ref{eq:ht2ratio_def} will
tend to be smaller, due to vectorial cancellation, than in the case of the $hh$ signal.
Due to the Higgs hemispheres in the signal, with separately collinear $WW^*$ and $bb$ systems,
the numerator in Equation~\ref{eq:ht2ratio_def} will be large.
It follows, then, that the quantity \htratio should tend towards unity for the dilepton $hh \rightarrow \bbww$
signal process and towards lower values for the SM top-quark processes.
This behavior is illustrated in Figure~\ref{fig:hh_shape_htratio}.

\begin{figure}[!htb]
    \begin{center}
        \includegraphics[width=0.6\textwidth]{figures/search_hh/signal_pheno/shape_plots/hh_shape_plot_HT2Ratio}
        \caption{
            Normalized distributions showing the shape of the \htratio observable for the SM
            top-quark processes (\ttbar~and single-top $Wt$) and the dilepton $hh \rightarrow \bbww$ signal process.
        }
        \label{fig:hh_shape_htratio}
    \end{center}
\end{figure}

We also consider a variant of the \mttwo variable~\cite{MT2-Glamour,Lester2011,MT2-Tovey-Masses,Lester2014yga},
a transverse-mass variable that can be used in events where there are two or more invisible particles,
as is our case with the two neutrinos.
Generally, \mttwo assumes that the decay is symemtric, as in \ttbar, where there are two decay legs, each with
a visible and invisible child particle decay.
The \mttwo observable is generically defined as follows,
\begin{align}
    \mttwo^2 \equiv \min\limits_{ \bm{q}_{\text{T}}^{\text{miss},\,a} + \, \bm{q}_{\text{T}}^{\text{miss},\,b} = \,\bm{p}_{\text{T}}^{\text{miss}}}
        \left[
            \max
                \left\{
                    m_{\text{T}}^2 ( \bm{p}_{\text{T}}^{\text{vis},\, a}, \bm{q}_{\text{T}}^{\text{miss},\,a} ), m_{\text{T}}^2 (\bm{p}_{\text{T}}^{\text{vis},\, b}, \bm{q}_{\text{T}}^{\text{miss},\,b} )
                \right\}
        \right],
    \label{eq:mttwo_def}
\end{align}
where `a' (`b') indicate one of the two sides of the decay and the $\bm{p}_{\text{T}}^i$ are the transverse momenta
of the visible objects on side $i \in (a,b)$, and $\bm{q}_{\text{T}}^i$ is a partition of the total \ptmiss given to side $i \in (a,b)$.
The minimization is carried out over all partionings of the \ptmiss into the $\bm{q}_{\text{T}}^i$ and the $m_{\text{T}}$ are the typical
transverse masses,
\begin{align}
    m_{\text{T}}^2 (\bm{p}_{\text{T}}^{\text{vis}}, \bm{q}_{\text{T}}^{\text{miss}} ) = ( E_{\text{T}}^{\text{vis}} + | \bm{q}_{\text{T}}^{\text{miss}} |)^2
            - |\bm{p}_{\text{T}}^{\text{vis}} + \bm{q}_{\text{T}}^{\text{miss}} |^2
    \label{eq:mt_def}
\end{align}

The \mttwo-based observable defined for the present analysis is referred to as `\mtbb',
and is an \mttwo construction in which the visible objects are the two $b$-tagged jets in the event.
This means that side `a' (`b'), using the notation of Equation~\ref{eq:mttwo_def}, contains
one of the $b$-tagged jets as $\bm{p}_{\text{T}}^{\text{vis, a}}$ ($\bm{p}_{\text{T}}^{\text{vis}, b}$).

Given the fact that in the SM top-quark backgrounds, the neutrinos (approximated by the $\bm{q}_{\text{T}}^{\text{miss}, i}$)
and the $b$-tagged jets originate from the same origin (a top-quark), the \mtbb observable tends
to have a kinematic bound at roughly the mass of the top-quark, $m_{\text{top}}$.
In the SM \ttbar~process, where both $b$-tagged jets originate from the decay of a top-quark, this kinematic
endpoint is strict.
In the case of single-top $Wt$, however, where only one $b$-tagged jet arises from the decay of a
top-quark, the kinematic endpoint at $m_{\text{top}}$ is not so strict and the \mtbb observable can extend beyond $m_{\text{top}}$.
Figure~\ref{fig:hh_shape_mtbb} illustrates the \mtbb observable.

\begin{figure}[!htb]
    \begin{center}
        \includegraphics[width=0.6\textwidth]{figures/search_hh/signal_pheno/shape_plots/hh_shape_plot_mt2_bb}
        \caption{
            Normalized distributions showing the shape of the \mtbb observable for the SM
            top-quark processes (\ttbar~and single-top $Wt$) and the dilepton $hh \rightarrow \bbww$ signal process.
        }
        \label{fig:hh_shape_mtbb}
    \end{center}
\end{figure}
