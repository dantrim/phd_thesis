%\chapter{Common Elements in the Analysis of High Energy Physics Collision Data}
\chapter{Common Elements in the Search for New Physics}
\label{chap:common_search}

\epigraph{
\textit{Our `Age of Anxiety' is, in great part, the result of trying to do today's jobs with yesterday's tools!}
}{--Marshall McLuhan}
%Our Age of Anxiety is, in great part, the result of trying to do today’s jobs with yesterday’s tools! -- Marshall McLuhan
%Diaper spelled backwards is repaid , think about it. -- Marshall McLuhan

%%%%%%%%%%%%%%%%%%%%%%%%%%%%%%%%%%%%%%%%%%%%%%%%%%%%%%%%%%%%%%%%%%%%%%
%%%%%%%%%%%%%%%%%%%%%%%%%%%%%%%%%%%%%%%%%%%%%%%%%%%%%%%%%%%%%%%%%%%%%%

There are many ways in which physicists study and analyse the data collected by
the ATLAS detector.
Once the gathered data has been aggregated and the physics objects therein
have been reconstructed (Chapter~\ref{chap:objects}), an analysist has at their fingertips access
to the stuff of high energy particle physics from which they
can test their hypotheses or perform measurements of fundamental quantities.
In the present work, several analyses representing the search for BSM physics
will be discussed in detail.
The details of each of these analyses differ quite a bit, but each follow
a general \textit{analysis strategy} that will be discussed
further in Section~\ref{sec:gen_strategy}.
Physics analyses start with a stated goal in mind; for example,
a measurement of some quantity to a desired level of precision
or, as in the analyses to be presented, a statement about the existence of a new particle or theory of new physics.
The methods by which these statements are to be made are intertwined with, and generally
dictate the initial design of, the overall
analysis strategy and are statistical in nature.
In Section~\ref{sec:stat_hypo} a discussion of the statistical methods used
in the analyses to be presented will be given, highlighting how statements about
new physics are made and how broad-strokes physics conclusions can be drawn from them.
Section~\ref{sec:fakes} follows up on the topic of how one estimates particular sources
of SM backgrounds for which the MC simulation cannot be relied upon to give reasonable
predictions, particularly on those processes that lead to leptons being produced in the detector
that do not originate from the $pp$ hard-scatter processes of interest.

%%%%%%%%%%%%%%%%%%%%%%%%%%%%%%%%%%%%%%%%%%%%%%%%%%%%%%%%%%%%%%%%%%%%%%
%%%%%%%%%%%%%%%%%%%%%%%%%%%%%%%%%%%%%%%%%%%%%%%%%%%%%%%%%%%%%%%%%%%%%%
%
% GENERAL STRATEGY
%
%%%%%%%%%%%%%%%%%%%%%%%%%%%%%%%%%%%%%%%%%%%%%%%%%%%%%%%%%%%%%%%%%%%%%%
%%%%%%%%%%%%%%%%%%%%%%%%%%%%%%%%%%%%%%%%%%%%%%%%%%%%%%%%%%%%%%%%%%%%%%
\section{General Analysis Strategy}
\label{sec:gen_strategy}

In this section the analysis strategy used in the searches for new physics
to be presented in Chapters~\ref{chap:search_stop} and \ref{chap:search_hh}
will be given.
The general analysis workflow for designing an analysis is outlined in the following
sub-sections.

%%%%%%%%%%%%%%%%%%%%%%%%%%%%%%%%%%%%%%%%%%%%%%%%%%%%%%%%%%%%%%%%%%%%%%%%%%%%%
% SIGNAL PHENO
%%%%%%%%%%%%%%%%%%%%%%%%%%%%%%%%%%%%%%%%%%%%%%%%%%%%%%%%%%%%%%%%%%%%%%%%%%%%%
\subsection{Target the Signal}
\label{sec:sig_pheno}

The search for a particular source of new physics, such as a particular model of SUSY (Chapter~\ref{chap:bsm}),
begins first with the thorough understanding of the signatures that the new physics model
will leave in the ATLAS detector.
This generally requires a strict definition of the \textit{final state} of the
new physics model that one wishes to look for; for example, deciding to search for
evidence of SUSY via the production of the SUSY partners to the SM top-quark
in final states having exactly two leptons (electrons or muons) instead of
exactly zero or exactly one lepton, as in Chapter~\ref{chap:search_stop}.\footnote{Performing
searches for new physics by the partitioning of specific new physics models
by their resulting final states allows for separate, independent dedicated analyses to be carried out
for each possible final state with the idea that each one will be more sensitive
to the presence of the new physics in their respective final state than would be
a single analysis attempting to target all possible final states of the new physics production.
The results of the independent analyses' searches can be statistically combined once they are finished,
leading to enhanced sensitivities to the new physics scenario in question that is more or less independent
of the final state.}
Once a new physics model has been chosen, along with its final state, there is a well-defined
\textit{signal} to be looked for in the data recorded by the ATLAS detector.
The production and decay of the sought-for signal is then simulated via MC methods in the exact
same manner as for the SM processes, as described in Chapter~\ref{chap:simulation}.
In physics analyses, the physics processes not inclusive of the sought-for signal processes
are referred to as the \textit{background} processes.

The simulation of the signal process allows one to study the kinematics of the signal in detail, in order
to get an overall feel for what phase space the signal inhabits.
Knowledge of both the signal final state and its kinematics therein informs the analyst
about the specific SM background processes that are likely to be relevant to the analysis.
For example, if the sought-for signal decays to two leptons with opposite electric charge
that are of the same flavor (both leptons are electrons or both are muons, for example)
it is very likely that the SM processes inclusive of $Z$-boson production will be relevant,
since this is one of the main $Z$-boson decay final states, as opposed to the production of a single $W$-boson
whose decay does not lead directly to final states with two leptons.
Knowledge of the dominant SM background processes, then, allows
one to determine how the phenomenology and kinematics of the signal differ
with respect to those of the relevant backgrounds by comparing the simulated events
of each.
The aim of this is to be able to define a basis of kinematic observables that allows
for the discrimination between the signal and background.
From such a basis of observables, one can define regions of phase space in which
the signal-to-background ratio is large, such that the likelihood of observing
the presence of the signal is (ideally) maximal.
Such regions of increased signal purity\footnote{The `purity' of a process is defined
as the fraction of a given process in a region of phase space, relative to the sum
of all processes (inclusive of the process in question).} are referred
to as \textit{signal regions} (SR).
As an example, take the case where there is a single discriminating variable in our
basis of useful kinematic observables.
One would apply a selection on this observable in such a way that $pp$ collision events
satisfying this selection are likely to be enhanced in signal events.
This is one-diminsional SR case is illustrated in Figure~\ref{fig:sr_search_v}.

\begin{figure}[!htb]
    \begin{center}
        \includegraphics[width=0.65\textwidth]{figures/common_ana/sr_search_vPDF}
        \caption{
            Signal region concept illustrated in the case of a one-dimensional selection
            made on a discriminating kinematic observable.
            The dominant SM background (red) is characterised by typically small values
            of the discriminating variable whereas the signal (blue) has values that extend
            beyond that of the background.
            The signal region in this case is defined by requiring $pp$ collision events
            to have values of the discriminating variable that are larger than
            the value indicated by the dashed vertical line, where the signal purity is
            enhanced.
            The $y$-axis represents the probability distribution of the background and signal
            processes, not their absolute yield for a given range of the quantity on the $x$-axis.
        }
        \label{fig:sr_search_v}
    \end{center}
\end{figure}



%%%%%%%%%%%%%%%%%%%%%%%%%%%%%%%%%%%%%%%%%%%%%%%%%%%%%%%%%%%%%%%%%%%%%%%%%%%%%
% GATHER THE DATA
%%%%%%%%%%%%%%%%%%%%%%%%%%%%%%%%%%%%%%%%%%%%%%%%%%%%%%%%%%%%%%%%%%%%%%%%%%%%%
\FloatBarrier
\subsection{Define the Trigger Strategy}
\label{sec:gather_data}

Once an analysis understands the final state and its kinematics that it will be searching for,
the strategy for gathering the $pp$ collision data consistent with that final state must be defined.
This requires that the analysis make a choice about which triggers to use for recording
the collision events in ATLAS, a process referred to as defining the analysis' \textit{trigger strategy}.
The analyses to be presented in Chapters~\ref{chap:search_stop} and \ref{chap:search_hh} are based
on searches for signals with leptons (electrons or muons) in the final state.
As a result, the triggers these analyses use are all based on signatures of high-\pT~leptons.

The lepton triggers used in the analyses to be presented trigger are based on the \pT~of the leptons.
They are configured to have a given \pT~threshold and if a lepton is identified in the online trigger
system during the $pp$ collisions to have a \pT~value at or above this threshold, the trigger `fires' and the
event is recorded (see Section~\ref{sec:tdaq}).
Lepton triggers are characterised by a sharp `turn-on', meaning that they become fully efficient
very near the online threshold at which they operate. 
Here, `efficiency' is defined as the ability for the trigger to make a decision to record the event
when there is actually an object satisfying its requirements. A trigger is 100\% efficiency if it
has a threshold of 10\,GeV and it fires for every single lepton with a \pT~at or above 10\,GeV.
The maximum attainable trigger efficiency is never exactly 100\% due to non-100\% coverage of the
trigger system, however.
Examples of the trigger efficiencies, measured in data and in MC, are shown in Figure~\ref{fig:trig_example} which shows
the trigger efficiency turn-on curves for representative electron and muon triggers.
It can be seen, for example, that the electron trigger efficiency reaches a plateau very near
its configured threshold of 28\,GeV.
As described in Section~\ref{sec:tdaq}, the trigger system has two-levels: Level 1 (L1) and HLT.
The efficiencies for the trigger at L1 and HLT differ, primarily due to the lower lepton momentum resolution
achieved at L1.
This can be seen in the right side of Figure~\ref{fig:trig_example}, which shows the trigger
efficiency turn-on also for the L1 trigger that seeds the HLT trigger.
The rise in efficiency, with respect to the offline reconstructed object's \pT, is shallower
at L1 than at the HLT as a result of the poorer muon momentum resolution at L1.

As can be seen in the left side of Figure~\ref{fig:trig_example}, the MC description of the trigger
response is fairly good.
However, analyses typically require that their offline reconstructed objects on which they base
their trigger strategy have \pT~values that lie on the trigger efficiency plateau.
This is because the trigger efficiency plateau represents a region in which the trigger response
is stable (i.e. unchanging) in both data and MC and can therefore be easily calibrated in the offline analysis
with scale-factors that account for differences in the measured trigger efficiencies in data and in MC simulation.
The description in the MC of the region at the trigger threshold, which has non-vertical rise,
is potentially difficult to model accurately and thefore the calibrations are not typically
derived for these regions in which the efficiency is not at its plateau.
This is illustrated in Figure~\ref{fig:trig_plateau_cartoon}, showing the typical case of a stable
and flat response in both the observed data and MC in the region of the trigger efficiency
plateau but a potentially difficult to characterise response during the turn-on phase of the trigger
efficiency.
Given, however, the relatively good momentum resolution for leptons at L1, the offline \pT~requirements on leptons
can be kept very near the online \pT~thresholds, typically within 1\,GeV or so.
Triggers based on jets or on the missing transverse momentum, however, typically require offline \pT~requirements
that are significantly higher than the online thresholds due to the fact that the online reconstruction of these
objects in the trigger is not at the level of precision attainable in the offline reconstruction.
This is illustrated by the trigger efficiency curves shown in Figure~\ref{fig:met_trig_example}, showing
the efficiency curves for triggers based on the reconstruction of the missing transverse momentum.
These \met-based triggers all have online thresholds near $80\,\GeV$, but they do not reach their
efficiency plateau until offline \met values nearing 250\,GeV.

\begin{figure}[!htb]
    \begin{center}
        \includegraphics[width=0.48\textwidth]{figures/common_ana/trig/egam_trig_example} 
        \includegraphics[width=0.48\textwidth]{figures/common_ana/trig/muon_trig_example} 
        \caption{
            Figures showing the measured trigger efficiency as a function of the associated
            offline object for a representative electron trigger (\textbf{\textit{left}})
            and muon trigger (\textbf{\textit{right}}).
        }
        \label{fig:trig_example}
    \end{center}
\end{figure}


\begin{figure}[!htb]
    \begin{center}
        \includegraphics[width=0.75\textwidth]{figures/common_ana/trig_plateauPDF}
        \caption{
            Cartoon illustrating the principle of a trigger efficiency `turn on' curve.
            The efficiency for the lepton to fire the trigger, as a function of the \pT~of the offline object, is plotted
            as a function of the offline object's \pT.
            At the online level, given the typically poorer lepton momentum resolution, there is generally
            not a perfectly sharp (i.e. vertical) turn on at the \pT~threshold of the trigger.
            Instead there is an `S'-curve, with the efficiency increasing with a steep slope
            until it reaches a point where it flattens out.
            This latter point is referred to as the trigger efficiency `plateau'.
            Offline analyses typically apply offline \pT~requirements on their objects
            such that they are always on the plateau of the associated trigger used for event selection.
        }
        \label{fig:trig_plateau_cartoon}
    \end{center}
\end{figure}

\begin{figure}[!htb]
    \begin{center}
        \includegraphics[width=0.6\textwidth]{figures/common_ana/trig/met_trig_example}
        \caption{
            Trigger efficiency turn-on curve for typical triggers based on the missing transverse momentum.
        }
        \label{fig:met_trig_example}
    \end{center}
\end{figure}

%%%%%%%%%%%%%%%%%%%%%%%%%%%%%%%%%%%%%%%%%%%%%%%%%%%%%%%%%%%%%%%%%%%%%%%%%%%%%
% THE CONTROL REGION METHOD
%%%%%%%%%%%%%%%%%%%%%%%%%%%%%%%%%%%%%%%%%%%%%%%%%%%%%%%%%%%%%%%%%%%%%%%%%%%%%
\FloatBarrier
\subsection{Background Estimation and the Control Region Method}
\label{sec:control_region_method}

The general principle behind searches for new physics is to define a SR, or a set of SRs,
and then make predictions about how the signal and background behave therein.
Such predictions can then be compared to the data actually recorded by the ATLAS detector
and the statistical procedures described in Section~\ref{sec:stat_hypo} can be used to
make statements about whether or not --- or to what degree --- the data is likely to contain the specified signal.
The emphasis, then, in physics analyses is on the understanding and precise estimation of the backgrounds.
Without being able to properly estimate the contribution of the background processes to the
events observed in the SRs, well-defined predictions cannot therein be made, resulting in ineffective
analyses.

The process of estimating the backgrounds in an analysis' SRs is aptly referred to as
\textit{background estimation}.
There are many background estimation methods that are used.
There exist general background estimation techniques, applicable to a wide range of SM processes,
as well as more dedicated estimation techniques that are specific to a smaller subset of
SM processes.
Most rely on the MC simulation of the SM processes, either as the primary source of providing
the prediction of a given SM process in an analysis' SR(s) or secondarily, as a means of providing a
cross-check on or input to a prediction obtained using the observed data obtained in auxiliary measurements as the primary source.
The high levels of accuracy imposed upon the ATLAS MC simulation infrastructure is derived
from the large and dominant role that the MC simulation plays in the background estimation
procedures in almost all analyses performed by the ATLAS experiment.

The same background estimation strategy is used in each of the analyses to be presented in the
present thesis and is as follows.
Once the SRs designed to capture the sought-for signal are defined, MC simulation can be
used to determine the overall contribution of all SM background processes.
In such a way, the MC simulation can be used to understand which SM processes in the SR(s)
are dominant and which are sub-dominant.
The former are those whose relative contribution to the total background prediction
are large and the latter are those for which this quantity is small or negligible.

As illustrated in Figure~\ref{fig:sr_search_v} for the one-dimensional example,
SR(s) are typically defined by events populating the \textit{tails} of discriminating
observables or probe extreme regions of phase space.
%The tails of such observables are regions of low cross-section and
Such regions typically exhibit low cross-section (background rates) and
are regions of phase space for which the underlying theoretical inputs
to the MC may be less well-understood as compared to the bulk of the phase space.
The MC simulation by itself, therefore, may not be able to adequately describe the background
processes in the SRs defined in these regions, failing to describe
either the overall cross-section of specific processes or the actual shape of the discriminating observables' distributions
therein.
The former results in a failure in the overall predicted \textit{rate}, or normalisation, of a specific backgrounds' contribution
to the event yields in the regions
and the latter results in a failure in predicting the background process' \textit{acceptance}.
%Indeed, the tails of many observables are highly sensitive to theoretical parameters used as
%inputs to the MC simulation process which leads the background estimation being sensitive to these
%parameters.
In order to increase the confidence in the background estimates in such SR(s), the analyses in the present
thesis make use of the so-called \textit{control region method}.
This method is characterised by defining a (set of) region(s) in which there is (are) high purities
of the dominant background process(es).
These regions are referred to as \textit{control regions} (CRs) and are ideally defined using the
same basis of observables used to define the analysis' SRs.
The CRs are defined to be orthogonal to the SRs, meaning that no events that satisfy the requirements of
the SRs populate the CRs.
The observed data in the CRs, which are enriched in a specific background process, are used to derive
factors that correct the cross-section predictions of the MC estimates of the dominant background processes for
which the CRs are defined.
The per-process normalisation corrections, $\mu_p$, can essentially be thought of as those factors
that adjust the process' normalisation in such a way as to cover any discrepancy between the observed
data yield and MC prediction for the process in question:
\begin{align}
    \mu_{p} = \frac{  N_{\text{data}}^{\text{CR}} - \sum\limits_{\substack{i \\ i\ne p}} N_{\text{MC},\,i}^{\text{CR}}} { N_{\text{MC},\,p}^{\text{CR}}},
    \label{eq:mu_fac}
    %\mu = \frac{ \left( N_{\text{obs}}^{\text{CR}} - \sum\limits_{i_{i\ne \text{proc}}} \right)} {4}
\end{align}
where `$p$' indicates the process for which the CR is defined, $N_{\text{data}}^{\text{CR}}$ is the observed
data yield in the CR, and $N_{\text{MC},\,j}^{\text{CR}}$ is the predicted yield in MC for the background
process $j$.
If there is more than one process for which a normalisation correction factor is being derived, and therefore
more than one CR, the normalisation factors are constrained by the process' contribution across all CRs in which
it is present and the expression in Equation~\ref{eq:mu_fac} is expanded into a system of equations,
\begin{align}
    N_{\text{data,sub}}^{\text{CR1}} &= \mu_i N_i^{\text{CR1}} + \mu_j N_j^{\text{CR1}} + ... \nonumber \\
    N_{\text{data,sub}}^{\text{CR2}} &= \mu_i N_i^{\text{CR2}} + \mu_j N_j^{\text{CR2}} + ...     \label{eq:mu_fac_expand} \\
        &\vdots \nonumber
\end{align}
where $N_{\text{data,sub}}^{a}$ is the observed data yield in the region $a$ with the MC predictions
for those processes not having a dedicated  CR subtracted (analogous to the numerator appearing in Equation~\ref{eq:mu_fac}),
$\mu_p$ are the normalisation factors for each process being solved for, and $N_p^{a}$ are the MC predictions
for process $p$ in region $a$.
Each process' dedicated CR ideally exhibits both a high purity of the given process and a relatively large
number of events\footnote{A too large difference in the numbers of events observed in the CRs, as compared to the SRs, may indicate
that the CRs are kinematically very different from the SRs, however.} and therefore
is the only CR that has any real constraining power on the process' normalisation correction factor.
If this is true, the expression in Equation~\ref{eq:mu_fac} generally holds true for each process with a
CR, even in the case of multiple CRs and
normalisation factors.

As mentioned above, the CRs are ideally defined using the same basis of kinematic observables
as used in the definition of the SR.
When this is the case, it is more likely that the constructed CRs probe a similar kinematic
phase space as that of the SRs.
It is important that the CRs are kinematically similar to the SRs so that the correction factors
derived in them are representative of the SRs; that is, that the underlying root cause of the need for the
correction is the same in both the CRs, where the corrections are derived, and the SRs, where
the corrections are applied.
If an SR requires high numbers of jets (high event activity), for example, but the CR is defined to have zero
jets (low event activity) then any normalisation correction derived in the CR may be correcting for physics effects
that are not relevant to the phase space probed by the SR.
In such a case, extrapolation uncertainties will generally be incurred in the final background estimate in the SR.

In addition to the CRs, so-called \textit{validation regions} (VRs) are typically defined.
The VRs are typically kinematically more similar to the SRs than the CRs, while still maintaining orthogonality
between the CRs and the SRs.
VRs are defined for each CR and allow for one to validate the extrapolation of the CR-derived normalisation
correction for each process in a region more similar to the SR.
As they are kinematically closer to the SRs, VRs are generally less pure in the specific process for which they are defined,
and will also have generally fewer events, as compared to the associated CRs.
The validation is done by comparing the post-corrected MC prediction of the backgrounds to the
observed data in the VRs, ensuring that both the overall normalisation of the backgrounds
agrees with the observed data as well as the overall shape of the relevant observables used
in the definition of the SRs.
The relationship between the CR, VR, and SR is illustrated in the one-dimensional case in
Figure~\ref{fig:sr_search_v_CR}.

When constructing a set of CRs and VRs, it is important to do so using the right set of observables
out of the total basis of observables from which the SRs are defined.
It is important that the shape as predicted by the simulated background process for which
the normalisation correction is being derived reproduces that of the observed data.
If this is not the case, then the extrapolation from the CR to the SR suffers.
This is illustrated in Figure~\ref{fig:crvr_extrap_shape} for two scenarios in which
the MC-based background prediction of the shape of the observable used for defining
the various regions both agrees and does not agree with the observed data.
%in the regions used to derive and validate the normalisation correction.
If the MC simulation for the specific background for which a normalisation correction is being derived
has monotonic shape mis-modellings, as in the case of Scenario B in Figure~\ref{fig:crvr_extrap_shape},
the normalisation correction will generally not be applicable in the SR and may lead to
SR background estimates with false over- or under-predictions of the data.
In the latter case, a false excess in data may be observed and lead to mis-statements about the
likelihood of the existence of new physics in the SRs.
In the former case, a false over-prediction will lead to too-prematurely excluding the possibility for new
physics to arise when it may in fact exist.
For this reason, dedicated studies on the dependence of the derived normalisation corrections
on the set of observables used to define the CRs and VRs should generally be made so
that the analysis avoids these susceptibilities to shape mis-modellings in the MC.
Of course, this can become challenging in the general case where the SRs are defined using a large basis of potentially
correlated observables
and/or when one wishes to define several CRs to correct multiple processes' normalisations.

\begin{figure}[!htb]
    \begin{center}
        \includegraphics[width=0.65\textwidth]{figures/common_ana/sr_search_v_CRPDF}
        \caption{
            Illustration of the control region method, in the one-dimensional case analogous to that
            presented in Figure~\ref{fig:sr_search_v}.
            The control region (CR) is pure in the background process but is defined kinematically alongside the signal region (SR).
            A validation region (VR), ideally still with high background purity, is defined between the CR and SR and is used to validate
            the extrapolation of the background estimate from the CR to the SR.
            The $y$-axis represents the probability distribution of the background and signal
            processes, not their absolute yield for a given range of the quantity on the $x$-axis.
        }
        \label{fig:sr_search_v_CR}
    \end{center}
\end{figure}

\begin{figure}[!htb]
    \begin{center}
        \includegraphics[width=0.8\textwidth]{figures/common_ana/crvr_extrap_shape}
        \caption{
            Illustration of CR extrapolation scenarios in the control region method.
            In Scenario A (green data) the predicted shape of the discriminating variable
            used to define the CR, VR, and SR agrees well with the observed data
            in both the CR and VR, as seen by the flat data-to-background ratio in the bottom.
            In Scenario B (red data) the predicted shape of the discriminanting variable
            differs with respect to that of the observed data, leading to an observed
            slope in the data-to-background ratio.
            In Scenario A, the normalisation correction derived for the background process
            in the CR will be well extrapolated into the VR and gives confidence in its applicability
            in the SR.
            In Scenario B, the CR-derived normalisation factor for the background process
            will pull the background prediction in the wrong direction when extrapolated
            to the VR, making the data-to-background agreement even worse and reducing
            its applicability in the SR.
        }
        \label{fig:crvr_extrap_shape}
    \end{center}
\end{figure}

\FloatBarrier
%%%%%%%%%%%%%%%%%%%%%%%%%%%%%%%%%%%%%%%%%%%%%%%%%%%%%%%%%%%%%%%%%%%%%%%%%%%%%
% SYSTEMATIC UNCERTAINTIES
%%%%%%%%%%%%%%%%%%%%%%%%%%%%%%%%%%%%%%%%%%%%%%%%%%%%%%%%%%%%%%%%%%%%%%%%%%%%%


%%%%%%%%%%%%%%%%%%%%%%%%%%%%%%%%%%%%%%%%%%%%%%%%%%%%%%%%%%%%%%%%%%%%%%
%%%%%%%%%%%%%%%%%%%%%%%%%%%%%%%%%%%%%%%%%%%%%%%%%%%%%%%%%%%%%%%%%%%%%%
%
% STATISTICS AND HYPOTHESIS TESTING
%
%%%%%%%%%%%%%%%%%%%%%%%%%%%%%%%%%%%%%%%%%%%%%%%%%%%%%%%%%%%%%%%%%%%%%%
%%%%%%%%%%%%%%%%%%%%%%%%%%%%%%%%%%%%%%%%%%%%%%%%%%%%%%%%%%%%%%%%%%%%%%
\section{Hypothesis Testing and Statistics}
\label{sec:stat_hypo}

This section describes the statistical procedures used in the analyses to be
presented in Chapters~\ref{chap:seach_stop} and \ref{chap:search_hh} that allow
for conclusions to be drawn about the compatibility of the observed data with theories of
BSM physics.
The statistical inference tools described are inherently Frequentist and
are, for the most part, the \textit{de facto} standard for physics analyses searching for evidence of BSM physics
at the large experiments at the LHC.
Their widespread adoption by the experiments at the LHC does not indicate the
philosophical merit of Frequentist inference methodology, but rather highlights the technically simpler implementation
of Frequentist hypothesis testing that allows for physics analyses to not get
bogged down in some of the details associated with Bayesian analyses, computational
or otherwise.
Indeed, most people by default think and interact with the world around them
in a Bayesian manner.
Taking the path of least resistance, physicists have tended to opt for the simpler
implementation of reporting their results, which, at the end of the day,
tend to not lose out much in terms of the picture of the objective truth that they draw~\cite{CousinsBayes}.
Section~\ref{sec:hypo_test} will describe, in somewhat general terms, what
hypotheses tests are and the way in which they are performed in ATLAS.
Section~\ref{sec:likelihood} describes the details by which the measurements and systematic
uncertainties of
an analysis are transcribed into the language of the hypothesis test described
in Section~\ref{sec:hypotest} using a likelihood-based test statistic.


%%%%%%%%%%%%%%%%%%%%%%%%%%%%%%%%%%%%%%%%%%%%%%%%%%%%%%%%%%%%%%%%%%%
%%%%%%%%%%%%%%%%%%%%%%%%%%%%%%%%%%%%%%%%%%%%%%%%%%%%%%%%%%%%%%%%%%%
%
% HYPOTHESIS TESTING
%
%%%%%%%%%%%%%%%%%%%%%%%%%%%%%%%%%%%%%%%%%%%%%%%%%%%%%%%%%%%%%%%%%%%
%%%%%%%%%%%%%%%%%%%%%%%%%%%%%%%%%%%%%%%%%%%%%%%%%%%%%%%%%%%%%%%%%%%

\subsection{Hypothesis Testing and the \cls Construction}
\label{sec:hypo_test}

Hypothesis testing starts with the unambiguous formulation of the hypothesis being
tested.
In the search for evidence of BSM physics, there are two hypothesis pitted
against one another.
The first is the \textit{null hypothesis}, denoted $H_0$, which is the hypothesis
subject to the test and corresponds to the SM hypothesis. The null hypothesis is commonly
referred to simply as the background-only (B) hypothesis.
The second hypothesis is the \textit{alternative hypothesis}, denoted $H_1$, and corresponds
to the SM with the addition of the BSM physics process being sought out.
The hypothesis $H_1$ is commonly referred to as the signal-plus-background (S+B) hypothesis.
In both searches presented in Chapters~\ref{chap:search_stop} and \ref{chap:search_hh},
$H_0$ is taken to be the SM.
In the search presented in Chapter~\ref{chap:search_stop}, $H_1$ is taken to be a specific
instantiation of the MSSM ({\color{red}{Section XXX}}), with specific masses of the
stop quark and LSP.
In the search presented in Chapter~\ref{chap:search_hh}, $H_1$ is taken to be the
non-resonant production of Higgs boson pairs.
In this latter case, the $H_1$ hypothesis is indeed a process predicted by the SM ({\color{red}{SECTION XXX about HH pheno and EWSB}}) but
it is one that is not included in the $H_0$ hypothesis.

%%%%%%%%%%%%%%%%%%%%%%%%%%%%%%%%%%%%%%%%%%%%%%%%%%%%%%%%%%%%%%%%%%%
%%%%%%%%%%%%%%%%%%%%%%%%%%%%%%%%%%%%%%%%%%%%%%%%%%%%%%%%%%%%%%%%%%%
%
% TEST STATISTICS
%
%%%%%%%%%%%%%%%%%%%%%%%%%%%%%%%%%%%%%%%%%%%%%%%%%%%%%%%%%%%%%%%%%%%
%%%%%%%%%%%%%%%%%%%%%%%%%%%%%%%%%%%%%%%%%%%%%%%%%%%%%%%%%%%%%%%%%%%
\subsubsection{The Test Statistic and $p$-Values}

In order to perform a hypothesis test in the Frequentist arena, a \textit{test statistic}, $q(x)$,
is defined.
A test statistic is defined using the analysis' measurements alone and is
used in order to define metrics by which the observed data is said to agree with
one of the two hypothesis, either $H_0$ or $H_1$.
In Section~\ref{sec:likelihood}, the exact form of the likelihood used in modern LHC experiments,
and that used in the analyses discussed in Chapters~\ref{chap:search_stop} and \ref{chap:search_hh},
will be introduced.
Here we will discuss general features of Frequentist test statistics and introduce
some of the language that will be used later on when discussing the results of the
analyses.

The conclusions eventually drawn about a given hypothesis are baesd on the observed value
of $q(x)$ and where this value lies in relation to the pre-defined \textit{critical region}.
The critical region is defined by a cut value, $q_c$, on the distribution of $q(x)$ under a specified hypothesis.
In the one-sided tests to be considered in the present thesis, $H_1$
will tend to have larger values of $q(x)$ as compared to the $H_0$.
The critical region defines two important parameters associated with the hypothesis test.
The first is the quantity $\alpha$, which is referred to as the \textit{significance level},
and is defined as follows,
\begin{align}
    \int\limits_{q_c}^{+\infty} \, f(q | H_0) \, \mathrm{d}q = \alpha,
    \label{eq:sig_level}
\end{align}
where $f(q|H_0)$ is the probability distribution for the test statistic under
the background-only hypothesis.
The quantity $\alpha$ reports the probability for the background-only hypothesis (the SM) to be rejected when it is actually
true. This is commonly referred to as the Type I error rate.
The second quantity is $\beta$ and is defined as,
\begin{align}
    \int\limits_{-\infty}^{q_c} \, f(q|H_1) \, \mathrm{d}q = \beta,
    \label{eq:power_level}
\end{align}
where $f(q|H_1)$ is the probability distribution for the test statistic under
the signal-plus-background hypothesis.
The quantity $\beta$ gives the probability to reject the signal-plus-background hypothesis
when it is actually true. This is commonly referred to as the Type II error rate.
The quantity $(1-\beta)$ is referred to as the \textit{power of the test}.
The better a given physics analysis is at being able to discriminate between the signal
and background, i.e. to have clear separation between the $H_0$ and $H_1$ hypotheses,
the smaller (larger) is $\beta$ (the power of the test).

For simplicity, the two hypotheses $H_0$ and $H_1$ can be generalised by introducing a so-called
`signal strength' parameter, $\mu$, which acts as a multiplicative factor on the signal cross-section
appearing in $H_1$.
The hypothesis $H_0$, then, corresponds to the case $\mu = 0$ and that of $H_1$ corresponds to
$\mu = 1$.
With this general notation, then, the test statistic under either hypothesis is labelled as $q_{\mu}$.

Once a test statistic is specified, and its expected distribution under a given hypothesis is obtained,
$p$-values can be defined in order to compute the probability that the observed data originates from the
considered hypothesis (value of $\mu$).
They are computed as follows,
\begin{align}
    p_{\mu} = \int\limits_{q_{\mu, \text{obs}}}^{+\infty} \, f(q_{\mu} | \mu) \, \mathrm{d}q_{\mu},
    \label{eq:test_stat_pvalue}
\end{align}
where $q_{\mu, \text{obs}}$ is the observed value of the test statistic in data and $f(q_{\mu} | \mu)$ is the probability
density function of $q_{\mu}$ assuming hypothesis $\mu$.
A particular case of Equation~\ref{eq:test_stat_pvalue} is that of $p_0$, which quantifies the agreement of the data with the background-only
hypothesis ($\mu = 0$).
The $p_{\mu}$-value associated with a given hypothesis ($\mu$-value) is typically converted into the equivalent corresponding Gaussian significance, $Z$, defined
as the number of standard deviations that correspond to an upper-tail probability of $p_{\mu}$.
This is illustrated in Figure~\ref{fig:pval_sig}.

As the value of $p_{\mu}$ gets smaller, the confidence that the assumed hypothesis (value of $\mu$) is true
decreases.
At a certain point, it becomes acceptable to say that the assumed hypothesis is incompatible with
reality and the hypothesis described by the particular value of $\mu$ is said to be \textit{excluded}.
In the particle physics community, the conventional threshold to take for the value of $p_{\mu}$
at which point a hypothesis is said to be excluded is $p_{\mu} = 0.05$, corresponding to $Z=1.64$ as
illustrated in Figure~\ref{fig:pval_sig}.
This choice of the $p_{\mu}$-value at which point exclusion is said to occur defines
the critical region, described above, of the test.
The value of $0.05$ corresponds to the significance level of the test (c.f. Equation~\ref{eq:sig_level}), and is referred
to a hypothesis test being performed at the $95\%$ confidence level (CL) (i.e. CL $\equiv (1-\alpha)$)

In order to claim that new physics has been seen, the null hypothesis ($\mu = 0$) must be rejected.
The thresholds at which new physics can be said to have been observed and discovered are
much more stringent than that used for the exclusion of a specified hypothesis.
Incompatibilities with the null-hypothesis at the level of $p_0 = 1.3 \times 10^{-3}$ and
$p_0 = 2.9\times 10^{-7}$ are required in order to state that observation and discovery, respectively,
of new phenomena has occurred.
These thresholds, illustrated in Figure~\ref{fig:pval_sig}, for claiming observation and discovery are the fabled `$3\sigma$' and `$5\sigma$'
$p_0$-value criterion adopted by the particle physics community.

\begin{figure}[!htb]
    \begin{center}
        \includegraphics[width=0.48\textwidth]{figures/common_ana/stat_hypo/pval_sig_lin}
        \includegraphics[width=0.48\textwidth]{figures/common_ana/stat_hypo/pval_sig_log}
        \caption{
            Gaussian tail significance levels corresponding to $Z = 1.64\sigma$, $3\sigma$, and $5\sigma$
            significances, corresponding to $p_0$-values of $0.05$, $1.3 \times 10^{-3}$, and $2.9\times 10^{-7}$, respectively.
            \textbf{\textit{Left}}: Linear $y$-scale. \textbf{\textit{Right}}: Logarithmic $y$-scale.
        }
        \label{fig:pval_sig}
    \end{center}
\end{figure}

%The process of performing a hypothesis test, then, is specified by the following procedure:
%\begin{enumerate}
%    \item Unambiguously define the background-only (the SM prediction) and signal-plus-background (the prediction of the SM with BSM physics added)
%            hypotheses, $H_0$ and $H_1$.
%    \item Define an appropriate test statistic, $t(x)$ (see Section~\ref{sec:likelihood}).
%    \item Construct the distribution of the test statistic under the background-only hypothesis, $g(t|H_0)$.
%    \item Define the desired significance level of the hypothesis, $\alpha$
%    \item Obtain the value of $t(x)$ as observed in data
%    \item If the observed value of $t(x)$ is within the critical region defined by $\alpha$ as in Equation~\ref{eq:sig_level} ($t_{\text{obs}} > t_c$), reject $H_0$.
%            Otherwise, $H_0$ cannot be rejected.
%\end{enumerate}


%%%%%%%%%%%%%%%%%%%%%%%%%%%%%%%%%%%%%%%%%%%%%%%%%%%%%%%%%%%%%%%%%%%
%%%%%%%%%%%%%%%%%%%%%%%%%%%%%%%%%%%%%%%%%%%%%%%%%%%%%%%%%%%%%%%%%%%
%
% THE CLS METHOD
%
%%%%%%%%%%%%%%%%%%%%%%%%%%%%%%%%%%%%%%%%%%%%%%%%%%%%%%%%%%%%%%%%%%%
%%%%%%%%%%%%%%%%%%%%%%%%%%%%%%%%%%%%%%%%%%%%%%%%%%%%%%%%%%%%%%%%%%%

\subsubsection{The \cls Construction}
\label{sec:cls_method}

In searches for new physics, the statement that a given signal hypothesis has been excluded
is an important one.
Once made by the LHC experiments, the specific signal model is essentially considered
no longer important to be searched for.
Therefore, the metrics by which the experiments make claims of exclusion have surrounding them
a wide-ranging literature discussing the merits and drawbacks of the many such metrics
that have been proposed over the years.
The bare $p_{\mu}$-value, for example, extracted from the observed data is subject to statistical fluctuations
and it can lead to unphysical exclusions when a downward fluctuation in the observed
number of events occurs.
This would lead to a premature exclusion of perhaps a broad region of new physics
that would perhaps no longer be looked into by future analyses or experiments.

The standard metric used by the LHC experiments today is known as `\cls'~\cite{CLSReadI,CLSReadII},
and is constructed in such a way as to minimize the likelihood of excluding signal
hypotheses that a search is not a-priori sensitive to.
The \cls metric is given by,
\begin{align}
    \text{CL}_s = \frac{p_{\mu}}{1-p_0},
    \label{eq:cls_def}
\end{align}
where the quantities $p_{\mu}$ and $p_0$ quantify the compatibilities between the data and the signal-plus-background
and background-only hypotheses, respectively.
Downward fluctuations in data, as those described above, will lead to larger values of $p_0$; thus
leading to larger values of \cls that avoid premature exclusion.

At the LHC, the \cls metric is used primarily for performing hypothesis tests aimed at claiming exclusion.
The standard null-hypothesis $p_0$-value is still used for claiming observation and discovery, as described above.
A given signal hypothesis with $\mu = 1$ is considered excluded when $\cls \le 0.05$.
Note that this prescription for exclusion, $\cls \le \alpha$, is generally a stronger requirement than the
standard prescription, $p_{\mu} \le \alpha$.
The \cls metric is also used to compute \textit{upper limits}.
An upper limit on a given signal hypothesis specified by $\mu$
is the largest value of $\mu$ satisfying $\cls \ge 0.05$.
The interpretation being that this corresponds to the largest possible signal cross-section
that is unable to be excluded and therefore smaller values of $\mu$, corresponding
to smaller signal cross-sections, are still consistent with the observed data and cannot therefore
be excluded.
The process of scanning $\mu$ hypotheses and computing the \cls in order to find
an upper limit on $\mu$ is illustrated in Figure~\ref{fig:upper_limit_scan_cartoon}.
%An upper limit on a given signal hypothesis specified by $\mu$ is
%the value of $\mu$ at which $\cls = 0.05$.
%An illustration of how an upper limit is obtained is provided by Figure~\ref{fig:upper_limit_scan_cartoon}.
%%the largest value of $\mu$ describing the signal process that satisfies $\cls \le 0.05$.

\begin{figure}[!htb]
    \begin{center}
        \includegraphics[width=0.5\textwidth]{figures/common_ana/stat_hypo/upper_limit_scan_examplePDF}
        \caption{
            An upper limit scan on the signal strength parameter $\mu$ associated with a signal hypothesis.
            The \cls, given by Equation~\ref{eq:cls_def}, is recomputed for a range of $\mu$ values
            describing a given signal hypothesis.
            This is shown by the blue line.
            The $\mu$ value at which the \cls curve crosses the line $\cls = 0.05$, $\mu^{\text{UL}}$, is the
            upper limit on $\mu$ for the signal hypothesis.
            Values of $\mu$ smaller than $\mu^{\text{UL}}$ remain compatible with the observed data,
            while those values greater than $\mu^{\text{UL}}$ are excluded at $95\%$ CL.
        }
        \label{fig:upper_limit_scan_cartoon}
    \end{center}
\end{figure}

%%%%%%%%%%%%%%%%%%%%%%%%%%%%%%%%%%%%%%%%%%%%%%%%%%%%%%%%%%%%%%%%%%%
%%%%%%%%%%%%%%%%%%%%%%%%%%%%%%%%%%%%%%%%%%%%%%%%%%%%%%%%%%%%%%%%%%%
%
% PROFILE LIKELIHOOD
%
%%%%%%%%%%%%%%%%%%%%%%%%%%%%%%%%%%%%%%%%%%%%%%%%%%%%%%%%%%%%%%%%%%%
%%%%%%%%%%%%%%%%%%%%%%%%%%%%%%%%%%%%%%%%%%%%%%%%%%%%%%%%%%%%%%%%%%%

\subsection{Likelihood Analysis}
\label{sec:likelihood}



%%%%%%%%%%%%%%%%%%%%%%%%%%%%%%%%%%%%%%%%%%%%%%%%%%%%%%%%%%%%%%%%%%%%%%
%%%%%%%%%%%%%%%%%%%%%%%%%%%%%%%%%%%%%%%%%%%%%%%%%%%%%%%%%%%%%%%%%%%%%%
%
% FAKES
%
%%%%%%%%%%%%%%%%%%%%%%%%%%%%%%%%%%%%%%%%%%%%%%%%%%%%%%%%%%%%%%%%%%%%%%
%%%%%%%%%%%%%%%%%%%%%%%%%%%%%%%%%%%%%%%%%%%%%%%%%%%%%%%%%%%%%%%%%%%%%%
\section{Estimation of Sources of Fake and Non-prompt Leptons}
\label{sec:fakes}

Despite both the high levels of accuracy achieved by the ATLAS simulation
infrastructure and the lepton reconstruction and identification algorithms
described in Chapter~\ref{chap:objects}, sources of misidentified reconstructed
leptons still exist and lead to an additional source of backgrounds to
the analyses discussed in Chapters~\ref{chap:search_stop} and \ref{chap:search_hh}.
These background sources of leptons are broken down into two categories:
\begin{itemize}
    \item \textbf{Fake leptons}: Cases in which signals in the ATLAS detector
        are selected as being leptons when in fact there is no real lepton present
    \item \textbf{Non-prompt leptons}: When real, genuine leptons are identified
        but they are not leptons originating from the primary $pp$ hard-scatter interaction process
        of interest
\end{itemize}
In the subsequent discussion, the term `fake' will be used in reference to the two
categories listed above, unless specified otherwise.

The contribution of backgrounds leading to sources of fake leptons are generally predicted
using methods based on the observed data --- referred to as `data-driven' methods ---
and arise from various sources and mechanisms.
The sources of fake leptons will be described in Section~\ref{sec:fake_lepton_sources}, separately for
electrons and muons.
Section~\ref{sec:fake_dd_motivation} provides some reasoning for why a data-driven
approach is generally taken for estimating these backgrounds.
Sections~\ref{sec:matrix_method} and \ref{sec:same_sign_extrap} go on to describe
the two data-driven approaches taken in the analyses to be presented in this thesis
for estimating fake lepton contributions: the so-called `Matrix Method' and the `Same-sign Extrapolation Method', respectively.

%%%%%%%%%%%%%%%%%%%%%%%%%%%%%%%%%%%%%%%%%%%%%%%%%%%%%%%%%%%%%%%%%%%
%%%%%%%%%%%%%%%%%%%%%%%%%%%%%%%%%%%%%%%%%%%%%%%%%%%%%%%%%%%%%%%%%%%
%
% SOURCES OF FAKE LEPTONS
%
%%%%%%%%%%%%%%%%%%%%%%%%%%%%%%%%%%%%%%%%%%%%%%%%%%%%%%%%%%%%%%%%%%%
%%%%%%%%%%%%%%%%%%%%%%%%%%%%%%%%%%%%%%%%%%%%%%%%%%%%%%%%%%%%%%%%%%%
\subsection{Sources of Fake Leptons}
\label{sec:fake_lepton_sources}

The types and sources of fake leptons generally have different experimental signatures
than those leptons that genuinely originate from the $pp$ hard-scatter.
However, due to the non-perfect lepton identification and isolation algorithms,
such sources are able to contaminate the various regions of an analysis.
The rates of contamination are generally quite low for the analyses to be presented, but their inclusion in the background
estimates of the analyses has measurable consequences nevertheless.

The analyses to be presented in the current thesis make use of $b$-tagging algorithms
to identify jets originating from $b$-hadrons.
Fake leptons, both electrons and muons, can originate from the semi-leptonic decays of
$b$- and $c$-quarks within these $b$-tagged jets, following $b\rightarrow \ell$ or cascade-type
$b \rightarrow c \rightarrow \ell$ decays of the $B$ hadrons within the jets.
The leptons resulting from such decays are typically embedded within or very close to
the originating reconstructed jet object and the lepton isolation requirements
are intended to reduce this type of background.
The subsequent paragraphs will describe additional sources of fake electrons
and muons, which generally differ between the two lepton species.

%%%%%%%%%%%%%%%%%%%%%%%%%%%%%%%%%%%%%%%%%%%%%%%%%%%%%%%%%%%%%%%%%%%
%%%%%%%%%%%%%%%%%%%%%%%%%%%%%%%%%%%%%%%%%%%%%%%%%%%%%%%%%%%%%%%%%%%
%
% SOURCES OF FAKE ELECTRONS
%
%%%%%%%%%%%%%%%%%%%%%%%%%%%%%%%%%%%%%%%%%%%%%%%%%%%%%%%%%%%%%%%%%%%
%%%%%%%%%%%%%%%%%%%%%%%%%%%%%%%%%%%%%%%%%%%%%%%%%%%%%%%%%%%%%%%%%%%
\subsubsection{Sources of Fake Electrons}
\label{sec:fake_electron_sources}

As described in Section~\ref{sec:electrons}, electrons are reconstructed based
on the presence of well-reconstructed tracks in the ID matched to deposited
energy clusters in the EM calorimeter.
Light-flavor jets, originating from the production of light quarks ($u$, $d$, $s$),
or gluon jets, which are associated with a large number of tracks due to their
increased radiation pattern, are able to fake electrons as they leave
tracks in the ID as well as subsequent energy depositions in both the EM and hadronic calorimeters.
This background, due to mis-identified jets, is typically suppressed by the use
of lepton isolation and by jet shower-shape information used in the electron identification: the
hadronic shower shapes and radial extent differ with respect to the electromagnetic shower
produced by a genuine electron.

An additional large source of fake electrons is due to photon conversion processes,
$\gamma \rightarrow e^+ e^-$, and other electromagnetic scattering processes
that happen as a result of detector material interactions.
These processes leave both tracks in the ID and electromagnetic energy depositions
in the EM calorimeter which are difficult to distinguish from genuine electrons.
Neutral hadron decays, such as the $\pi^0 \rightarrow e^+ e^- \gamma$ Dalitz decay,
also lead to electron-like signatures.
This decay of the $\pi^0$ only has a branching fraction of just over $1\%$~\cite{PDGRef}, but given
the large production of $\pi^0$ states in the $pp$ collision this has the potential to be
a relevant source of fake electrons.
These electromagnetic sources of fake electrons are distinguished by their generally
larger impact parameters relative to genuine prompt electrons.

\begin{figure}[!htb]
    \begin{center}
        \includegraphics[width=0.45\textwidth]{figures/common_ana/fakes/electron_brem_fake}
        \caption{
        }
        \label{fig:electron_brem_fake}
    \end{center}
\end{figure}


%%%%%%%%%%%%%%%%%%%%%%%%%%%%%%%%%%%%%%%%%%%%%%%%%%%%%%%%%%%%%%%%%%%
%%%%%%%%%%%%%%%%%%%%%%%%%%%%%%%%%%%%%%%%%%%%%%%%%%%%%%%%%%%%%%%%%%%
%
% SOURCES OF FAKE MUONS
%
%%%%%%%%%%%%%%%%%%%%%%%%%%%%%%%%%%%%%%%%%%%%%%%%%%%%%%%%%%%%%%%%%%%
%%%%%%%%%%%%%%%%%%%%%%%%%%%%%%%%%%%%%%%%%%%%%%%%%%%%%%%%%%%%%%%%%%%
\subsubsection{Sources of Fake Muons}
\label{sec:fake_muon_sources}

As described in Section~\ref{sec:muons}, muons are primarily reconstructed via the combination
of tracking information provided by the ID and MS, and, generally speaking, they should be the only particle species to reach
the MS.
In addition to the semi-leptonic decays of heavy-flavored jets described above, however,
there are several sources of fake muons.
Highly energetic jets can have elongated shower profiles that reach the outer
radii of the hadronic calorimeter, with a non-zero chance of exiting the calorimeter
and resulting in particle leakage into the MS.
Such cases are referred to as calorimeter punch-through, and have been illustrated
in Figure~\ref{fig:jet_punch_through}.
Punch-through particles can leave signatures similar to charged muons whose subsequent
MS tracks are associated with a track in the ID, leading to a reconstructed combined muon
faking a genuine muon.
An additional source of fake muons come from the in-flight decays of charged hadrons,
such as the $K^\pm$ and $\pi^{\pm}$, that can decay to $\mu^{\pm} \nu$.
These sources of non-prompt muons typically result in low-\pT~muons and are characterised
by combined tracks with exhibiting a kink topology,
briefly mentioned when discussing the muon combined reconstruction in Section~\ref{sec:muon_id}.
The \pT~dependence of muon cross-sections for various sources of muons (prompt, fake, and non-prompt)
is shown in Figure~\ref{fig:fake_muon_kink}, along with an illustration of the kinked-track topology
characterising the in-flight decays of Kaons and charged pions.

\begin{figure}[!htb]
    \begin{center}
        \includegraphics[width=0.43\textwidth]{figures/common_ana/fakes/reco_muon_sources}
        \raisebox{0.45cm}{\includegraphics[width=0.54\textwidth]{figures/common_ana/fake_muon_kinkPDF}}
        \caption{
            \textbf{\textit{Left}}:  Transverse momentum dependence of muon cross-sections for muons originating
                from various prompt and non-prompt sources.
                Figure taken from Ref.~\cite{CERN-LHCC-97-022}.
            \textbf{\textit{Right}}: Illustration of a reconstructed non-prompt muon resulting from a kinked-track topology.
                A produced $K^{\pm}$ or $\pi^{\pm}$ is produced and decays in-flight to a muon and muon-neutrino.
                The point at which the hadron decays is indicated by the yellow dot.
                The red circles indicate detector hits in the ID and MS layers indicated
                by the horizontal black lines.
        }
        \label{fig:fake_muon_kink}
    \end{center}
\end{figure}


%%%%%%%%%%%%%%%%%%%%%%%%%%%%%%%%%%%%%%%%%%%%%%%%%%%%%%%%%%%%%%%%%%%
%%%%%%%%%%%%%%%%%%%%%%%%%%%%%%%%%%%%%%%%%%%%%%%%%%%%%%%%%%%%%%%%%%%
%
% DATA DRIVEN MOTIVATION
%
%%%%%%%%%%%%%%%%%%%%%%%%%%%%%%%%%%%%%%%%%%%%%%%%%%%%%%%%%%%%%%%%%%%
%%%%%%%%%%%%%%%%%%%%%%%%%%%%%%%%%%%%%%%%%%%%%%%%%%%%%%%%%%%%%%%%%%%
\subsection{The Need for a Data-driven Approach}
\label{sec:fake_dd_motivation}

In the analyses to be presented in Chapters~\ref{chap:search_stop} and \ref{chap:search_hh},
relatively tight identification working points are used for electrons and muons.
As a result, the contamination of fake leptons in these analyses is relatively
minor.
Although small, their contamination does have measurable effects and so their contribution
must be accounted for in order to achieve accurate estimates of the backgrounds
in each analysis.

Several methods exist to estimate the background rates arising from the sources of fake
leptons, those relying on data-driven methods or those based entirely on the MC
simulation.
Relying on the MC simulation of these sources of fake leptons, described in previous
sections, means to rely entirely on the \textsc{GEANT4} simulation of the ATLAS detector
and on the MC generation and showering processes to accurately predict the
rates of these processes.
There are several problems with this approach and they are (nonexhaustively) as follows.
Given the very small region of phase space being probed by the analysis,
the number of MC events needed to appropriately sample the sources of production of fake
leptons as described above would be prohibitively large if a statistically relevant sample
is desired.
An accurate prediction of the production rates of several of these fake lepton sources
would require an accurate underlying theoretical model of many processes, such as
heavy-flavor jet fragmentation, which is challenging.
Additionally, many sources of fake leptons arise as a result of detector material interactions
or as a result of subtle and difficult-to-model failure modes of the detector response.
%or inaccurate simulation of the detector response.
The accurate prediction of the rate of electron bremsstrahlung and photon conversions, for example, requires
high levels of precision in the simulation and measurement of the active and passive material in the ATLAS detector and cavern,
which is not necessarily possible.
The rates of jets being mis-identified as electrons and jet punch-through, for example,
require that the MC simulation of the calorimeter response and shower evolution are
accurately modelled.
The MC simulation is not expected to perform to the degree at which these subtle, and comparatively rare, effects
are accurately predicted.
For this reason, data-driven approaches are typically taken for estimating the background rates
of these fake lepton sources.
In the analyses to be presented, two data-driven approaches are taken.
In the search described in Chapter~\ref{chap:search_stop}, the Matrix Method
is used.
In the search described in Chapter~\ref{chap:search_hh}, the Same-sign Extrapolation
Method is used.
These methods are introduced in Sections~\ref{sec:matrix_method} and \ref{sec:same_sign_extrap}, respectively.


\FloatBarrier
%%%%%%%%%%%%%%%%%%%%%%%%%%%%%%%%%%%%%%%%%%%%%%%%%%%%%%%%%%%%%%%%%%%
%%%%%%%%%%%%%%%%%%%%%%%%%%%%%%%%%%%%%%%%%%%%%%%%%%%%%%%%%%%%%%%%%%%
%
% THE MATRIX METHOD
%
%%%%%%%%%%%%%%%%%%%%%%%%%%%%%%%%%%%%%%%%%%%%%%%%%%%%%%%%%%%%%%%%%%%
%%%%%%%%%%%%%%%%%%%%%%%%%%%%%%%%%%%%%%%%%%%%%%%%%%%%%%%%%%%%%%%%%%%
\subsection{The Matrix Method}
\label{sec:matrix_method}

The Matrix Method, discussed thoroughly in Ref.~\cite{TOPFake}, is one of the most common
methods used in ATLAS analyses for estimating backgrounds due to processes containing
fake leptons.
It is characterised by the definition of two levels of lepton selection:
\begin{itemize}
    \item[]\textbf{Tight Leptons}: Those leptons passing all reconstruction, identification, and isolation criteria
        as the leptons used in the final analysis' results
    \item[]\textbf{Loose Leptons}: Leptons requiring similar selections as the Tight leptons but typically with either, or both, identification
        and isolation criteria relaxed
\end{itemize}
The Tight leptons are a subset of the Loose, by definition.
In the analysis described in Chapter~\ref{chap:search_stop}, the Loose leptons are defined by loosening
only the lepton identification working points.
Generally speaking, both samples of Loose and Tight leptons will contain
both fake and real leptons.\footnote{Genuine, prompt
leptons originating from the $pp$ hard-scatter interaction point are typically referred to as `real' leptons
in order to distinguish them, semantically, from fake and non-prompt leptons.
}
The Matrix Method consists of measuring a set of efficiencies: the \textit{real}
(\textit{fake}) \textit{efficiencies}, $\varepsilon_r$ ($\varepsilon_f$),
defined as the efficiency for a real (fake) electron or muon that satisfies the Loose selection criteria
to also satisfy the Tight selection criteria.
This is illustrated in Figure~\ref{fig:fake_effs}.
As illustrated, both the Loose and Tight lepton samples will contain contributions of both fake
and real leptons.
The Matrix Method can be generalised to final states with any number of leptons.
In the discussion to follow, we will discuss that of final states with two leptons: the dilepton Matrix Method.

\begin{figure}[!htb]
    \begin{center}
        \includegraphics[width=0.65\textwidth]{figures/common_ana/fakes/fake_effs_illustration}
        \caption{
        }
        \label{fig:fake_effs}
    \end{center}
\end{figure}

The real efficiencies, $\varepsilon_r$, are measured in data using $Z$-boson tag-and-probe methods, requiring the probe lepton
to satisfy the Tight lepton selection and to be matched to the trigger.
The probe lepton must satisfy the Loose lepton selection.
The fraction of Loose probe leptons to pass the Tight lepton selection then gives a measure of $\varepsilon_r$.

The fake efficiencies, $\varepsilon_f$, are measured in data using events with different-flavor leptons,
where one lepton is an electron and the other is a muon, that have the same electric charge.
A tag-and-probe method similar to that used in the measurement of $\varepsilon_r$ is used
and relies on the fact that a comparatively small amount of SM processes can result in same-sign
and different-flavor events.
Therefore, when a probe satisfies the Tight selection it is very likely to be the case that the
probe lepton is a fake lepton.
An additional component of the $\varepsilon_f$ is measured by additionally requesting that there
be at least one $b$-tagged jet in the same-sign, different-flavor selection.
This allows for $\varepsilon_f$ to be measured in a region enriched in fake leptons originating
from semi-leptonic decays of heavy-flavor jets.
The various measurements of $\varepsilon_f$ are combined following an averaging scheme, weighted according
to the composition of fake lepton sources expected to contaminate the SRs.

The real and fake efficiencies are not global quantities.
Instead, they are measured as a function of both the lepton $\pT$ and $\eta$
such that they may track the effects of changes in detector response and
material interaction over the $\pT$ and $\eta$ ranges relevant to the leptons used in the analysis.

Once $\varepsilon_r$ and $\varepsilon_f$ are obtained, the fake lepton background
can be obtained by inverting the equation relating the measured quantities (Tight versus Loose), taken
from the observed data, in terms
of those that we want to know (Fake versus Real):
\begin{equation}
    \begin{pmatrix}
        N_{TT} \\ N_{TL} \\ N_{LT} \\ N_{LL}
    \end{pmatrix}
        = M
    \begin{pmatrix}
        N_{LL}^{RR} \\ N_{LL}^{RF} \\ N_{LL}^{FR} \\ N_{LL}^{FF}
    \end{pmatrix}
    \label{eq:matrix_method}
\end{equation}
\noindent where in the sub- and super-scripts, the first (second) index refers to the leading (sub-leading) lepton.
The sub-script `$T$' (`$L$') refers to the lepton passing the Tight (Loose) lepton
selection.
The super-script `$R$' (`$F$') indicates whether or not the lepton is a real (fake) lepton.
For example, the quantity $N_{LL}^{FR}$ is the number of events in which the leading lepton
in the sample of Loose leptons is fake and the sub-leading is real.
The matrix $M$ is given by,
\begin{align}
    M = \begin{pmatrix}
            \varepsilon_{r,1}\,\varepsilon_{r,2} & \varepsilon_{r,1}\,\varepsilon_{f,2}  & \varepsilon_{f,1}\, \varepsilon_{r,2} & \varepsilon_{f,1}\, \varepsilon_{f,2} \\
            \varepsilon_{r,1}\, \overline{\varepsilon_{r,2}} & \varepsilon_{r,1}\,\overline{\varepsilon_{f,2}} & \varepsilon_{f,1}\, \overline{\varepsilon_{r.2}} & \varepsilon_{f,1}\, \overline{\varepsilon_{f,2}} \\
            \overline{\varepsilon_{r,1}} \varepsilon_{r,2} & \overline{\varepsilon_{r,1}}\, \varepsilon_{f,2} & \overline{\varepsilon_{f,1}}\, \varepsilon_{r,2} & \overline{\varepsilon_{f,1}}\, \varepsilon_{f,2} \\
            \overline{\varepsilon_{r,1}}\, \overline{\varepsilon_{r,2}} & \overline{\varepsilon_{r,1}}\, \overline{\varepsilon_{f,2}} & \overline{\varepsilon_{f,1}}\, \overline{\varepsilon_{r,2}} & \overline{\varepsilon_{f,1}}\, \overline{\varepsilon_{f,2}}
        \end{pmatrix}
    \label{eq:matrix_method_matrix}
\end{align}
where the notation $\overline{\varepsilon}$ indicates $(1 - \varepsilon)$.
%To make clear the data-driven aspect of the method:
%both sets of qauntities --- those appearing on the left-hand-side of Equation~\ref{eq:matrix_method} and the $r_i$ and $f_i$ ---
%are measured using the observed data.

The number of events with double-fake and single-fake leptons satisfying the analysis' Tight selection
($N_{TT}^{FF}$ and $N_{TT}^{RF} + N_{TT}^{FR}$, respectively) can then be obtained from the
number of events with double-fake and single-fake leptons satisfying the Loose selection
($N_{LL}^{FF}$ and $N_{LL}^{RF} + N_{LL}^{FR}$, respectively) through inversion
of Equation~\ref{eq:matrix_method}  and
noting the following relations:
\begin{align}
    N_{TT}^{RR} &= \varepsilon_{r,1}\,\varepsilon_{r,2} \times N_{LL}^{RR}  \label{eq:matrix_method_sol0}\\
    N_{TT}^{RF} &= \varepsilon_{r,1}\,\varepsilon_{f,2} \times N_{LL}^{RF}  \label{eq:matrix_method_sol1}\\
    N_{TT}^{FR} &= \varepsilon_{f,1}\,\varepsilon_{r,2} \times N_{LL}^{FR}  \label{eq:matrix_method_sol2}\\
    N_{TT}^{FF} &= \varepsilon_{f,1}\,\varepsilon_{f,2} \times N_{LL}^{FF}. \label{eq:matrix_method_sol3}
\end{align}
The quantities appearing on the right-hand-side of Equations~\ref{eq:matrix_method_sol0}-\ref{eq:matrix_method_sol3}
depend entirely on the observed data.
The number of events in the analysis' Tight selection that have at least one fake lepton is then given by the sum
$N_{TT}^{RF} + N_{TT}^{FR} + N_{TT}^{FF}$.
This gives a total integrated yield for the fake background contribution in the analysis'
Tight selection; however, from Equations~\ref{eq:matrix_method_sol1}-\ref{eq:matrix_method_sol3}
a set of per-event weighting factors (`fake weights'), depending only on the $\varepsilon_r(\pT,\eta)$ and
$\varepsilon_f(\pT,\eta)$ efficiency factors for the two leptons, can be defined.
These fake weights allow for kinematic distributions of the fake lepton background
sources to be populated
by appropriately applying them to events in the data sample consisting of leptons satisfying the analysis' Loose selection.



%%%%%%%%%%%%%%%%%%%%%%%%%%%%%%%%%%%%%%%%%%%%%%%%%%%%%%%%%%%%%%%%%%%
%%%%%%%%%%%%%%%%%%%%%%%%%%%%%%%%%%%%%%%%%%%%%%%%%%%%%%%%%%%%%%%%%%%
%
% SAME SIGN EXTRAPOLATION
%
%%%%%%%%%%%%%%%%%%%%%%%%%%%%%%%%%%%%%%%%%%%%%%%%%%%%%%%%%%%%%%%%%%%
%%%%%%%%%%%%%%%%%%%%%%%%%%%%%%%%%%%%%%%%%%%%%%%%%%%%%%%%%%%%%%%%%%%
\subsection{Same-sign Extrapolation Method}
\label{sec:same_sign_extrap}

The Same-sign Extrapolation Method is another method for
estimating the numbers of events with fake and non-prompt leptons.
This method is well described in Refs.~\cite{TOPQ-2015-09,TOPQ-2017-05}.
The method is tailored for dilepton analyses that require the two
leptons to have opposite electric charge and takes advantage of the fact that the
production mechanisms for the fake leptons described above are generally
uncorrelated with the charges of leptons in the event.
This means that the rates of the various sources of fake leptons will generally
be the same in the sample of oppositely-charged dilepton events and in the
sample of dilepton events in which the leptons have the same charge.
The former are referred to as opposite-sign (OS) events and the latter
as same-sign (SS) events.
This assumption is not entirely correct, however, since the underlying sources of
the fake and/or non-prompt leptons are not all charge symmetric.\footnote{`Charge-symmetric' means that the
dilepton final state of a given process may be either OS or SS, with equal probability of each case occuring.}
For example, a dominant source of mis-identified leptons in dileptonic events
arises from the production of semi-leptonically decaying top-quark pairs, in which the sub-leading reconstructed
electrons arise from a mis-identified jet.
This process is charge-symmetric since
converted photons produced in jets are equally likely to give rise to a reconstructed
$e^+$ or $e^-$ and are uncorrelated to the sign of the real lepton.
In trident decays whereby bremsstrahlung photons (radiated from the lepton children
of the top-quarks) undergo conversion processes, however, there is a partial
charge correlation of the reconstructed (non-prompt) electron with the charge of the original lepton,
and hence conversion processes contribute more to the OS sample of events.
These two cases are illustrated in Figure~\ref{fig:ttbar_fake_charge_sym}.
The lepton definitions within the OS and SS event samples are the same.
That is, they use the same reconstruction, identification, and isolation working points.
This is compared to the Matrix Method (Section~\ref{sec:matrix_method}), in which the two samples of leptons used in the technique's
implementation differ by their lepton definitions.

\begin{figure}[!htb]
    \begin{center}
        \includegraphics[width=0.75\textwidth]{figures/common_ana/fakes/ttbar_fake_charge_sym_semi}
        \includegraphics[width=0.7\textwidth]{figures/common_ana/fakes/ttbar_fake_charge_sym_trident}
        \caption{
            Illustration of charge-asymmetry in fake lepton production
            arising in top-quark pair production events, leading to
            the rate of photon conversion sources of fake electrons being larger in OS dilepton events.
            \textbf{\textit{Top}}: Shower photons arising from decays within one of the jets
                in semi-leptonic decays of top-quark pairs may convert to an electron-positron pair,
                leading to one of the jets being reconstructed as an electron.
                Looking at the charge possibilities of each side of the top-quark pair decay,
                the event is equally likely to be classified as either an OS or SS event.
            \textbf{\textit{Bottom}}: Trident events arising in dileptonic top-quark pair production
                events can lead to a non-prompt electron from the photon conversion being selected
                as one of the event's candidate leptons.
                The overall charge possibilites of the lepton charges on the side of the trident decay
                are correlated with the charge of the initial lepton.
                Looking at the charge combinations possible between both sides of the top-quark pair decay,
                the event is more likely to be classified as an OS event.
        }
        \label{fig:ttbar_fake_charge_sym}
    \end{center}
\end{figure}

The method also assumes that the rate of dilepton events in which \textit{both} leptons
are fake is negligible, and that the composition of the fake background contributing to
the dilepton final state is composed of events in which one of the leptons is real.
In the majority of cases, the sub-leading lepton is the fake lepton.
For this reason, when using the MC simulation to gain predictions of the composition of
the fake backgrounds, the MC events in which only one of the leptons is fake are considered.
This will become clear in the discussion to follow and in the specific implementation described
in Chapter~\ref{chap:search_hh}.


The general method works as follows.
Since, as described above, the rates of the various {\color{red}{just put "sources" not "rates"}} sources of fake backgrounds in the OS and SS samples of events
are generally the same, the method relies on using the sample of SS events to provide a template of the fake backgrounds
to be used in the OS selections.
The contribution of fakes to each of the regions (CR, VR, or SR) in the analysis is estimated
by subtracting the prediction of the prompt (real) SM backgrounds from the observed data
in the associated SS selections, defined similarly to the OS selections used in the analysis but with the
charge requirements inverted.
The ratio of the number of OS to SS events with fake leptons, $f^{SS \rightarrow OS}$,
is taken entirely from MC and is applied to the SS data that has had the prompt-MC contribution
subtracted in order to extrapolate this number to the OS regions.
This SS extrapolation is described by Equation~\ref{eq:ss_extrap}:
\begin{align}
    N_{\text{OS}}^{\text{fake}} &= f^{SS \rightarrow OS} \times N_{\text{SS}}^{\text{fake}} \nonumber \\
        &= \frac{ N_{\text{MC,OS}}^{\text{fake}} }{ N_{\text{MC,SS}}^{\text{fake}} } \times ( N_{\text{data,SS}} - N_{\text{MC,SS}}^{\text{real}} )
        \label{eq:ss_extrap}
\end{align}
In addition to providing the overall yields of the fake backgrounds in a given region,
the method described by Equation~\ref{eq:ss_extrap} provides the means to
inspect the kinematics of the predicted fake backgrounds by simply computing
Equation~\ref{eq:ss_extrap} on a bin-by-bin basis when populating histograms
of kinematic observables.

Further details on the implementation of the Same-sign Extrapolation Method,
described by Equation~\ref{eq:ss_extrap}, will be given in Chapter~\ref{chap:search_hh}.
In particular, the implicit sensitivities to the MC simulation will be described,
as well as additional extrapolations specific to the analysis.

