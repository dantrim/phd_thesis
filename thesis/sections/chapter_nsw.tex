\chapter{The Phase-I New Small Wheel Upgrade Project}
\label{chap:nsw}

During its first two periods of operation --- Runs 1 and 2 --- the LHC operated
at or exceeding its design goals.
ATLAS observed a peak of more than 60 $pp$ interactions per bunch crossing with nominal running conditions for physics data
and instantaneous luminosities exceeding $2\times 10^{34}$ cm$^{-2}$\,s$^{-1}$.
These instantaneous luminosities levels were those originally envisaged as the end goal \textit{after} LS2,
during which the Phase 1 Upgrades are planned to take place both for the LHC and the large experiments.
During LS2, the LHC plans to upgrade its injector chain to allow for delivering ever
more increased peak instantaneous luminosities to the experiments.
During LS3, the LHC will have its entire magnet system overhauled and upgraded to become the
High Luminosity LHC (HL-LHC) allowing for subsequent runs to have increased bunch densities 
to the experiments, achieving peak luminosities far exceeding those observed thus far.

In the following sections, the large scale Phase 1 Upgrade to the ATLAS detector's forward muon system
will be introduced.
The upgraded is a replacement of the current Small Wheel section of the MS endcap, located at $z \approx 7$\,m (Figure~\ref{fig:muon_segmentation})
and is the largest such upgrade targetting HL-LHC data taking conditions to take place so far.
This upgrade is referred to as the New Small Wheel (NSW) upgrade and is planned for completion by
the end of LS2.
In Section~\ref{sec:nsw_motivation} the reasons necessitating an upgrade to the forward muon system
of ATLAS will be presented, with emphasis on how the current system will not be able to maintain high
levels of performance in the Run 3 and HL-LHC era.
Section~\ref{sec:nsw_layout} will then proceed to describe the detector technologies that make up
the NSW, and how they will meet the foreseen challenges of Run 3 and beyond, as well as their layout
in ATLAS.
Sections~\ref{sec:nsw_elx}-\ref{sec:nsw_verso}, then, will introduces the aspects of the NSW that the
current author predominantly worked on; namely, the development of the software interface to the NSW front-end readout
electronics and detectors necessary for both validation and integration of the NSW readout infrastructure.


%During LS2, the LHC plans to upgrade its injector chain to allow for delivering ever more increased instantaneous luminosities.
%during which the LHC plans to upgrade its injector chain to allow for delivering increased
%luminosities to the experiments.
%During LS3, the LHC will be upgraded to the High Luminosity LHC (HL-LHC) (Figure~\ref{fig:lhc_schedule}), upgrading entirely
%its magnet systems, to provide increased bunch densities at the high luminosity LHC points (Point 1 and 5)
%so as to reach peak luminosities far exceeding those already observed.
%In the following sections the need for an upgrade to the forward muon system in ATLAS, in view of
%the foreseen running conditions to be delieved by the LHC.
%The upgrade, planned to take place during the presently on-going LS2 period of the LHC, is
%the replacement of the Small Wheel sector of the ATLAS MS and is the largest detector upgrade
%currently taking place amongst the large LHC experiments.
%The challenges that ATLAS foresees with the current system and how the Phase 1 upgrade (c.f. Figure~\ref{fig:lhc_schedule}), the so-called
%New Small Wheel (NSW) upgrade, will address them will be discussed in Section~\ref{sec:nsw_motivation},
%followed by Section~\ref{sec:nsw_layout} describing the NSW detector layout.
%Section~\ref{sec:nsw_elx}-\ref{sec:nsw_verso} will introduce the aspects of the NSW that the current
%author predominantly worked on; namely, the development of the software interface to the NSW front-end readout
%electronics and detectors necessary for both validation and integration of the NSW readout infrastructure.

\section{The Need for an Upgraded Forward Muon System}

\section{The New Small Wheel Detector}
\subsection{Geometry and Layout}
\subsection{MicroMegas}
\subsection{sTGC}
\section{Readout Electronics}
\subsection{The VMM}
\subsection{Front-end Boards}

\section{Configuration, Data-acquisition, and Calibration Software for the NSW Front-end Electronics}
\subsection{VERSO}
\subsection{Calibration Algorithm Development}
\subsubsection{Gain}
\subsubsection{ADC Calibration}
\subsubsection{Noise Measurements}
\subsubsection{Timing Calibration}
\subsubsection{Per-channel Threshold Equilisation}
\subsection{Use Cases}
\subsubsection{Test Benches and Labs}
\subsubsection{High Rate Tests and Test Beams}

\section{The Upgrade of the ATLAS T/DAQ Infrastructure}
\subsection{FELIX}
\subsection{The Software ROD}
