This thesis reports on two searches for physics beyond the Standard Model (SM), performed using data collected from
the $\sqrt{s}=13$\,TeV proton-proton ($pp$) collisions recorded
by the ATLAS detector at CERN between the years 2015--2018, the period of LHC Run 2.
Both searches are performed in final states requiring exactly two leptons, where a lepton
is considered to be either an electron or muon, and enriched in $b$-tagged jets.
The first is a search for a relatively light supersymmetric partner to the SM
top-quark, referred to as the `stop quark', under the assumption of $R$-parity conserving Supersymmetry.
No statistically significant discrepancies are observed in the analysis' regions.
As a result, exclusion limits are set on the allowed mass range of the stop quark.
The second is a search for evidence of Higgs bosons ($h$) being produced in pairs, a process
predicted in the SM and that is sensitive to the Higgs
self-coupling parameter.
Given the small predicted cross-section for this process, the analysis searches for enhanced production
of Higgs pairs, to a level observable already in the Run 2 data and above that predicted by the SM alone.
The analysis is performed in the dilepton final state of the $hh \rightarrow bbWW$ channel, and is the first time that
this search has been performed by the ATLAS experiment.
No evidence for Higgs boson pairs are observed in the Run 2 data and
upper limits are therefore set on the cross-section of the SM Higgs boson pair production process.
In addition to these two physics analyses, this thesis discusses the upgrade of the forward
muon system of the ATLAS detector, referred to as the `New Small Wheel' (NSW) Upgrade,
that aims for an installation in the ATLAS cavern beginning in 2020.
Emphasis is given to discussion of the frontend readout electronics, both on their readout and calibration.
