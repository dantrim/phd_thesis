% All fonts, including those for sub- and superscripts, must be 10
% points or larger.  Recommended sizes are 14-point for chapter
% headings, 12-point for the main body of text and figure/table
% titles, and 10-point for footnotes, sub- and super-scripts, and text
% in figures and tables.
%
% Notes: Add short title to figures, sections, via square brackets,
% e.g. \section[short]{long}.
%
%\documentclass[12pt,fleqn]{style/ucithesis}
\documentclass[11pt]{style/ucithesis}

% A few common packages
\usepackage{amsmath}
\usepackage{amsthm}
\usepackage{array}
\usepackage{graphicx}
\usepackage{relsize}
\usepackage{geometry}

% Some other useful packages
\usepackage{caption}
\usepackage{subcaption}  % \begin{subfigure}...\end{subfigure} within figure
\usepackage{multirow}
\usepackage{tabularx}

% plainpages=false fixes the "duplicate ignored" error with page counters
% Set pdfborder to 0 0 0 to disable colored borders around PDF hyperlinks
\usepackage[plainpages=false,pdfborder={0 0 0}]{hyperref}
\usepackage{epigraph}
\setlength{\epigraphwidth}{0.8\textwidth}
\setlength\epigraphrule{0pt}

\usepackage[sorting=none,hyperref,backend=biber,backref,backrefstyle=none]{biblatex}
\bibliography{bib/references.bib,
                bib/chapter2.bib,
                bib/chapter3.bib
}

\usepackage{tabularx}
\usepackage{booktabs}

\usepackage{array}
\usepackage{makecell}
\usepackage{graphicx}
\usepackage{arydshln}
\usepackage{stackengine}
\usepackage{xcolor}
\usepackage{amsmath}
\usepackage{placeins}
\usepackage{mathtools}


%\usepackage[sorting=none]{biblatex}
%\addbibresource{bib/references.bib}

% Uncomment the following two lines to use the algorithm package,
% which provides an algorithm environment similar to figure and table
% ("\begin{algorithm}...\end{algorithm}"). A list of algorithms will
% automatically be added in the preliminary pages. Note that you
% probably want a package for the actual code to go with this (e.g.,
% algorithmic).
%\usepackage{algorithm}
%\renewcommand{\listalgorithmname}{\protect\centering\protect\Large LIST OF ALGORITHMS}

% Uncomment the following line to enable Unicode support. This will allow you
% to enter non-ASCII characters (such as accented characters) directly without
% having to use LaTeX's awkward escape syntax (e.g., \'{e})
% NOTE: You may have to install the ucs.sty package for this to work. See:
% http://www.unruh.de/DniQ/latex/unicode/
%\usepackage[utf8x]{inputenc}

% Uncomment the following to avoid "widowing", where page breaks cause
% single lines of paragraphs to float onto the next page (this is not
% a UCI requirement but more of an aesthetic choice).
%\widowpenalty=10000
%\clubpenalty=10000

% Modify or extend these at will.
\newtheorem{theorem}{\textsc{Theorem}}[chapter]
\newtheorem{definition}{\textsc{Definition}}[chapter]
\newtheorem{example}{\textsc{Example}}[chapter]

% Macros for title, author, abstract, etc.
\input{misc_metadata/preliminaries}

% Add PDF document info fields
\hypersetup{
	pdftitle={\Thesistitle},
	pdfauthor={\Authorname},
	pdfsubject={\Degreefield},
}

% Uncomment the following to have numbered subsubsections (by default
% numbering goes only to subsections).
%\setcounter{secnumdepth}{4}


% Set this to only select a subset of the includes directives below.
% Very handy to speed up compilation if you're working on a certain
% part of your thesis. It conserves page numbers, references, etc.
% even for non-included files.

%% commands
\newcommand{\SUewk}{$\mathcal{SU}(2)_{L} \times \mathcal{U}(1)_{Y}$}
\newcommand*{\Uone}{$\mathcal{U}(1)$}
\newcommand{\SUtwo}{$\mathcal{SU}(2)$}
\newcommand{\SUthree}{$\mathcal{SU}(3)$}

\newcommand{\SML}{$\mathcal{L}_{\text{SM}}$}
\newcommand{\fieldQi}{$Q_i$}
\newcommand{\fieldUri}{$u_{\text{R},i}$}
\newcommand{\fieldDri}{$d_{\text{R},i}$}
\newcommand{\fieldLi}{$L_i$}
\newcommand{\fieldEri}{$e_{\text{R},i}$}
\newcommand{\fieldB}{$B$}
\newcommand{\fieldW}{$W$}
\newcommand{\fieldWone}{$W_1$}
\newcommand{\fieldWtwo}{$W_2$}
\newcommand{\fieldWthree}{$W_3$}
\newcommand{\fieldWp}{$W^+$}
\newcommand{\fieldWm}{$W^-$}
\newcommand{\fieldWzero}{$W^0$}
\newcommand{\fieldWpm}{$W^{\pm}$}
\newcommand{\fieldZ}{$Z$}
\newcommand{\fieldZzero}{$Z^0$}
\newcommand{\fieldPhoton}{$\gamma$}
\newcommand{\fieldG}{$G$}
\newcommand{\quarkU}{$u$}
\newcommand{\quarkD}{$d$}
\newcommand{\quarkC}{$c$}
\newcommand{\quarkS}{$s$}
\newcommand{\quarkT}{$t$}
\newcommand{\quarkB}{$b$}
\newcommand{\leptonE}{$e$}
\newcommand{\leptonMu}{$\mu$}
\newcommand{\leptonTau}{$\tau$}
\newcommand{\neutrinoE}{$\nu_e$}
\newcommand{\neutrinoMu}{$\nu_{\mu}$}
\newcommand{\neutrinoTau}{$\nu_{\tau}$}
\newcommand{\fieldUl}{$u_{\text{L}}$}
\newcommand{\fieldDl}{$d_{\text{L}}$}
\newcommand{\fieldCl}{$c_{\text{L}}$}
\newcommand{\fieldSl}{$s_{\text{L}}$}
\newcommand{\fieldTl}{$t_{\text{L}}$}
\newcommand{\fieldBl}{$b_{\text{L}}$}
\newcommand{\fieldUr}{$u_{\text{R}}$}
\newcommand{\fieldDr}{$d_{\text{R}}$}
\newcommand{\fieldCr}{$c_{\text{R}}$}
\newcommand{\fieldSr}{$s_{\text{R}}$}
\newcommand{\fieldTr}{$t_{\text{R}}$}
\newcommand{\fieldBr}{$b_{\text{R}}$}
\newcommand{\fieldEl}{$e_{\text{L}}$}
\newcommand{\fieldMul}{$\mu_{\text{L}}$}
\newcommand{\fieldTaul}{$\tau_{\text{L}}$}
\newcommand{\fieldEr}{$e_{\text{R}}$}
\newcommand{\fieldMur}{$\mu_{\text{R}}$}
\newcommand{\fieldTaur}{$\tau_{\text{R}}$}
\newcommand{\fieldNuEl}{$\nu_{e,\text{L}}$}
\newcommand{\fieldNuMul}{$\nu_{\mu,\text{L}}$}
\newcommand{\fieldNuTaul}{$\nu_{\tau,\text{L}}$}
\newcommand{\fieldNuR}{$\nu_{\text{R}}$}
\newcommand{\fieldPhi}{$\mathcal{\phi}$}
\newcommand{\fieldPhip}{$\mathcal{\phi}^+$}
\newcommand{\fieldPhizero}{$\mathcal{\phi}^0$}
\newcommand{\fieldH}{$h$}
\newcommand*{\TeV}{\ensuremath{\text{Te\kern -0.1em V}}}
\newcommand*{\GeV}{\ensuremath{\text{Ge\kern -0.1em V}}}
\newcommand*{\MeV}{\ensuremath{\text{Me\kern -0.1em V}}}
\newcommand*{\pT}{\ensuremath{p_{T}}}
\newcommand*{\micron}{\ensuremath{\mu m}}

\usepackage{xspace}
\newcommand*{\ptmiss}{\ensuremath{\mathbf{p}_{\text{T}}^{\text{miss}}}\xspace}
\newcommand*{\met}{\ensuremath{E_{\text{T}}^{\text{miss}}}\xspace}
\newcommand*{\antikt}{\ensuremath{\text{anti-}k_t}\xspace}
\newcommand*{\npv}{\ensuremath{N_{\text{PV}}}\xspace}

\DeclareMathAlphabet\mathbfcal{OMS}{cmsy}{b}{n}

\begin{document}
% Preliminary pages are always loaded (TOC, CV, etc.})
%\preliminarypage}s

% if doing minimal compilation just add the table of contents here, otherwise use the "\preliminarypages" command above
\tableofcontents

% set the linespacing for the internal text
\onehalfspacing

% Include the different components of your thesis, in separate files.
% Using \include allows you to set \includeonly above.
%\chapter{The Standard Model of Particle Physics}

%\epigraph{\textit{So it goes...}}{---Kurt Vonnegut, \textit{Slaughterhouse
%		Five}}
	
%\epigraph{\textit{Science is a miracle.}}{--Ron Swanson}

\epigraph{\textit{If you wish to make an apple pie from scratch, you must first invent the universe.}}{--Carl Sagan, \textit{Cosmos: A Personal Voyage}}


As it stands, what has become known as the `Standard Model (SM) of Particle Physics'
is nothing less than one of the greatest achievments of mankind, due to both
the magnitude by which it has changed our perception of the underlying
nature of the universe and to the clever methods and tinkerings by which this
nature was unveiled by many clever physicists whose history has become veritable lore.
In terms of imagination and insight, it is second only to the special and general theories of relativity --
though the fields are nevertheless intricately intertwined.
%{\color{red}{The latter, though, being put forth by essentially a single person and the latter by a great many...}}.

Not considering the scientific progress made in the $18^{th}$ and $19^{th}$ centuries, and
ignoring the ancient Greeks despite their fabled invention of atomic theory,
the physical insights and major work that led to the current picture of elementary particle
physics described by the SM began with the \textit{annus mirabilis} papers of Albert
Einstein in the year 1905~\cite{einsteinPEE,einsteinSpecial,einsteinEnergyMass}.
In these papers, Einstein was able to shed light on the quantization of electromagnetic
radiation (building off of the seminal work of Max Planck~\cite{planckBlackBody})
and introduce the special theory of relativity.
These works laid the conceptual
and philosophical groundwork for the major breakthroughs in fundamental physics
of $20^{th}$ century physics: from the `old quantum theory' of Bohr and Sommerfeld
in the early 1900's to the equivalent wavefunction and matrix-mechanics formulations
of Schr{\"o}dinger and Heisenberg that
coalesced into `modern' quantum mechanics in the mid 1920's.
The modern approach, non-relativistic at its heart, provided a sufficient mathematical
and interpretable framework in which to work and match predictions to observed phenomena, old
and new. It has for the most part remained unchanged and is the quantum mechanics that is taught to
students at both the undergraduate and graduate level to this very day.
It is the theory that has since revolutionised all aspects of the physical sciences and
technologies that dictate our everyday-lives.
In the mid-1920's, however, despite
large efforts put forth by the forbears of modern quantum mechanics, the quantum-mechanical
world had yet to be made consistent with Einstein's theory of relativity --- a requirement
that must be met for all consistent physical theories of nature.
It was the insight of Paul Dirac who was finally able to successfully
marry the theory of the quantum with that of relativity when he introduced
his relativistic quantum-mechanical treatment of the electron in 1927 and 1928~\cite{diracEquation,Dirac:1927dy}.\footnote{
A complete history of the people and ideas involved in the development of the modern
theory of Quantum Mechanics can be found in references ~\cite{boffiRiseOfQM,historyQM},
and the references therein.
}
This work provided the starting point for a decades-long search of a consistent quantum-mechanical
and relativistic treatment of electrodynamics, known as \textit{quantum electrodynamics} (QED).
The search for QED ended at the end of the 1940's with the groundbreaking work of Dyson, Feynman, Schwinger, and Tomanaga~\cite{qedTomonaga,qedFeynman0,qedFeynman1,qedFeynman2,qedSchwinger0,qedSchwinger1,qedDyson0,qedDyson1} that introduced the covariant and gauge invariant
formulation of QED --- the first such relativistic quantum field theory (QFT).
QED allowed the phsycists to make predictions that agreed with observation to unprecedented levels
of accuracy and has since led to the adoption of its language and mathematical toolkit as the
foundational framework in which to construct models that accurately describe nature.\footnote{
	For a complete discussion of the developments leading up to QED, see the fabulous
	book by S. Schweber~\cite{Schweber:1994qa}.	
}
The SM is no less than an ultimate conclusion of these works: a consistent set of relativistic
quantum field theories, using the language developed by Feynman et al.,
that describes essentially all aspects of the known particles and forces that make up the 
observed universe.


\section{Particles and Forces}

Here we introduce the SM particle content and provide a description of the interactions that
link the particles together.


\begin{table}[!htb]
    \caption{
        The particle content of the SM and their transformation
        properties under the SM gauge groups, prior to electroweak symmetry breaking.
        The representations of each of the gauge groups are shown in the three-right
        columns. The \Uone symmetry of weak-hypercharge transformations is one-dimensional
        and the column gives the weak-hypercharge $\mathcal{Y}$ associated with each
        field. For \SUthree and \SUtwo, $\mathbf{1}$ refers to the field belonging to
        the associated singlet representation, $\mathbf{2}$ to the doublet representation,
        $\mathbf{3}$ to the triplet representation, and $\mathbf{8}$ to the octet representation.
    }
    \begin{center}
        \begin{tabularx}{0.96\textwidth}{m{1em} c c c c c c }
        \toprule
        \hline
        & Field Label & Content & Spin & \Uone~($\mathcal{=Y}$) & \SUtwo & \SUthree \\
        \hline
        \rotatebox{90}{\hspace{-0.1cm}\textbf{Quarks} } 
         &   \makecell{\fieldQi \\ \fieldUri \\ \fieldDri} % FIELD
         &   \makecell{ (\fieldUl, \fieldDl), (\fieldCl, \fieldSl), (\fieldTl, \fieldBl) \\ \fieldUr \\ \fieldDr}% CONTENT
         &   \makecell{ $1/2$ \\ $1/2$ \\ $1/2$} % SPIN
         &   \makecell{ $1/6$ \\ $2/3$ \\ $-1/3$}% U(1)
         &   \makecell{ $\mathbf{2}$ \\ $\mathbf{1}$ \\ $\mathbf{1}$}% SU(2)
         &   \makecell{ $\mathbf{3}$ \\ $\mathbf{3}$ \\ $\mathbf{3}$}\\ % SU(3)
        %\cdashline{1-7}
        \rotatebox{90}{\hspace{-0.1cm}\textbf{Leptons} }
         &   \makecell{\fieldLi \\ \fieldEri} % FIELD
         &   \makecell{ (\fieldEl, \fieldNuEl), (\fieldMul, \fieldNuMul), (\fieldTaul, \fieldNuTaul) \\ \fieldEr, \fieldMur, \fieldTaur}% CONTENT
         &   \makecell{ $1/2$ \\ $1/2$ }% SPIN
         &   \makecell{ $1/2$ \\ $-1$ }% U(1)
         &   \makecell{ $\mathbf{2}$ \\ $\mathbf{1}$ }% SU(2)
         &   \makecell{ $\mathbf{1}$ \\ $\mathbf{1}$ } \\ % SU(3)
        \midrule
        \rotatebox{90}{\textbf{\stackanchor{Gauge}{Fields}} }
         &   \makecell{\fieldB \\ \fieldW \\ \fieldG } % FIELD
         &   \makecell{ \fieldB \\ (\fieldWone, \fieldWtwo, \fieldWthree) \\ \fieldG$_a$, $a\in[1,..,8]$ }% CONTENT
         &   \makecell{ $1$ \\ $1$ \\ $1$} % SPIN
         &   \makecell{ $0$ \\ $0$ \\ $0$}% U(1)
         &   \makecell{ $\mathbf{1}$ \\ $\mathbf{3}$ \\ $\mathbf{1}$}% SU(2)
         &   \makecell{ $\mathbf{1}$ \\ $\mathbf{1}$ \\ $\mathbf{8}$}\\ % SU(3)
        \midrule
        \rotatebox{90}{\textbf{\stackanchor{Higgs}{Field}}} 
         &   \makecell{\fieldPhi } % FIELD
         &   \makecell{ (\fieldPhip, \fieldPhizero) }% CONTENT
         &   \makecell{ $0$  } % SPIN
         &   \makecell{ $1/2$  }% U(1)
         &   \makecell{ $\mathbf{2}$ }% SU(2)
         &   \makecell{ $\mathbf{1}$ }\\ % SU(3)
        \hline
        \bottomrule
        \end{tabularx}
    \end{center}
    \label{tab:sm_content}
\end{table}
\floatbarrier


\begin{table}[!htb]
    \caption{
        The particle content of the SM after the process of
        electroweak symmetry breaking.
    }
    \begin{center}
        \begin{tabularx}{1\textwidth}{m{1em} c c c c }
        \toprule
        \hline
        & Physical Field & Q & Coupling & Mass [GeV] \\
        \hline
        \rotatebox{90}{\hspace{-0.1cm}\textbf{Quarks} } 
            & \makecell{ \quarkU, \quarkC, \quarkT \\ \quarkD, \quarkS, \quarkB} % FIELD
            & \makecell{ $2/3$ \\ $-1/3$ }% Q
            %& \makecell{ $\mathbf{3}$ \\ $\mathbf{3}$ } % SU(3)
            & \makecell{ ($y_i=$) $1\times10^{-5}$, $7\times10^{-3}$, $1$ \\ ($y_i=$) $3\times10^{-5}$, $5\times10^{-4}$, $0.02$ } % Coupling
            & \makecell{ $2\times10^{-3}$, $1.27$, $173$ \\ $4\times10^{-4}$, $0.10$, $4.18$ }\\% Mass
        \rotatebox{90}{\hspace{-0.1cm}\textbf{Leptons} } 
            & \makecell{ \leptonE, \leptonMu, \leptonTau \\ \neutrinoE, \neutrinoMu, \neutrinoTau } % FIELD
            & \makecell{ $-1$ \\ $0$ }% Q
            %& \makecell{ $\mathbf{1}$ \\ $\mathbf{1}$ } % SU(3)
            & \makecell{ ($y_i=$) $3\times10^{-7}$, $6\times10^{-4}$, $0.01$ \\ -- } % Coupling
            & \makecell{ $5\times10^{-4}$, $0.106$, $1.777$ \\ --}\\% Mass
        \midrule
        \rotatebox{90}{\textbf{Bosons} } 
            & \makecell{ \fieldPhoton \\ \fieldZ \\ (\fieldWp, \fieldWm) \\ \fieldG } % FIELD
            & \makecell{ $0$ \\ $0$ \\ $(+1,-1)$ \\ $0$ }% Q
            %& \makecell{ $\mathbf{1}$ \\ $\mathbf{1}$ \\ $\mathbf{1}$ \\ $\mathbf{8}$ } % SU(3)
            & \makecell{ $\alpha_{\text{EM}} \simeq 1/137$ \\ $\sin \theta_{W} \simeq 0.5$ \\ -- \\ $\alpha_s \simeq 0.1$ } % Coupling
            & \makecell{ $0$ \\ $91.2$ \\ $80.4$ \\  $0$}\\% Mass
        \midrule
        \rotatebox{90}{\textbf{Higgs} } 
            & \makecell{ \fieldH } % FIELD
            & \makecell{ $0$ }% Q
            %& \makecell{ $\mathbf{1}$ } % SU(3)
            & \makecell{ $\lambda$, $\mu$ } % Coupling
            & \makecell{ $125.09$ }\\% Mass
        \hline
        \bottomrule
        \end{tabularx}
    \end{center}
    \label{tab:sm_content}
\end{table}



\subsection{Gauge Theories}

\subsubsection{The Electroweak Theory}



%\chapter{Experimental Setup}

%\epigraph{\textit{So it goes...}}{---Kurt Vonnegut, \textit{Slaughterhouse
%		Five}}
	
%\epigraph{\textit{Science is a miracle.}}{--Ron Swanson}

%\epigraph{\textit{If you wish to make an apple pie from scratch, you must first invent the universe.}}{--Carl Sagan, \textit{Cosmos: A Personal Voyage}}
\epigraph{\textit{Nice piece of wood in that counter. Nicely planed. Like the way it curves there.}}{--Leopold Bloom, in James Joyce's \textit{Ulysses}}
%\epigraph{\textit{The movements which work revolutions in the world are born
%out of the dreams and visions in a peasant's heart on the hillside.}}{--``Leopold Bloom'', in \textit{Ulysses} by James Joyce}

The work to be described in the present thesis was done at CERN\footnote{
The acronym CERN was historically derived from `\textit{Conseil europ{\'e}en pour la recherche
nucl{\'e}aire'}. Nowadays, `CERN' has become a standalone name for the lab itself and
is currently referred to as the `\textit{Organisation europ{\'e}enne pour la recherche nucl{\'e}aire}'; or, in English: the
`\textit{European Organisation for Nuclear Research.}'}, the particle
physics laboratory located along the French-Swiss border just outside of Geneva, Switzerland.
CERN is comprised of almost 18,000 personnel, of which over 13,000 are researchers in the
field of experimental particle physics.
It is a truly international workplace, with the personnel comprised of representatives of over 110 nationalities
and who are either working directly
for CERN\footnote{Of the roughly 18,000 researchers in experimental particle physics, only about
5\% are employed directly by CERN itself.} or for their respective home institutions
--- universities or national labs ---
located in more than 70 countries~\cite{CERN-HR-STAFF-STAT-2018}.
These researchers will generally work at any of the independent experiments located along the various
beamlines that network throughout the CERN campus (see Fig.~\ref{fig:cern_complex}).

As the present author is a member of one of the two general-purpose experiments at CERN located
along the Large Hadron Collider (LHC) -- the ATLAS experiment -- this chapter will present a
brief introduction to the workings of the LHC (Section~\ref{sec:lhc}) and then describe in some
detail the various components that make up the ATLAS detector (Section~\ref{sec:atlas}), the largest
and most complex scientific piece of equipment ever 
constructed by humans.\footnote{The ATLAS detector, along with its operation, is by far more complex
than any previous human endeavour --- generally more complex than anything operated and enacted by NASA, for
example. The only difference being the tolerance for failure: in the case of NASA space-based experiments and missions
this tolerance approaches zero, whereas the terrestrial particle physics experiments happening at the
LHC are generally accessible and amenable to errors.}


\begin{figure}[!htb]
    \begin{center}
        \includegraphics[width=0.8\textwidth]{figures/chapter2/cern_accelerator_complex2}
        \caption{
            Illustration of the various beamlines, accelerator and storage rings, and experimental
            points that the CERN accelerator complex is home to.
            The protons that circulate through the LHC, and that are eventually made to collide inside
            the ATLAS detector, follow the path: Linac 2 $\rightarrow$ Booster $\rightarrow$ Proton Synchotron (PS)
            $\rightarrow$ Super Proton Synchotron (SPS) $\rightarrow$ LHC.
        }
        \label{fig:cern_complex}
    \end{center}
\end{figure}


%%%%%%%%%%%%%%%%%%%%%%%%%%%%%%%%%%%%%%%%%%%%%%%%%%%%%%%%%%%%%%%%%%%
%%%%%%%%%%%%%%%%%%%%%%%%%%%%%%%%%%%%%%%%%%%%%%%%%%%%%%%%%%%%%%%%%%%
% sub-section describing the LHC
%%%%%%%%%%%%%%%%%%%%%%%%%%%%%%%%%%%%%%%%%%%%%%%%%%%%%%%%%%%%%%%%%%%
%%%%%%%%%%%%%%%%%%%%%%%%%%%%%%%%%%%%%%%%%%%%%%%%%%%%%%%%%%%%%%%%%%%
\section{The Large Hadron Collider}
\label{sec:lhc}

The LHC~\cite{Evans_2008} is a circular particle accelerator with a 27~kilometer ($\approx17$ miles)
circumference located, on average, approximately 100 meters beneath the Earth's surface. It is nominally
a proton-proton ($pp$) collider
but can also be run in heavy-ion configurations: proton-lead ($p$-Pb), lead-lead (Pb-Pb), or even
proton-gold ($p$-Au). It is designed to accelerate protons to a center-of-mass
energy of $\sqrt{s} = 14\,\TeV$.

To avoid the exorbitant costs in civil engineering and real-estate works associated with
constructing an even larger tunnel, it was decided that the LHC should be housed in the already-existing
tunnel that housed the Large Electron Positron (LEP) collider, in operation from 1989 to 2000.
LEP, a \textit{particle-antiparticle} collider, was able to take advantage of the fact that
 particle and anti-particle beams can be made to occupy the same phase space within a single ring: the same magnetic
fields could produce counter-rotating electron (negatively charged) and positron (positvely charged) beams.



\begin{figure}[!htb]
    \begin{center}
        \includegraphics[width=0.8\textwidth]{figures/chapter2/lhc_layout}
        \caption{
            Layout of the LHC and its two counter-rotating beams. Beam 1 is in blue and rotates
            counter-clockwise. Beam 2 is in red and rotates clock-wise.
            At the center of each octant is a straight section which houses
            the experimental caverns or LHC beam facilities.
            At the boundaries of each octant are located the curved sections.
            Figure taken from Figure 2.1 of Ref.~\cite{Evans_2008}.
        }
        \label{fig:lhc_layout}
    \end{center}
\end{figure}

\begin{figure}[!htb]
    \begin{center}
        \includegraphics[width=0.5\textwidth]{figures/chapter2/lhc_dipole_fig3p3}
        \caption{
        }
        \label{fig:lhc_dipole_xsec}
    \end{center}
\end{figure}

\subsection{Injection Chain}
\label{sec:lhc_injection}

\subsection{The Concept of Luminosity}
\label{sec:lhc_luminosity}

The Large Hadron Collider (LHC) can be thought of as the final part of the particle-beam injection line
that is comprised of many parts whose goal is to accelerate protons, or other particles, to
the energies requisite for CERN's large experiments to do perform fundamental physics research
at the high-energy frontier.


%%%%%%%%%%%%%%%%%%%%%%%%%%%%%%%%%%%%%%%%%%%%%%%%%%%%%%%%%%%%%%%%%%%
%%%%%%%%%%%%%%%%%%%%%%%%%%%%%%%%%%%%%%%%%%%%%%%%%%%%%%%%%%%%%%%%%%%
% sub-section describing ATLAS
%%%%%%%%%%%%%%%%%%%%%%%%%%%%%%%%%%%%%%%%%%%%%%%%%%%%%%%%%%%%%%%%%%%
%%%%%%%%%%%%%%%%%%%%%%%%%%%%%%%%%%%%%%%%%%%%%%%%%%%%%%%%%%%%%%%%%%%
\section{The ATLAS Detector}
\label{sec:atlas}


% ... and so on

% if doing minimal compilation, set the page counter to start at the document meat -- otherwise use the "\preliminarypages" command above
\setcounter{page}{0}
%\chapter{The Standard Model of Particle Physics}

%\epigraph{\textit{So it goes...}}{---Kurt Vonnegut, \textit{Slaughterhouse
%		Five}}
	
%\epigraph{\textit{Science is a miracle.}}{--Ron Swanson}

\epigraph{\textit{If you wish to make an apple pie from scratch, you must first invent the universe.}}{--Carl Sagan, \textit{Cosmos: A Personal Voyage}}


As it stands, what has become known as the `Standard Model (SM) of Particle Physics'
is nothing less than one of the greatest achievments of mankind, due to both
the magnitude by which it has changed our perception of the underlying
nature of the universe and to the clever methods and tinkerings by which this
nature was unveiled by many clever physicists whose history has become veritable lore.
In terms of imagination and insight, it is second only to the special and general theories of relativity --
though the fields are nevertheless intricately intertwined.
%{\color{red}{The latter, though, being put forth by essentially a single person and the latter by a great many...}}.

Not considering the scientific progress made in the $18^{th}$ and $19^{th}$ centuries, and
ignoring the ancient Greeks despite their fabled invention of atomic theory,
the physical insights and major work that led to the current picture of elementary particle
physics described by the SM began with the \textit{annus mirabilis} papers of Albert
Einstein in the year 1905~\cite{einsteinPEE,einsteinSpecial,einsteinEnergyMass}.
In these papers, Einstein was able to shed light on the quantization of electromagnetic
radiation (building off of the seminal work of Max Planck~\cite{planckBlackBody})
and introduce the special theory of relativity.
These works laid the conceptual
and philosophical groundwork for the major breakthroughs in fundamental physics
of $20^{th}$ century physics: from the `old quantum theory' of Bohr and Sommerfeld
in the early 1900's to the equivalent wavefunction and matrix-mechanics formulations
of Schr{\"o}dinger and Heisenberg that
coalesced into `modern' quantum mechanics in the mid 1920's.
The modern approach, non-relativistic at its heart, provided a sufficient mathematical
and interpretable framework in which to work and match predictions to observed phenomena, old
and new. It has for the most part remained unchanged and is the quantum mechanics that is taught to
students at both the undergraduate and graduate level to this very day.
It is the theory that has since revolutionised all aspects of the physical sciences and
technologies that dictate our everyday-lives.
In the mid-1920's, however, despite
large efforts put forth by the forbears of modern quantum mechanics, the quantum-mechanical
world had yet to be made consistent with Einstein's theory of relativity --- a requirement
that must be met for all consistent physical theories of nature.
It was the insight of Paul Dirac who was finally able to successfully
marry the theory of the quantum with that of relativity when he introduced
his relativistic quantum-mechanical treatment of the electron in 1927 and 1928~\cite{diracEquation,Dirac:1927dy}.\footnote{
A complete history of the people and ideas involved in the development of the modern
theory of Quantum Mechanics can be found in references ~\cite{boffiRiseOfQM,historyQM},
and the references therein.
}
This work provided the starting point for a decades-long search of a consistent quantum-mechanical
and relativistic treatment of electrodynamics, known as \textit{quantum electrodynamics} (QED).
The search for QED ended at the end of the 1940's with the groundbreaking work of Dyson, Feynman, Schwinger, and Tomanaga~\cite{qedTomonaga,qedFeynman0,qedFeynman1,qedFeynman2,qedSchwinger0,qedSchwinger1,qedDyson0,qedDyson1} that introduced the covariant and gauge invariant
formulation of QED --- the first such relativistic quantum field theory (QFT).
QED allowed the phsycists to make predictions that agreed with observation to unprecedented levels
of accuracy and has since led to the adoption of its language and mathematical toolkit as the
foundational framework in which to construct models that accurately describe nature.\footnote{
	For a complete discussion of the developments leading up to QED, see the fabulous
	book by S. Schweber~\cite{Schweber:1994qa}.	
}
The SM is no less than an ultimate conclusion of these works: a consistent set of relativistic
quantum field theories, using the language developed by Feynman et al.,
that describes essentially all aspects of the known particles and forces that make up the 
observed universe.


\section{Particles and Forces}

Here we introduce the SM particle content and provide a description of the interactions that
link the particles together.


\begin{table}[!htb]
    \caption{
        The particle content of the SM and their transformation
        properties under the SM gauge groups, prior to electroweak symmetry breaking.
        The representations of each of the gauge groups are shown in the three-right
        columns. The \Uone symmetry of weak-hypercharge transformations is one-dimensional
        and the column gives the weak-hypercharge $\mathcal{Y}$ associated with each
        field. For \SUthree and \SUtwo, $\mathbf{1}$ refers to the field belonging to
        the associated singlet representation, $\mathbf{2}$ to the doublet representation,
        $\mathbf{3}$ to the triplet representation, and $\mathbf{8}$ to the octet representation.
    }
    \begin{center}
        \begin{tabularx}{0.96\textwidth}{m{1em} c c c c c c }
        \toprule
        \hline
        & Field Label & Content & Spin & \Uone~($\mathcal{=Y}$) & \SUtwo & \SUthree \\
        \hline
        \rotatebox{90}{\hspace{-0.1cm}\textbf{Quarks} } 
         &   \makecell{\fieldQi \\ \fieldUri \\ \fieldDri} % FIELD
         &   \makecell{ (\fieldUl, \fieldDl), (\fieldCl, \fieldSl), (\fieldTl, \fieldBl) \\ \fieldUr \\ \fieldDr}% CONTENT
         &   \makecell{ $1/2$ \\ $1/2$ \\ $1/2$} % SPIN
         &   \makecell{ $1/6$ \\ $2/3$ \\ $-1/3$}% U(1)
         &   \makecell{ $\mathbf{2}$ \\ $\mathbf{1}$ \\ $\mathbf{1}$}% SU(2)
         &   \makecell{ $\mathbf{3}$ \\ $\mathbf{3}$ \\ $\mathbf{3}$}\\ % SU(3)
        %\cdashline{1-7}
        \rotatebox{90}{\hspace{-0.1cm}\textbf{Leptons} }
         &   \makecell{\fieldLi \\ \fieldEri} % FIELD
         &   \makecell{ (\fieldEl, \fieldNuEl), (\fieldMul, \fieldNuMul), (\fieldTaul, \fieldNuTaul) \\ \fieldEr, \fieldMur, \fieldTaur}% CONTENT
         &   \makecell{ $1/2$ \\ $1/2$ }% SPIN
         &   \makecell{ $1/2$ \\ $-1$ }% U(1)
         &   \makecell{ $\mathbf{2}$ \\ $\mathbf{1}$ }% SU(2)
         &   \makecell{ $\mathbf{1}$ \\ $\mathbf{1}$ } \\ % SU(3)
        \midrule
        \rotatebox{90}{\textbf{\stackanchor{Gauge}{Fields}} }
         &   \makecell{\fieldB \\ \fieldW \\ \fieldG } % FIELD
         &   \makecell{ \fieldB \\ (\fieldWone, \fieldWtwo, \fieldWthree) \\ \fieldG$_a$, $a\in[1,..,8]$ }% CONTENT
         &   \makecell{ $1$ \\ $1$ \\ $1$} % SPIN
         &   \makecell{ $0$ \\ $0$ \\ $0$}% U(1)
         &   \makecell{ $\mathbf{1}$ \\ $\mathbf{3}$ \\ $\mathbf{1}$}% SU(2)
         &   \makecell{ $\mathbf{1}$ \\ $\mathbf{1}$ \\ $\mathbf{8}$}\\ % SU(3)
        \midrule
        \rotatebox{90}{\textbf{\stackanchor{Higgs}{Field}}} 
         &   \makecell{\fieldPhi } % FIELD
         &   \makecell{ (\fieldPhip, \fieldPhizero) }% CONTENT
         &   \makecell{ $0$  } % SPIN
         &   \makecell{ $1/2$  }% U(1)
         &   \makecell{ $\mathbf{2}$ }% SU(2)
         &   \makecell{ $\mathbf{1}$ }\\ % SU(3)
        \hline
        \bottomrule
        \end{tabularx}
    \end{center}
    \label{tab:sm_content}
\end{table}
\floatbarrier


\begin{table}[!htb]
    \caption{
        The particle content of the SM after the process of
        electroweak symmetry breaking.
    }
    \begin{center}
        \begin{tabularx}{1\textwidth}{m{1em} c c c c }
        \toprule
        \hline
        & Physical Field & Q & Coupling & Mass [GeV] \\
        \hline
        \rotatebox{90}{\hspace{-0.1cm}\textbf{Quarks} } 
            & \makecell{ \quarkU, \quarkC, \quarkT \\ \quarkD, \quarkS, \quarkB} % FIELD
            & \makecell{ $2/3$ \\ $-1/3$ }% Q
            %& \makecell{ $\mathbf{3}$ \\ $\mathbf{3}$ } % SU(3)
            & \makecell{ ($y_i=$) $1\times10^{-5}$, $7\times10^{-3}$, $1$ \\ ($y_i=$) $3\times10^{-5}$, $5\times10^{-4}$, $0.02$ } % Coupling
            & \makecell{ $2\times10^{-3}$, $1.27$, $173$ \\ $4\times10^{-4}$, $0.10$, $4.18$ }\\% Mass
        \rotatebox{90}{\hspace{-0.1cm}\textbf{Leptons} } 
            & \makecell{ \leptonE, \leptonMu, \leptonTau \\ \neutrinoE, \neutrinoMu, \neutrinoTau } % FIELD
            & \makecell{ $-1$ \\ $0$ }% Q
            %& \makecell{ $\mathbf{1}$ \\ $\mathbf{1}$ } % SU(3)
            & \makecell{ ($y_i=$) $3\times10^{-7}$, $6\times10^{-4}$, $0.01$ \\ -- } % Coupling
            & \makecell{ $5\times10^{-4}$, $0.106$, $1.777$ \\ --}\\% Mass
        \midrule
        \rotatebox{90}{\textbf{Bosons} } 
            & \makecell{ \fieldPhoton \\ \fieldZ \\ (\fieldWp, \fieldWm) \\ \fieldG } % FIELD
            & \makecell{ $0$ \\ $0$ \\ $(+1,-1)$ \\ $0$ }% Q
            %& \makecell{ $\mathbf{1}$ \\ $\mathbf{1}$ \\ $\mathbf{1}$ \\ $\mathbf{8}$ } % SU(3)
            & \makecell{ $\alpha_{\text{EM}} \simeq 1/137$ \\ $\sin \theta_{W} \simeq 0.5$ \\ -- \\ $\alpha_s \simeq 0.1$ } % Coupling
            & \makecell{ $0$ \\ $91.2$ \\ $80.4$ \\  $0$}\\% Mass
        \midrule
        \rotatebox{90}{\textbf{Higgs} } 
            & \makecell{ \fieldH } % FIELD
            & \makecell{ $0$ }% Q
            %& \makecell{ $\mathbf{1}$ } % SU(3)
            & \makecell{ $\lambda$, $\mu$ } % Coupling
            & \makecell{ $125.09$ }\\% Mass
        \hline
        \bottomrule
        \end{tabularx}
    \end{center}
    \label{tab:sm_content}
\end{table}



\subsection{Gauge Theories}

\subsubsection{The Electroweak Theory}



%\chapter{Experimental Setup}

%\epigraph{\textit{So it goes...}}{---Kurt Vonnegut, \textit{Slaughterhouse
%		Five}}
	
%\epigraph{\textit{Science is a miracle.}}{--Ron Swanson}

%\epigraph{\textit{If you wish to make an apple pie from scratch, you must first invent the universe.}}{--Carl Sagan, \textit{Cosmos: A Personal Voyage}}
\epigraph{\textit{Nice piece of wood in that counter. Nicely planed. Like the way it curves there.}}{--Leopold Bloom, in James Joyce's \textit{Ulysses}}
%\epigraph{\textit{The movements which work revolutions in the world are born
%out of the dreams and visions in a peasant's heart on the hillside.}}{--``Leopold Bloom'', in \textit{Ulysses} by James Joyce}

The work to be described in the present thesis was done at CERN\footnote{
The acronym CERN was historically derived from `\textit{Conseil europ{\'e}en pour la recherche
nucl{\'e}aire'}. Nowadays, `CERN' has become a standalone name for the lab itself and
is currently referred to as the `\textit{Organisation europ{\'e}enne pour la recherche nucl{\'e}aire}'; or, in English: the
`\textit{European Organisation for Nuclear Research.}'}, the particle
physics laboratory located along the French-Swiss border just outside of Geneva, Switzerland.
CERN is comprised of almost 18,000 personnel, of which over 13,000 are researchers in the
field of experimental particle physics.
It is a truly international workplace, with the personnel comprised of representatives of over 110 nationalities
and who are either working directly
for CERN\footnote{Of the roughly 18,000 researchers in experimental particle physics, only about
5\% are employed directly by CERN itself.} or for their respective home institutions
--- universities or national labs ---
located in more than 70 countries~\cite{CERN-HR-STAFF-STAT-2018}.
These researchers will generally work at any of the independent experiments located along the various
beamlines that network throughout the CERN campus (see Fig.~\ref{fig:cern_complex}).

As the present author is a member of one of the two general-purpose experiments at CERN located
along the Large Hadron Collider (LHC) -- the ATLAS experiment -- this chapter will present a
brief introduction to the workings of the LHC (Section~\ref{sec:lhc}) and then describe in some
detail the various components that make up the ATLAS detector (Section~\ref{sec:atlas}), the largest
and most complex scientific piece of equipment ever 
constructed by humans.\footnote{The ATLAS detector, along with its operation, is by far more complex
than any previous human endeavour --- generally more complex than anything operated and enacted by NASA, for
example. The only difference being the tolerance for failure: in the case of NASA space-based experiments and missions
this tolerance approaches zero, whereas the terrestrial particle physics experiments happening at the
LHC are generally accessible and amenable to errors.}


\begin{figure}[!htb]
    \begin{center}
        \includegraphics[width=0.8\textwidth]{figures/chapter2/cern_accelerator_complex2}
        \caption{
            Illustration of the various beamlines, accelerator and storage rings, and experimental
            points that the CERN accelerator complex is home to.
            The protons that circulate through the LHC, and that are eventually made to collide inside
            the ATLAS detector, follow the path: Linac 2 $\rightarrow$ Booster $\rightarrow$ Proton Synchotron (PS)
            $\rightarrow$ Super Proton Synchotron (SPS) $\rightarrow$ LHC.
        }
        \label{fig:cern_complex}
    \end{center}
\end{figure}


%%%%%%%%%%%%%%%%%%%%%%%%%%%%%%%%%%%%%%%%%%%%%%%%%%%%%%%%%%%%%%%%%%%
%%%%%%%%%%%%%%%%%%%%%%%%%%%%%%%%%%%%%%%%%%%%%%%%%%%%%%%%%%%%%%%%%%%
% sub-section describing the LHC
%%%%%%%%%%%%%%%%%%%%%%%%%%%%%%%%%%%%%%%%%%%%%%%%%%%%%%%%%%%%%%%%%%%
%%%%%%%%%%%%%%%%%%%%%%%%%%%%%%%%%%%%%%%%%%%%%%%%%%%%%%%%%%%%%%%%%%%
\section{The Large Hadron Collider}
\label{sec:lhc}

The LHC~\cite{Evans_2008} is a circular particle accelerator with a 27~kilometer ($\approx17$ miles)
circumference located, on average, approximately 100 meters beneath the Earth's surface. It is nominally
a proton-proton ($pp$) collider
but can also be run in heavy-ion configurations: proton-lead ($p$-Pb), lead-lead (Pb-Pb), or even
proton-gold ($p$-Au). It is designed to accelerate protons to a center-of-mass
energy of $\sqrt{s} = 14\,\TeV$.

To avoid the exorbitant costs in civil engineering and real-estate works associated with
constructing an even larger tunnel, it was decided that the LHC should be housed in the already-existing
tunnel that housed the Large Electron Positron (LEP) collider, in operation from 1989 to 2000.
LEP, a \textit{particle-antiparticle} collider, was able to take advantage of the fact that
 particle and anti-particle beams can be made to occupy the same phase space within a single ring: the same magnetic
fields could produce counter-rotating electron (negatively charged) and positron (positvely charged) beams.



\begin{figure}[!htb]
    \begin{center}
        \includegraphics[width=0.8\textwidth]{figures/chapter2/lhc_layout}
        \caption{
            Layout of the LHC and its two counter-rotating beams. Beam 1 is in blue and rotates
            counter-clockwise. Beam 2 is in red and rotates clock-wise.
            At the center of each octant is a straight section which houses
            the experimental caverns or LHC beam facilities.
            At the boundaries of each octant are located the curved sections.
            Figure taken from Figure 2.1 of Ref.~\cite{Evans_2008}.
        }
        \label{fig:lhc_layout}
    \end{center}
\end{figure}

\begin{figure}[!htb]
    \begin{center}
        \includegraphics[width=0.5\textwidth]{figures/chapter2/lhc_dipole_fig3p3}
        \caption{
        }
        \label{fig:lhc_dipole_xsec}
    \end{center}
\end{figure}

\subsection{Injection Chain}
\label{sec:lhc_injection}

\subsection{The Concept of Luminosity}
\label{sec:lhc_luminosity}

The Large Hadron Collider (LHC) can be thought of as the final part of the particle-beam injection line
that is comprised of many parts whose goal is to accelerate protons, or other particles, to
the energies requisite for CERN's large experiments to do perform fundamental physics research
at the high-energy frontier.


%%%%%%%%%%%%%%%%%%%%%%%%%%%%%%%%%%%%%%%%%%%%%%%%%%%%%%%%%%%%%%%%%%%
%%%%%%%%%%%%%%%%%%%%%%%%%%%%%%%%%%%%%%%%%%%%%%%%%%%%%%%%%%%%%%%%%%%
% sub-section describing ATLAS
%%%%%%%%%%%%%%%%%%%%%%%%%%%%%%%%%%%%%%%%%%%%%%%%%%%%%%%%%%%%%%%%%%%
%%%%%%%%%%%%%%%%%%%%%%%%%%%%%%%%%%%%%%%%%%%%%%%%%%%%%%%%%%%%%%%%%%%
\section{The ATLAS Detector}
\label{sec:atlas}


\chapter{Physics Building Blocks}
\label{chap:objects}

\section{Tracks and Vertices}

\section{Leptons}

\section{Jets}

\section{Ambiguity Solving}

\section{Missing Transverse Momentum}

%\chapter{The Phase-I New Small Wheel Upgrade Project}
\label{chap:nsw}

%\chapter{Physics Beyond The Standard Model}
\label{chap:bsm}

\section{Supersymmetry}

%\chapter{Common Elements in the Analysis of High Energy Physics Collision Data}
\label{chap:common_search}

\section{The Simulation of Standard Model Physics Processes}

\section{Statistics and Hypothesis Testing}

\section{The Control Region Method}

%\chapter{The Search for the Supersymmetric Top-quark}
\label{chap:search_stop}

%\chapter{The Search for the Non-resonant Production of Higgs Boson Pairs}
\label{chap:search_hh}

%\include{sections/chapter9}

% These commands fix an odd problem in which the bibliography line
% of the Table of Contents shows the wrong page number.
\clearpage
\phantomsection

% "References should be formatted in style most common in discipline",
% abbrv is only a suggestion.



% The Thesis Manual says not to include appendix figures and tables in
% the List of Figures and Tables, respectively, so these commands from
% the caption package turn it off from this point onwards. If needed,
% it can be re-enabled later (using list=yes argument).
\captionsetup[figure]{list=no}
\captionsetup[table]{list=no}


\addcontentsline{toc}{chapter}{Bibliography}
\printbibliography
%\bibliographystyle{unsrt}
%\bibliography{bib/references.bib}


% If you have an appendix, it should come after the references.
%\appendix

\chapter{Basics of Machine Learning}
\label{app:ml}



\end{document}
